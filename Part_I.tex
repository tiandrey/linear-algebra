% Я кладу какашку сюда
\chapter{Понятие матрицы}
\begin{opred}
\mybold{Матрицей размера $m\times{}n$} называют набор из $mn$ чисел, упорядоченых в прямоугольную таблицу, состоящую из m строк и n столбцов.
Эти числа - \mybold{элементы} этой матрицы.
\newlineЗаписывается это так: $A^{m\times{}n}=\left( a_{ij} \right) \in \mathbb {R}^{m\times{}n}$
\newlineЕсли m=n, то матрица --- \mybold{квадратная}.
\newlineДля матрицы A:
\newline$a'_i$ --- i-я строка
\newline$a_i$ --- i-й столбец
\newline$A_{m\times{}1}$ --- \mybold{вектор-столбец}
\newline$A_{1\times{}n}$ --- \mybold{вектор-строка}
\end{opred}
\begin{opred}
\mybold{Главная диагональ} --- совокупность элементов, расположенных в строках и столбцах с одинаковыми номерами.
\newlineЭти элементы --- \mybold{диагональные}.
\end{opred}
\begin{opred}
Матрица, в которой все элементы равны нулю, называется \mybold{нулевой} и обозначается $\Theta_{m\times{}n}$
\end{opred}
\section{Квадратные матрицы}
Квадратные матрицы делят на:
\begin{itemize}
\item Верхние треугольные
\item Нижние треугольные
\item Диагональные
\item Скалярные
\item Единичные
\end{itemize}
\begin{opred}
\mybold{Верхними треугольными} называются квадратные матрицы, в которых для $\forall{}i>j$ $a_{ij}=0$, то есть матрицы, в которых все элементы ниже главной диагонали --- нулевые:\newline
{$$
\begin{pmatrix}
a_{11}& a_{12}& \ldots& a_{1n} \\
0&      a_{22}& \ldots& a_{2n} \\
\vdots& \vdots& \ddots& \vdots \\
0&      0&      \ldots& a_{nn}
\end{pmatrix}
$$}
\end{opred}
\begin{opred}
\mybold{Нижними треугольными} называются квадратные матрицы, в которых для $\forall{}i<j$ $a_{ij}=0$, то есть матрицы, в которых все элементы выше главной диагонали --- нулевые:\newline
{$$
\begin{pmatrix}
a_{11}& 0&      \ldots& 0      \\
a_{21}& a_{22}& \ldots& 0      \\
\vdots& \vdots& \ddots& \vdots \\
a_{n1}& a_{n2}& \ldots& a_{nn}
\end{pmatrix}
$$}
\end{opred}
\begin{opred}
\mybold{Диагональными} называются квадратные матрицы, в которых для $\forall{}i\ne{}j$ $a_{ij}=0$, то есть матрицы, в которых все элементы, не лежащие на главной диагонали, нулевые:\newline
$$
\begin{pmatrix}
a_{11}& 0&      \ldots& 0      \\
0&      a_{22}& \ldots& 0      \\
\vdots& \vdots& \ddots& \vdots \\
0&      0&      \ldots& a_{nn}
\end{pmatrix}
$$
Диагональные матрицы обозначаются символом $\Lambda$, или $diag(a_{11},\ldots,a_{nn})$.
\end{opred}
\begin{opred}
\mybold{Скалярными} называются диагональные матрицы, в которых все диагональные элементы равны:\newline
$$
\begin{pmatrix}
\lambda & 0       & \ldots & 0      \\
0       & \lambda & \ldots & 0      \\
\vdots  & \vdots  & \ddots & \vdots \\
0       & 0       & \ldots & \lambda
\end{pmatrix}
$$
\end{opred}
\begin{opred}
\mybold{Единичными} называются скалярные матрицы, в которых диагональные элементы равны 1:\newline
$$
\begin{pmatrix}
1&     0&       \ldots& 0      \\
0&     1&       \ldots& 0      \\
\vdots& \vdots& \ddots& \vdots \\
0&      0&      \ldots& 1
\end{pmatrix}
$$
Единичные матрицы обозначаются символом $I=(e_1,\ldots,e_n)=\begin{pmatrix}
e'_1 \\
\vdots \\
e'_n\end{pmatrix}$.
\end{opred}
\section{Ступенчатые матрицы}
Матрицы любого размера принято делить на верхние и нижние ступенчатые.
\begin{opred}Признак \mybold{верхней ступенчатой матрицы}:
\newlineВ каждой строке отметим позицию, в которой находится первый ненулевой элемент.
\newline1) Если какая-либо строка нулевая, то все последующие строки тоже нулевые
\newline2) Местоположение первых ненулевых элементов каждой строки таково, что номера столбцов, в которых они располагаются, образуют возрастающую последовательность.
$$
A=\begin{pmatrix}
a_{11} & a_{12} & a_{13} & a_{14} & a_{15} \\
0      & 0      & a_{23} & a_{24} & a_{25} \\
0      & 0      & 0      & 0      & a_{35} \\
0      & 0      & 0      & 0      & 0      \\
0      & 0      & 0      & 0      & 0
\end{pmatrix}
$$
Если в этом определении поменять ролями строки и столбцы, то получится определение \mybold{нижней ступенчатой матрицы}.

Если в верхней ступенчатой матрице $a_{kj_k}$ (первые ненулевые элементы) таковы, что $j_k=k$, то это \mybold{верхняя трапецевидная матрица}.
$$
A=\begin{pmatrix}
a_{11} & a_{12} & a_{13} & a_{14} \\
0      & a_{22} & a_{23} & a_{24} \\
0      & 0      & a_{33} & a_{34} \\
0      & 0      & 0      & a_{44} \\
0      & 0      & 0      & 0
\end{pmatrix}
$$
\end{opred}
\section{Блочная (клеточная) структура}
$$
A=\begin{pmatrix}
a_{11} & a_{12} & a_{13} & a_{14} & a_{15} \\
a_{21} & a_{22} & a_{23} & a_{24} & a_{25} \\
a_{31} & a_{32} & a_{33} & a_{34} & a_{35} \\
a_{41} & a_{42} & a_{43} & a_{44} & a_{45}
\end{pmatrix}=\begin{BMAT}(b,20pt,15pt){|cc;cc;c|}{|cc;cc|}
a_{11} & a_{12} & a_{13} & a_{14} & a_{15} \\
a_{21} & a_{22} & a_{23} & a_{24} & a_{25} \\
a_{31} & a_{32} & a_{33} & a_{34} & a_{35} \\
a_{41} & a_{42} & a_{43} & a_{44} & a_{45}
\end{BMAT}=\begin{BMAT}(b,20pt,15pt){|c;c;c|}{|c;c|}
A_{11} & A_{12} & A_{13} \\
A_{21} & A_{22} & A_{23}
\end{BMAT}
$$
\begin{opred}
Если при некотором делении матрицы на клетки получилось так, что клетки, стоящие на главной диагонали, являются квадратными матрицами, а клетки, стоящие под главной диагональю - нулевые, то говорят, что эта матрица является \mybold{(верхней) квазитреугольной}.
\newlineАналогично определяются \mybold{нижняя квазитреугольная} и \mybold{квазидиагональная} матрицы.
\end{opred}
\chapter{Операции над матрицами}
\section{Равенство матриц}
$A=\left( a_{ij} \right) \in \mathbb {R}^{m\times{}n}$
\newline$B=\left( b_{ij} \right) \in \mathbb {R}^{m\times{}n}$
\begin{opred}\label{eq_matrices}
$A$ и $B$ называюся \mybold{равными}, если $\forall{}i,j$ $a_{ij}=b_{ij}$.
\end{opred}
\section{Сложение матриц}
\begin{opred}
Матрица $C=\left( a_{ij}+b_{ij} \right) \in \mathbb {R}^{m\times{}n}$ называется \mybold{суммой} матриц $A$ и $B$.
\end{opred}
\begin{theor}
\mybold{Свойства операции сложения.}
\newlineДля любых мартиц одинакового размера выполнено:
\newline1) $A+B=B+A$ (\mybold{коммутативность сложения})
\newline2) $(A+B)+C=A+(B+C)$ (\mybold{ассоциативность сложения})
\newline3) $A+\Theta=A$ (\mybold{нулевая матрица является нейтральным элементом относительно сложения})
\newline4) $\forall{}A\ \exists{}(-A):\ A+(-A)=\Theta$ (\mybold{существование противоположного элемента})
\end{theor}
\begin{proof}
Справедливость всех этих свойств вытекает непосредственно из определения операции сложения и свойств рациональных чисел.
\end{proof}
\begin{opred}
Матрица $C=\left( a_{ij}-b_{ij} \right) \in \mathbb {R}^{m\times{}n}$ называется \mybold{разностью} матриц $A$ и $B$.
\end{opred}
\section{Умножение матрицы на число}
\begin{opred}
Матрица $(\alpha{}A)=\left( \alpha{}a_{ij} \right) \in \mathbb {R}^{m\times{}n}$ называется \mybold{произведением матрицы $A$ на число $\alpha$}. Операция умножения матрицы на число выполнима всегда.
\end{opred}
\begin{theor}
\mybold{Свойства операции умножения матрицы на число}
\newlineДля любых мартиц соответствующего размера выполнено:
\newline1) $1A=A$
\newline2) $\alpha(\beta{}A)=(\alpha{}\beta)A$ (\mybold{ассоциативность умножения на число})
\newline3) $(\alpha+\beta)A=\alpha{}A+\beta{}A$ (\mybold{дистрибутивность умножения относительно сложения})
\newline4) $\alpha(A+B)=\alpha{}A+\alpha{}B$ (\mybold{дистрибутивность сложения относительно умножения})
\end{theor}
\begin{proof}
Справедливость всех этих свойств вытекает непосредственно из определения операции сложения и свойств рациональных чисел.
\end{proof}
\mybold{Замечания}
\begin{itemize}
\item $-A=(-1)A$
\item Любая скалярная матрица
$
A=\begin{pmatrix}
\alpha & 0       & \ldots & 0      \\
0       & \alpha & \ldots & 0      \\
\vdots  & \vdots  & \ddots & \vdots \\
0       & 0       & \ldots & \alpha
\end{pmatrix}=\alpha{}I
$
\end{itemize}
\begin{opred}
\mybold{Линейной комбинацией матриц $A_1,\ldots,A_k$ с коэффициентами $\alpha_1,\ldots,\alpha_k$} называется матрица $B=\sum\limits_{i=1}^k\alpha_iA_i$
\end{opred}
\section{Умножение матриц}
\begin{opred}
\mybold{Произведением матриц $A\in{}\mathbb{R}^{m\times{}n}$ и $B\in{}\mathbb{R}^{k\times{}l}$} называется матрица $C\in{}\mathbb{R}^{m\times{}l}$ такая, что $c_{ij}=\sum\limits^k_{p=1}a_{ip}b_{pj}$
\end{opred}
\begin{remark}
Операция умножения определена только в том случае, когда $A$ и $B$ \mybold{согласованны}, т.е. $n=k$.
\end{remark}
Пусть $AB$ определена. Тогда:\begin{itemize}
\item $BA$ может быть неопределена ($m\neq{}l$)
\item Даже если $BA$ определена, то $AB$ и $BA$ могут быть разного размера.
\item Даже если $AB$ и $BA$ одинакового размера ($m=n=k=l$), результаты умножения могут быть разными. Пример:
$$
A=\begin{pmatrix}
1 & 1 \\
2 & 2
\end{pmatrix};\ B=\begin{pmatrix}
1 & 0 \\
0 & 2
\end{pmatrix}\ \Rightarrow\ AB=\begin{pmatrix}
1 & 2 \\
2 & 4
\end{pmatrix};\ BA=\begin{pmatrix}
1 & 1 \\
4 & 4
\end{pmatrix}
$$
\end{itemize}
\begin{opred}
\mybold{Перестановочными (коммутирующими)} называются матрицы $A$ и $B$ такие, что $AB=BA$
\end{opred}
Известно, что для $\forall\ i,j\in\mathbb{R}, ab=0 \Rightarrow \left[ \begin{aligned} 
a=0 \\
b=0
\end{aligned} \right.$. В случае с матрицами это неверно. Пример:
$
A=\begin{pmatrix}
1 & 0 \\
0 & 0
\end{pmatrix}$; $B=\begin{pmatrix}
0 & 0 \\
0 & 2
\end{pmatrix} \Rightarrow\ AB=\Theta$, но $A,B\neq\Theta$.
\begin{opred}
Матрицы $A$ и $B$ такие, что $AB=\Theta$, называются \mybold{делителями нуля}.
\end{opred}
\begin{theor}
\mybold{Свойства операции умножения}
\newlineДля любых матриц $A,B,C$ подходящего размера выполнено:
\newline1) $(AB)C=A(BC)$ (\mybold{ассоциативность умножения матриц})
\newline2) $\alpha(AB)=(\alpha{}A)B$
\newline3) $(A+B)C=AC+BC$ (\mybold{Дистрибутивность умножения относительно сложения})
\end{theor}
\begin{proof}
1) Если операция в левой части этого равенства определена, то определена операция в правой части равенства и результирующие матрицы правой и левой частей совпадают.
\newlineПусть \rmatrix{A}{m}{n}, \rmatrix{B}{n}{k} $\Rightarrow$ \rmatrix{AB}{m}{k}; \rmatrix{C}{k}{l}.
\newlineТогда: \rmatrix{(AB)C}{m}{l}, \rmatrix{BC}{n}{l}, \rmatrix{A(BC)}{m}{l}, следовательно, размеры правой и левой части совпадают.
\newlineДано: $\{(AB)C\}_{ij}=\sum\limits_{p=1}^k(AB)_{cp}C_{pj}=\sum\limits_{p=1}^k(\sum\limits_{q=1}^na_{iq}b_{qp})c_{pj}=\sum\limits_{p=1}^k\sum\limits_{q=1}^na_{iq}b_{qp}c_{pj}=\sum\limits_{q=1}^n\sum\limits_{p=1}^ka_{iq}b_{qp}c_{pj}=$\newline$=\sum\limits_{q=1}^na_{iq}(\sum\limits_{p=1}^kb_{qp}c_{pj})=\sum\limits_{q=1}^na_{iq}\{BC\}_{qj}=\{A(BC)\}_{ij}$, т.о., для $\forall\ i,j\ \{(AB)C\}_{ij}=\{A(BC)\}_{ij}$, т.е. \newline$(AB)C=A(BC)$ по определению (см. с. \pageref{eq_matrices}).
\newline\mybold{\slshape2 и 3 части - аналогично}
\end{proof}
\begin{remark}
Большинство формул сокращенного умножения для матриц имеют другой, более сложный, вид.
\newlineа) $(a-b)(a+b)=a^2-b^2$, но $(A-B)(A+B)=A^2-BA+AB-B^2$.
\newlineб) $(a\pm{}b)^2=a^2\pm{}2ab+b^2$, но $(A\pm{}B)^2=A^2\pm{}AB\pm{}BA+B^2$.
\newlineОднако, эти формулы верны и для матриц, если $A$ и $B$ --- перестановочные.
\end{remark}
\section{Транспонирование матриц}
\begin{opred}
Матрица $B$ называется транспонированием к матрице \rmatrix{A}{m}{n}, если \rmatrix{B}{n}{m} и\newlineдля $\forall\ i,j\ b_{ij}=a_{ji}$. Это обозначается $B=A^T$.
\end{opred}
\begin{remark}
Если $A^T=A$, то $A$ --- \mybold{симметрическая}.
\end{remark}
\begin{theor}
Для $\forall\ A,B$ подходящих размеров выполнены следующие свойства:
\newline1) $(A^T)^T=A$
\newline2) $(A+B)^T=A^T+B^T$
\newline3) $(\alpha{}A)^T=\alpha{}A^T$
\newline4) $(AB)^T=B^TA^T $
\end{theor}
\begin{proof}
1),2),3) - очевидно, доказательство по определению.
\newline4) Пусть \rmatrixof{A}{a_{ij}}{m}{n}, \rmatrixof{B}{b_{ij}}{n}{k}. Тогда \rmatrix{(AB)^T}{k}{m}, \rmatrix{B^T}{k}{n}, \rmatrix{A^T}{n}{m},\newline\rmatrix{B^TA^T}{k}{m}.
\newline$\{(AB)^T\}_{ij}=\{(AB)\}_{ji}=\sum\limits_{p=1}^nA_{jp}B_{pi}=\sum\limits_{p=1}^n\{A^T\}_{pj}\{B^T\}_{ip}=\sum\limits_{p=1}^n\{B^T\}_{ip}\{A^T\}_{pj}=\{B^TA^T\}_{ij}$
\newlineТ.о., для $\forall\ i,j\ \{(AB)^T\}_{ij}=\{B^TA^T)\}_{ij}$, т.е. $(AB)^T=B^TA^T$ по определению (см. с. \pageref{eq_matrices}).
\end{proof}
\section{Некоторые дополнительные особенности операции\newlineумножения матриц}
1) $Ae_i=a_i$, $e'_iA=a'_i$

2) Пусть $b=\begin{pmatrix}\beta_1 \\ \vdots \\ \beta_n\end{pmatrix}$, тогда $Ab=A(\beta_1e_1+\cdots+\beta_ne_n)=\beta_1Ae_1+\cdots+\beta_nAe_n=\beta_1a_1+\cdots+\beta_na_n$
\newline\mybold{Произведение матрицы на столбец является линейной комбинацией столбцов этой матрицы.}
\newlineАналогично, \mybold{произведение матрицы на строку является линейной комбинацией строк этой матрицы.}

3) $AB=A\begin{BMAT}(b,15pt,20pt){|c;c;c|}{|c|} b_1 & \cdots & b_k \end{BMAT} =\begin{BMAT}(b,15pt,20pt){|c;c;c|}{|c|} Ab_1 & \cdots & Ab_k \end{BMAT}$; 
$BA=\begin{BMAT}(b,15pt,20pt){|c|}{|c;c;c|} b'_1 \\ \vdots \\ b'_l \end{BMAT} A=\begin{BMAT}(b,15pt,20pt){|c|}{|c;c;c|} b'_1A \\ \vdots \\ b'_lA \end{BMAT}$
\newline\mybold{Столбцы произведения $AB$ являются линейной комбинацией столбцов $A$.}
\newline\mybold{строки произведения $AB$ являются линейной комбинацией строк $A$.}

$3^0$) $ AI=A\begin{BMAT}(b,15pt,20pt){|c;c;c|}{|c|} e_1 & \cdots & e_k \end{BMAT} = \begin{BMAT}(b,15pt,20pt){|c;c;c|}{|c|} Ae_1 & \cdots & Ae_k \end{BMAT} = \begin{BMAT}(b,15pt,20pt){|c;c;c|}{|c|} a_1 & \cdots & a_k \end{BMAT} = A$
\newline\mybold{$I$ является нейтральным элементом относительно операции умножения как справа, так и слева.}
\chapter{Элементарные преобразования матриц. Основной процесс}
Элементарные преобразования (ЭП) матриц - преобразование строк или столбцов матрицы.

ЭП строк бывают:

 I типа - перестановка 2-х строк матрицы местами

 II типа - умножение какой-либо строки матрицы на число, отличное от нуля

 III типа - прибавление к какой-либо строке матрицы другой строки этой матрицы, умноженной на любое число
Аналогично для столбцов.
\begin{opred}
\mybold{Основной процесс} --- приведение матрицы к верхнему ступенчатому виду, используя только ЭП строк I и III типа.
\end{opred}
\begin{theor}(Об основном процессе).

Любая ненулевая матрица путем элементарных преобразований только строк может быть приведена к верхнему ступенчатому виду.
\end{theor}
\begin{proof}1) Так как матрица ненулевая, то есть хотя бы один ненулевой столбец. Пусть $k_1$ - номер первого ненулевого столбца. Выполним, при необходимости, ЭП I рода так, чтобы элемент в позиции $(1,k)$ был ненулевым. Этот элемент --- \mybold{<<ведущий элемент первого шага>>}.

2) Для $\forall\ i>1$ из $i$-й строки вычтем первую,умноженную на $a_{ik_1}$ и разделенную на $a_{1k_1}$. Получим матрицу такого вида:
$$
\begin{BMAT}(b,15pt,20pt){|c;c;c|}{|c;c|}
\Theta & a_{1k_1} & a_{1k_1+1}\ \cdots\ a_{1n} \\
\Theta & \Theta   & A' \\
\end{BMAT}
$$
3) Перейдем к 1 пункту, но будем рассматривать не $A$, а $A'$. Так как размер рассматриваемой матрицы каждый раз уменьшается, то этот процесс конечен.
\end{proof}
\begin{theor}
Любое ЭП строк матрицы $A$ может быть описано как умножение её слева на специальным образом подобранную матрицу.
$$
A\epofs{}A'=SA
$$
Аналогично, для столбцов:
$$
A\epofc{}A'=AS
$$
\end{theor}
\begin{proof}
ЭП I типа: $A=\begin{bmatrix} a'_1 \\ a'_2 \\ a'_3 \\ \vdots \end{bmatrix} \epofs A'=\begin{bmatrix} a'_2 \\ a'_1 \\ a'_3 \\ \vdots \end{bmatrix}=\begin{bmatrix} s'_1 \\ s'_2 \\ s'_3 \\ \vdots \end{bmatrix} A=\begin{bmatrix} s'_1A \\ s'_2A \\ s'_3A \\ \vdots \end{bmatrix} \Rightarrow \begin{aligned}& s'_1=(0,1,0,\ldots) \equiv e'_2 \\ & s'_2=(1,0,0,\ldots) \equiv e'_1 \\ & s'_3=e'_3,\ \ldots \end{aligned}$
$$
S=\begin{bmatrix}
0      & 1      & 0      & \cdots & 0      \\
1      & 0      & 0      & \cdots & 0      \\
0      & 0      & 1      & \cdots & 0      \\
\vdots & \vdots & \vdots & \ddots & \vdots \\
0      & 0      & 0      & \cdots & 1
\end{bmatrix}
$$
ЭП II типа: $A=\begin{bmatrix} a'_1 \\ a'_2 \\ a'_3 \\ \vdots \end{bmatrix} \epofs A'=\begin{bmatrix} a'_1 \\ \alpha{} a'_2 \\ a'_3 \\ \vdots \end{bmatrix}=\begin{bmatrix} s'_1 \\ s'_2 \\ s'_3 \\ \vdots \end{bmatrix} A=\begin{bmatrix} s'_1A \\ s'_2A \\ s'_3A \\ \vdots \end{bmatrix} \Rightarrow \begin{aligned}& s'_1=(1,0,0,\ldots) \equiv e'_1 \\ & s'_2=(0,\alpha,0,\ldots) \equiv \alpha{}e'_2 \\ & s'_3=e'_3,\ \ldots \end{aligned}$
$$
S=\begin{bmatrix}
1      & 0      & 0      & \cdots & 0      \\
0      & \alpha & 0      & \cdots & 0      \\
0      & 0      & 1      & \cdots & 0      \\
\vdots & \vdots & \vdots & \ddots & \vdots \\
0      & 0      & 0      & \cdots & 1
\end{bmatrix}
$$
ЭП III типа: $A=\begin{bmatrix} a'_1 \\ a'_2 \\ a'_3 \\ \vdots \end{bmatrix} \epofs A'=\begin{bmatrix} a'_1 \\ \alpha{}a'_1+a'_2 \\ a'_3 \\ \vdots \end{bmatrix}=\begin{bmatrix} s'_1 \\ s'_2 \\ s'_3 \\ \vdots \end{bmatrix} A=\begin{bmatrix} s'_1A \\ s'_2A \\ s'_3A \\ \vdots \end{bmatrix} \Rightarrow \begin{aligned}& s'_1=(1,0,0,\ldots) \equiv e'_1 \\ & s'_2=(\alpha,1,0,\ldots) \equiv \alpha{}e'_1+e'_2 \\ & s'_3=e'_3,\ \ldots \end{aligned}$
$$
S=\begin{bmatrix}
1      & 0      & 0      & \cdots & 0      \\
\alpha & 1      & 0      & \cdots & 0      \\
0      & 0      & 1      & \cdots & 0      \\
\vdots & \vdots & \vdots & \ddots & \vdots \\
0      & 0      & 0      & \cdots & 1
\end{bmatrix}
$$
Т.о., для каждого типа преобразований мы сформировали специальную матрицу, умножение на которую эквивалентно применению этого ЭП; произведение таких матриц эквивалентно последовательному применению нескольких ЭП.

\mybold{Для столбцов - аналогично.}
\end{proof}
\begin{remark}
Применим теорему к единичной матрице $A=I:A\epofs{}A'=SA=SI=S$, $A\epofc{}A'=AS=IS=S$

Т.о., чтобы построить матрицу $S$, соответствующую ЭП, необходимо это ЭП применить к единичной матрице.

$S$ --- \mybold{матрицы перехода}.
\end{remark}
\chapter{Определители}
\section{Перестановки и их свойства}
\begin{opred}
\mybold{Перестановкой из множества $M$} называется упорядоченная совокупность чисел из $M$ $(\alpha_1,\alpha_2,\ldots,\alpha_n)$, в которой в разных позициях стоят разные числа $(\forall\ i\neq{}j\ \alpha_i\neq{}\alpha_j)$

$(1,2,\ldots,n)$ --- \mybold{натуральная перестановка}
\end{opred}
\begin{theor}
Всего существует $n!$ перестановок из $n$ первых чисел.
\end{theor}
\begin{proof}
На первое место можно поставить $n$ чисел, на второе --- $n-1$ остальных чисел, $\ldots$, на последнее место можно поставить одно оставшееся число. Итого вариантов $n\cdot(n-1)\cdot\ldots\cdot1=n!$
\end{proof}
\begin{opred}
Если $\alpha_i$ и $\alpha_j$ таковы, что $\alpha_i>\alpha_j$ при $i<j$, то говорят, что они образуют \mybold{инверсию}, иначе --- \mybold{порядок}.

Количество инверсий в перестановке обзоначается так: $\sigma(\alpha_1,\cdots,\alpha_n)$

$\sigma(1,2,\cdots,n)=0$

$\sigma(n,n-1,\cdots,1)=\frac{n(n-1)}2$
\end{opred}
\begin{opred}
Будем говорить, что перестановка \mybold{чётная}, если количество инверсий в ней чётное, и \newline\mybold{нечётная} --- в противном случае.
\end{opred}
\begin{opred}
Преобразование, при котором два элемента перестановки меняются местами, называется \mybold{транспозицией}.
\end{opred}
\begin{theor}
Любая транспозиция в перестановке меняет ее чётность.
\end{theor}
\begin{proof}
Рассмотрим два случая:

1) поменялись местами два соседних элемента - очевидно, что количество инверсий изменилось на 1 $\Rightarrow$ чётность изменилась.

2) поменялись местами два несоседних элемента. Тогда перегоним первый элемент ко второму, поменяем их местами и перегоним второй на место первого. Итого $2(j-i-1)+1$ транспозиций соседних элементов \then чётность менялась $2(j-i-1)+1$ раз - это нечётное число \then чётность изменилась.
\end{proof}
\begin{lemma}
Пусть $(\alpha_1,\ldots,\alpha_n)$ - перестановка с S инверсиями. Запишем числа этой перестановки в порядке возрастания. Тогда их индексы в исходной последовательность будут образовывать новую перестановку с тем же количеством $(S)$ инверсий.
\end{lemma}
$$
\begin{bmatrix}
\alpha_1 & \alpha_2 & \ldots & \alpha_n \\
1        & 2        & \ldots & n
\end{bmatrix} \xrightarrow[\mbox{столбцов}]{\mbox{перестановка}}
\begin{bmatrix}
1       & 2       & \ldots & n       \\
\beta_1 & \beta_2 & \ldots & \beta_n
\end{bmatrix}
$$
\begin{proof}
Рассмотрим два элемента: $\alpha_i$ и $\alpha_j$, $i<j$. Если в исходной перестановке они образовывали порядок, т. е. $\alpha_i<\alpha_j$, тогда их индексы в новой перестановке тоже будут образовывать порядок, т. к. их взаимное расположение не изменится.
Если в исходной перестановке они образовывали инверсию, т. е. $\alpha_i>\alpha_j$, тогда их индексы в новой перестановке тоже будут образовывать инверсию, т. к. их взаимное расположение изменится.
Итого для любых двух элементов отношение их индексов в новой перестановке такое же, как отношение самих элементов в исходной \then $\sigma(\alpha_1,\alpha_2,\ldots,\alpha_n)=\sigma(\beta_1,\beta_2,\ldots,\beta_n)$
\end{proof}
\section{Понятие определителя $n$--го порядка}
Пусть \rmatrixof{A}{a_{ij}}{n}{n}. Определителем $n$--го порядка называется число, сопоставленное этой матрице по определенному правилу.
Обозначение: $|A|,\ \det A$.
\begin{opred}
\mybold{Определителем} называется алгебраическая сумма произведений элементов матрицы $A$, взятых по одному из каждой строки и каждого столбца матрицы $A$; знак произведения определяется по следующему правилу: если сомножители в нем упорядочить по возрастанию номеров строк, то знак ставится в зависимости от четности перестановки, образованной номерами столбцов. Если она чётная, то знак плюс, иначе --- минус.
\end{opred}
$n=2: |A|=\begin{vmatrix} a_{11} & a_{12} \\ a_{21} & a_{22} \end{vmatrix}=\underset{(1,2)}{a_{11}a_{22}}-\underset{(2,1)}{a_{12}a_{21}}$

$n=3: |A|=\begin{vmatrix}
a_{11} & a_{12} & a_{13} \\ 
a_{21} & a_{22} & a_{23} \\
a_{31} & a_{32} & a_{33}
\end{vmatrix}=\underset{(1,2,3)}{a_{11}a_{22}a_{33}}-\underset{(2,1,3)}{a_{12}a_{21}a_{33}}-\underset{(1,3,2)}{a_{11}a_{23}a_{32}}+\underset{(2,3,1)}{a_{12}a_{23}a_{31}}+\underset{(3,1,2)}{a_{13}a_{21}a_{32}}-\underset{(3,2,1)}{a_{13}a_{22}a_{31}}$

$\det A=\sum\limits_{\alpha=(\alpha_1,\alpha_2,\ldots,\alpha_n)}(-1)^{\sigma(\alpha)}a_{1\alpha_1}a_{2\alpha_2}\cdots{}a_{n\alpha_n}$ (*) - $n!$ слагаемых
\section{Свойства определителя}
\begin{enumerate}
\item Определитель треугольной матрицы равен произведению элементов главной диагонали.
\begin{proof}
Все остальные слагаемые обязательно содержат нулевые элементы.
\end{proof}
\item Определитель квадратной матрицы не меняется при её транспонировании.
\begin{proof}
$\det A^T=\sum\limits_{\beta=(\beta_1,\beta_2,\ldots,\beta_n)}(-1)^{\sigma(\beta)}\{A^T\}_{1\beta_1}\{A^T\}_{2\beta_2}\cdots{}\{A^T\}_{n\beta_n}==$

$=\sum\limits_{\beta(\beta_1,\beta_2,\ldots,\beta_n)}
(-1)^{\sigma(\beta)}
\overbrace{a_{\beta_11}a_{\beta_22}\cdots{}a_{\beta_nn}}\limits^{\mbox{переупорядочим сомножители}}$
$$
\begin{pmatrix}
\beta_1 & \beta_2 & \ldots & \beta_n \\
1        & 2        & \ldots & n
\end{pmatrix} \xrightarrow[\mbox{столбцов}]{\mbox{перестановка}}
\begin{pmatrix}
1       & 2       & \ldots & n       \\
\alpha_1 & \alpha_2 & \ldots & \alpha_n
\end{pmatrix}
$$
По лемме, $\sigma(\beta)=\sigma(\alpha)$. Значит, у каждого слагаемого одинаковые знаки
\end{proof}
\begin{remark}
В силу того, что при транспонировании матрицы каждый столбец переводится в соответствующую строку, а строка --- в столбец, то во всех свойствах определителя строки и столбцы равноправны, т. е. каждое свойство о строках определителя имеет аналог для столбцов.
\end{remark}
\item Если в квадратной матрице есть нулевая строка, то ее определитель равен нулю.
\begin{proof}
В суммe (*) в каждое слагаемое обязательно входит элемент из той строки, которая нулевая \then все слагаемые равны нулю.
\end{proof}
\item Если все элементы какой-либо строки квадратной матрицы умножить на число $\alpha$, то определитель также умножится на $\alpha$.
\begin{proof}
Пусть на $\alpha$ умножится $i$--я строка. В сумме (*) в каждом слагаемом есть элемент из $i$--й строки \then вместо него стоит $\alpha{}a_{i\alpha_i}$. Из каждого слагаемого можно вынести $\alpha$ \then весь определитель умножится на $\alpha$.
\end{proof}
\begin{remark}
$|\alpha{}A|=\alpha^n|A|$, где $n$ --- размер матрицы.
\end{remark}
\item Если какая-либо строка квадратной матрицы является суммой двух строк, то определитель этой матрицы равен сумме двух определителей, в которых вместо рассматриваемой сроки стоит соответственно 1--я и 2--я слагаемая строка.
\begin{proof}
В сумме (*) вместо элемента, соответсвующего выбранной строке, каждый раз стоит сумма двух чисел, поэтому каждое слагаемое в сумме представимо в виде суммв двух произведений. С учетом определения определителя, каждая из двух сумм является определителем соответствующей матрицы.
\end{proof}
\item При перестановке двух строк квадратной матрицы её определитель меняет знак.
\begin{proof}
В произведениях четность перестановок номеров столбцов элементов новой матрицы относительно исходной матрицы изменится на противоположную \then знак каждого слагаемого в сумме изменится \then знак суммы изменится.
\end{proof}
\item Если в квадратной матрице есть две одинаковые строки, то ее определитель равен нулю.
\begin{proof}
Переставим эти две местами, при этом матрица не изменится. По свойству имеем:

$|A|=-|A|$ \then $|A|=0$.
\end{proof}
\item Если какая-либо строка квадратной матрицы является линейной комбинацией других строк этой матрицы, то определитель матрицы равен нулю.
\begin{proof}
Не ограничивая общности, будем считать, что таковой является первая строка.

$\begin{vmatrix} a'_1 \\ a'_2 \\ \vdots \\ a'_n \end{vmatrix}=
\begin{vmatrix} \alpha_2{}a'_2+\cdots+\alpha_n{}a'_n \\ a'_2 \\ \vdots \\ a'_n \end{vmatrix}=\begin{vmatrix} \alpha_2{}a'_2 \\ a'_2 \\ \vdots \\ a'_n \end{vmatrix}+\cdots+\begin{vmatrix} \alpha_n{}a'_n \\ a'_2 \\ \vdots \\ a'_n \end{vmatrix}=\alpha_2\begin{vmatrix} a'_2 \\ a'_2 \\ \vdots \\ a'_n \end{vmatrix}+\cdots+\alpha_n\begin{vmatrix} a'_n \\ a'_2 \\ \vdots \\ a'_n \end{vmatrix}$

В каждом определителе есть одинаковые строки, следовательно, $|A|=0$.
\end{proof}
\item Если к какой-либо строке квадратной матрицы прибавить линейную комбинацию других строк этой матрицы, то её определитель не изменится.
\begin{proof}
$\begin{vmatrix} a'_1+\alpha_2{}a'_2+\cdots+\alpha_n{}a'_n \\ a'_2 \\ \vdots \\ a'_n \end{vmatrix}=\begin{vmatrix} a'_1 \\ a'_2 \\ \vdots \\ a'_n \end{vmatrix}+\begin{vmatrix} \alpha_2{}a'_2+\cdots+\alpha_n{}a'_n \\ a'_2 \\ \vdots \\ a'_n \end{vmatrix}=\begin{vmatrix} a'_1 \\ a'_2 \\ \vdots \\ a'_n \end{vmatrix}$
\end{proof}
\end{enumerate}
\begin{remark}
Свойства 4 и 5 принято называть свойством линейности определителя относительно выбранной строки.

Свойства 4,6 и 9 показывают, как меняется определитель при элементарных преобразованиях матрицы: \begin{itemize}
\item I типа: определитель меняет знак.
\item II типа: определитель умножается на коэффициент.
\item III типа: определитель не меняется.
\end{itemize}
Так как коэффициент $\alpha$ в ЭП II типа отличен от нуля, то изменение определителя при ЭП контролируемо.
$|A|\rightarrow\left\{\right.
\end{remark}
\section{-Метод Гаусса в вычислении определителя-}
\section{-Миноры и алгебраические дополнения-}
\section{-Определители квазитреугольной матрицы-}
\section{-Определитель произведения матриц-}
\chapter{-Обратная матрица-}

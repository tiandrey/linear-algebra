\documentclass[draft]{report}
\usepackage[utf8]{inputenc}
\usepackage[russian]{babel}
\usepackage{amsmath}
\usepackage{amsthm}
\usepackage{amsfonts}
\usepackage{amssymb}
\textwidth=195mm % 210
\oddsidemargin=-17.9mm
\textheight=272mm % 297
\topmargin=-17.9mm
\headheight=10mm
\hoffset=0mm
\voffset=-20mm
\parindent=0in

\newcommand{\res}{\mathop{\mathrm{res}}\nolimits}
\renewcommand{\bf}{\bfseries}
\newcommand{\rra}{\rightrightarrows}
\newcommand{\rrae}[1]{\underset{#1}{\rightrightarrows}}
\newcommand{\forcenewline}{$\phantom{\mbox{newline}}$\newline}
\newcommand{\dd}{\partial}
\newcommand{\then}{\ \Rightarrow\ }
\newcommand{\R}{\mathbb{R}}
\newcommand{\Z}{\mathbb{Z}}
\renewcommand{\C}{\mathbb{C}}
\newcommand{\N}{\mathbb{N}}
\newcommand{\ind}[3]{\underset{#1}{\overset{#2}{#3}}}
\newcommand{\mint}[2]{\underset{#1}{\overset{#2}{\int}}}
\newcommand{\moint}[1]{\underset{#1}{\oint}}
\newcommand{\msum}[2]{\underset{#1}{\overset{#2}{\sum}}}
\newcommand{\ksum}{\msum{k=1}{n}}
\newcommand{\rsum}{\msum{n=1}{\infty}}
\newcommand{\ssum}{\msum{n=0}{\infty}}
\newcommand{\lsum}{\msum{n=-\infty}{+\infty}}
\newcommand{\mlim}[1]{\underset{#1}{\lim}}
\newcommand{\mres}[1]{\underset{#1}{\res}}
\newcommand{\mmax}[1]{\underset{#1}{\max}}
\newcommand{\mmin}[1]{\underset{#1}{\min}}
\newcommand{\LRA}{\Leftrightarrow}
\newcommand{\epsdelta}{\forall \e>0\ \exists \delta(\e)>0\colon}
\renewcommand{\bar}{\overline}
\renewcommand{\Im}{\mathop{\mathrm{Im}}\nolimits}
\renewcommand{\Re}{\mathop{\mathrm{Re}}\nolimits}
\newcommand{\Ln}{\mathop{\mathrm{Ln}}\nolimits}
\newcommand{\Arg}{\mathop{\mathrm{Arg}}\nolimits}
\newcommand{\diam}{\mathop{\mathrm{diam}}\nolimits}
\newcommand{\Int}{\mathop{\mathrm{int}}\nolimits}
\newcommand{\Ext}{\mathop{\mathrm{ext}}\nolimits}

\newcommand{\ninf}[1]{\underset{n\to\infty}{#1}}
\renewcommand{\a}{\alpha}
\renewcommand{\b}{\beta}
\newcommand{\g}{\gamma}
\newcommand{\G}{\Gamma}
\renewcommand{\f}{\phi}
\renewcommand{\d}{\delta}
\renewcommand{\l}{\lambda}
\renewcommand{\L}{\Lambda}
\renewcommand{\r}{\rho}
\newcommand{\D}{\Delta}
\newcommand{\e}{\varepsilon}
\newcommand{\E}{\ \exists}
\newcommand{\F}{\ \forall}
\newcommand{\z}{\bar{z}}
\renewcommand{\o}{\bar{o}}
\newcommand{\rd}{\underset{n=1}{\overset{\infty}{\sum}}}
\newcommand{\CC}{\bar{\C}}
\newcommand{\sys}[1]{\left\{\begin{matrix}#1\end{matrix}\right.}
\newcommand{\bsys}[1]{\left.\begin{matrix}#1\end{matrix}\right\}}
\newcommand{\mat}[1]{\begin{matrix}#1\end{matrix}}
\newcommand{\opr}[1]{\begin{opred}#1\end{opred}}
\newcommand{\lra}[1]{\underset{#1}{\longrightarrow}}

\newtheorem*{lemma}{Лемма}
\newtheorem*{theor}{Теорема}
\newtheorem*{theor*}{Теорема}
\newtheorem*{opred}{Определение}
\theoremstyle{remark}
\newtheorem*{remark}{\mybold{Замечание}}

\begin{document}
\tableofcontents
\chapter{Комплексные числа и функции комплексной переменной}
\section{Комплексные числа и их свойства}
\opr{Комплексное число $z=(x;y), x,y\in\R$

$x=\Re z$ -- действительная часть $z$

$y=\Im z$ -- мнимая часть $z$}

$z_1=z_2 \LRA x_1=x_2, y_1=y_2$
\subsection{Арифметические операции}
$z_1+z_2=(x_1+x_2,y_1+y_2)$

$z_1z_2=(x_1x_2-y_1y_2,x_1y_2+x_2y_1)$

Вычитание -- обратное к сложению

Деление -- обратное к умножению

$z=\cfrac{z_1}{z_2} \LRA z_1=z*z_2$

$0=(0;0)$

$1=(1;0)$

$a\in\R\LRA a=(a;0)$

$\{z\}=\C$ -- поле комплексных чисел

$\R$ -- поле действ. чисел -- подполе $\C$
\subsection{Комплексное сопряжение}

$\bar{z}=(x;-y)$ -- комплексное сопряжённое к $z$

{\bfseries Свойства:}
\begin{enumerate}
\item$\bar{\left(\bar{z}\right)}=z$
\item$z\bar{z}=(x^2+y^2;0)$
\item$z=\bar{z}\LRA z\in\R$
\item$\bar{z_1\pm z_2}=\bar{z_1}\pm\bar{z_2}$
\item$\bar{z1*z2}=\bar{z1}*\bar{z2}$
\item$\bar{\left(\cfrac{z1}{z2}\right)}=\cfrac{\bar{z1}}{\bar{z2}}$
\end{enumerate}
\subsection{Геометрическая интерпретация}

$z=(x,y)=(r,\phi)$

$r=\sqrt{x^2+y^2}$

$\left\{\begin{matrix}
\cos\phi=\cfrac{x}{\sqrt{x^2+y^2}} \\
\sin\phi=\cfrac{y}{\sqrt{x^2+y^2}}
\end{matrix}\right.\,,\ z\neq0$

$z=0\LRA r=0$

$\left\{\begin{matrix}
x=r\cos\phi\\
y=r\cos\phi
\end{matrix}\right.$

$\phi\in\Phi=\Arg z=\{\phi_0+2\pi k,\ k\in\Z\}$

$\phi_0=\arg z,\ \phi_0\in[0;2\pi]$

Множество точек (векторов) с арифметическими операциями: сложение и умножение на действительное число $\lambda z=(\lambda x, \lambda y)$ -- линейное пространство.

Если $||z_1-z_2||=\sqrt{(x_1-x_2)^2+(y_1-y_2)^2}$, то это евклидово пространство.

Sample: доказать, что $|z_1+z_2|^2+|z_1-z_2|^2=2(|z_1|^2+|z_2|^2)$ (используя правило параллелограмма и тот факт, что сумма квадратов диагоналей параллелограмма равна двойной сумме квадратов сторон)

\subsection{Алгебраическая форма комплексных чисел}

В линейном пространстве $\forall z$

$z=x\bar{e}_1+y\bar{e}_2$

$\bar{e}_1,\bar{e}_2$ -- векторы системы координат

$\bar{e}_1=(1,0)=1$

$\bar{e}_2=(0,1)=i$

$i^2=(-1,0)=-1$

\subsection{Тригонометрическая форма}

$r=\sqrt{x^2+y^2}/=|z|/$

$\phi\in \Arg z$

$x=r\cos\phi, y=r\sin\phi$

$z=r(\cos\phi+i\sin\phi)$

{\bfseries Утверждение (свойства модуля):}
\begin{enumerate}
\item$|z|\geq0; |z|=0\LRA z=0$
\item$|\bar{z}|=|z|$
\item$|-z|=|z|$
\item$z\bar{z}=|z|^2$
\item$|z_1z_2|=|z_1||z_2|$
\item$\left|\cfrac{z_1}{z_2}\right|=\cfrac{|z_1|}{|z_2|}$
\item$|z_1+z_2|\leq|z_1|+|z_2|;\ |z_1-z_2|\leq|z_1|+|z_2|$
\item$|Re\ z|\leq|z|;\ |Im\ z|\leq|z|$
\item$|z_1-z_2|\geq\bigl||z_1|-|z_2|\bigr|$
\end{enumerate}
\begin{proof}
$|z_1|=|z_1-z_2+z_2|\leq|z_1-z_2|+|z_2|$
\end{proof}

\subsection{Показательная форма}

$e^{i\phi}=\cos\phi+i\sin\phi$ (Формула Эйлера)

$e^{-i\phi}=\cos\phi-i\sin\phi$

$z=re^{i\phi}; r=|z|, \phi\in\Arg z (z\neq0);\ \bar{z}=re^{-i\phi}$

$z=0\LRA r=0$

{\bfseries Утверждение:}
\begin{enumerate}
\item$e^{i\phi_1}e^{i\phi_2}=e^{i(\phi_1+\phi_2)}$
\item$\cfrac{e^{i\phi_1}}{e^{i\phi_2}}=e^{i(\phi_1-\phi_2)}$
\item$\left(e^{i\phi}\right)^n=e^{in\phi},\ n\in\Z$
\item$\left|e^{i\phi}\right|=1$
\item$e^{i(\phi+2\pi k)}=e^{i\phi},\ k\in\Z$
\end{enumerate}
\forcenewline
\begin{proof}
\forcenewline
\begin{enumerate}
\item[1.] $e^{i\phi_1}e^{i\phi_2}=(\cos\phi_1+i\sin\phi_1)(\cos\phi_2+i\sin\phi_2)=\cos\phi_1\cos\phi_2-\sin\phi_1\sin\phi_2+i(\sin\phi_1\cos\phi_2+\sin\phi_2\cos\phi_1)=\cos(\phi_1+\phi_2)+i\sin(\phi_1+\phi_2)=e^{i(\phi_1+\phi_2)}$
\item[4.] $|e^{i\phi}|=|\cos\phi+i\sin\phi|=\sqrt{\cos^2\phi+\sin^2\phi}=1$
\end{enumerate}
\end{proof}

\subsection{Извлечение корня $n$-й степени}

\opr{$z=\sqrt[n]{c}\ (n\in\N,\ n>1)\LRA z^n=c$}
\begin{enumerate}
\item$\sqrt[n]{0}=0$
\item$c\neq0$

$c=\rho e^{i\alpha}$

$z=re^{i\phi}$

$z^n=c\LRA r^ne^{in\phi}=\rho e^{i\alpha}\then\left\{\begin{matrix}
r^n=\rho \\
n\phi=\alpha+2\pi k
\end{matrix}\right.$

$\sqrt[n]{c}=\sqrt[n]{\rho}e^{i\frac{\alpha+2\pi k}{n}},\ k=0,1,\ldots,n-1$
\end{enumerate}

\section{Расширенная комплексная плоскость. Кривые. Множества на комплексной плоскости.}

\subsection{Расширенная комплексная плоскость}

\begin{opred}
Расширенная комплексная плоскость $\bar{\C}=\C\cup\{\infty\}$
\end{opred}
Модель расширенной комплексной плоскости -- сфера Римана $S$.

{\bfseries Утверждение.}

Пусть $N\leftrightarrow M$. Тогда

$\xi=\cfrac{x}{1+|z|^2}, \eta=\cfrac{y}{1+|z|^2}, \zeta=\cfrac{|z|^2}{1+|z|^2}$ (1)

$x=\cfrac{\xi}{1-\zeta}, y=\cfrac{\eta}{1-\zeta}$ (2)
\begin{proof}
$N(x,y,0)$

$\vec{PM}=(\xi,\eta,\zeta-1)$

$\vec{PN}=(x,y,-1)$

$\vec{PM}||\vec{PN}\then\exists k\neq0$, что $\cfrac\xi x=\cfrac\eta y=\cfrac{\zeta-1}{-1}=k$ (3) $\then\zeta=1-k,\xi=kx, \eta=ky$

$S\colon\xi^2+\eta^2+\left(\zeta-\frac12\right)^2=\left(\frac12\right)^2$

$\xi^2+\eta^2+\zeta^2=\zeta$

$k^2(x^2+y^2)+(1-k)^2=1-k$

$k=\cfrac1{1+x^2+y^2}=\cfrac1{1+|z|^2}$

$\xi=\cfrac{x}{1+|z|^2}, \eta=\cfrac{y}{1+|z|^2}, \zeta=\cfrac{|z|^2}{1+|z|^2}$
\end{proof}
{\bfseries Замечания:}
\begin{enumerate}
\item $\infty$ -- одна на $\bar{\C}$
\item Множество окружностей на $S\ \leftrightarrow$ множество окружностей и прямых $\bar{\C}$
\item Любая окружность на $S$, не проходящая через точку $P$, соответствует окружности $\bar{\C}$; любая окружность на $S$, проходящая через точку $P$, соответствует прямой $\bar{\C}$
\end{enumerate}

\subsection{Кривые на плоскости}

Пусть $x(t),y(t),t\in[\alpha,\beta]$ -- вещественные непрерывные функции.

$\gamma\colon\left\{\begin{matrix}z(t)=x(t)+iy(t) \\ t\in[\alpha,\beta]\end{matrix}\right.$ (4)

Уравнение (4) задает непрерывную кривую на плоскости.

На $\gamma$ в силу (4) индуцировано направление, соответствующее изменению параметра $t$.

\opr{$z(\alpha)$ -- начальная точка $\gamma$, $z(\beta)$ -- конечная точка $\gamma$.}

Кривая, которая отличается от $\gamma$ только направлением, обозначается $\gamma^{-1}$.

\opr{Если $\exists\ t_1\neq t_2$, что $z(t_1)=z(t_2)$, то это точка самопересечения кривой $\gamma$ (кроме $t_1=\alpha,t_2=\beta$ и наоборот).}
\opr{Простая (жорданова) кривая -- непрерывная кривая без точек самопересечения.}
\opr{Если $z(\alpha)=z(\beta)$, то $\gamma$ -- замкнутая кривая.}
\opr{Кривая $\g$ -- гладкая, если $x,y\in\C^1[\a,\b]$, причём $(x'(t))^2+(y'(t))^2\neq0$. Иначе говоря, $\exists z'(t), t\in[\a,\b]$, причём $z'(t)\neq0$.}
{\bfseries Замечание:}

Если кривая замкнута, то для гладкости необходимо дополнительное равенство $z'(\a+0)=z'(\b-0)$.

\opr{Непрерывная кривая называется кусочно-гладкой, если она состоит из конечного числа гладких кусков.}
\opr{Замкнутая простая кусочно-гладкая кривая называется замкнутым контуром.}

\subsection{Множества на комплексной плоскости}

$E\neq\varnothing$

$d=\diam E=\underset{z_1,z_2\in E}{sup}|z_1-z_2|$

Круговые окрестности:

$z_0\neq\infty$

$K_\varepsilon(z_0)=\left\{z\colon|z-z_0|<\varepsilon\right\}$

Проколотая окрестность: $\dot{K}_\varepsilon(z_0)=\bigl\{z\colon0<|z-z_0|<\varepsilon\bigr\}$

$z_0=\infty$

$K_B(\infty)=\bigl\{z\colon|z|>B\bigr\}$

$\dot{K}_B(\infty)=\bigl\{z\colon B<|z|<\infty\bigr\}$

\opr{Внутренняя точка -- точка, у которой есть окрестность, содержащая только точки, принадлежащие множеству.}
\opr{Внешняя точка -- точка, у которой есть окрестность, содержащая только точки, не принадлежащие множеству.}
\opr{Граничная точка -- точка, в любой ненулевой окрестности которой есть как внешние, так и внутренние точки.}
\opr{Предельная точка -- точка, в любой ненулевой окрестности которой бесконечно много внутренних точек.}
\opr{Открытое множество -- множество, все точки которого внутренние.}
\opr{Замкнутое множество -- множество, содержащее все свои предельные точки.}
\opr{Связное множество -- множество, для любых двух точек которого существует непрерывная кривая, принадлежащая данному множеству, соединяющая эти точки.}
\opr{Область -- непустое открытое связное множество.}
\begin{theor}[Жордан]
Всякая непрерывная простая замкнутая кривая $\g$ разбивает $\bar{\C}$ на две области: внутреннюю ($\Int\g$) с границей $\g$ и внешнюю ($\Ext\g$) с границей $\g$, содержащую точку $\infty.$
\end{theor}
\opr{Односвязное множество -- такое множество, с которым любой замкнутый контур непрерывной деформацией может быть стянут к точке, принадлежащей множеству.

Alternative one: односвязное множество -- множество, в котором для любого замкнутого контура, принадлежащего множеству, его внутренняя область тоже принадлежит множеству.}
\opr{Область $n$-связна, если у неё $n$ границ.}

\subsection{Задание множеств на комплексной плоскости}

Уравнение прямой:

$bx+cy+d=0,\ b^2+c^2>0$

$\sys{z=x+iy \\ \z=x-iy}\LRA\sys{x=\frac{z+\bar{z}}{2}\\y=\frac{z-\bar{z}}{2}}$

$b(z+\bar{z})-i(z-\bar{z})c+2d=0$

$\bar{B}z+B\bar{z}+D=0$ (5)

$B=b+ic, D=2d, B\bar{B}\neq0\ /B\neq0/$

Уравнение окружности:

$a(x^2+y^2)+2bx+2cy+d=0\ (a\neq0)$

$\left(x+\frac ba\right)^2+\left(y+\frac ca\right)^2=\frac{b^2+c^2-ad}{a^2}\ /b^2+c^2-ad>d/$

$Az\z+\bar{B}z+B\z+D=0$ (6)

$A=a,\ D=d,\ B=b+ic$

\section{Числовые последовательности и ряды}

\subsection{Сходимость}

$\{z_n\}\in\C,\ z_n=x_n+iy_n$

\opr{$\ninf{\{z_n\}\to z_0}\ (z_0\neq\infty)\ /\ninf{\lim}z_n=z_0/\ \LRA\ \forall \e>0\ \exists N\ \forall n>N\ |z_n-z_0|<\e$
}

\opr{$\{z_n\}$ -- бесконечно большая последовательность $\LRA\ \forall B>0\ \exists N\colon \forall n>N\ |z_n|>B$}

\begin{theor}
$\ninf{z_n\to z_0}\LRA\sys{x_n\to x_0 \\ y_n\to y_0},n\to\infty$
\end{theor}

\begin{proof}
\forcenewline
\begin{enumerate}
\item $|x|\leq|z|$
$|y|\leq|z|$
\item $\forall\e\ \exists N,\,\forall n>N\colon
\sys{|x_n-x_0|<\cfrac\e{\sqrt{2}} \\ |y_n-y_0|<\cfrac\e{\sqrt{2}}}$

$|z_n-z_0|=\sqrt{(x_n-x_0)^2+(y_n-y_0)^2}<\e$
\end{enumerate}
\end{proof}

{\bfseries Замечание.}

Для $z_0=\infty$ теорема не выполняется.

{\bfseries Утверждение 1.}

$z_n=r_ne^{i\f_n}$

Если $\sys{r_n\to r_0 \\ \f_n \to \f_0},\ n\to\infty\ \then\ z_n\to z_0=r_0e^{i\f_0}$

\begin{proof}
$z_n=r_ne^{i\f_n}=r_n(\cos\f_n+i\sin\f_n)$

$\sys{x_n=r_n\cos\f_n\to x_0 \\ y_n=r_n\sin\f_n\to y_0}\then z_n\to z_0$ по теореме.
\end{proof}

{\bfseries Замечание.}

Обратное утверждение неверно.

{\bfseries Утверждение 2.}

$z_n\to z_0\then r_n\to r_0$ (доказательство следует из $\bigl||z_n|-|z_0|\bigr|\leq|z_n-z_0|$)

{\bfseries Замечание.}

Из сходимости $z_n\to z_0$ не обязательно следует $\f_n\to\f_0$

{\bfseries Примеры:}
\begin{enumerate}
\item $z_n=\frac1n$ $\forall n\ \arg z_n=0$; $\arg 0$ не определён.
\item $\arg z\in[0,2\pi)$

$z_n=1+i\cfrac{(-1)^n}{n}\to z_0=1$

$n=2k\colon z_{2k}=1+\cfrac{i}{2k},\ \arg z_{2k}\to0+$

$n=2k-1\colon z_{2k-1}=1-\cfrac{i}{2k-1},\ \arg z_{2k-1}\to2\pi-$
\end{enumerate}

\subsection{Ряды}

$\ind{n=1}{\infty}{\sum}z_n$ (1) -- числовой ряд

\opr{Числовой ряд (1) сходится, если сходится к конечному пределу последовательность $\{\rho_n\}$}

\begin{theor}
Ряд (1) сходится $\LRA\ \sys{\mbox{сходится}\ \ind{n=1}{\infty}{\sum}x_n \\ \mbox{сходится}\ \ind{n=1}{\infty}{\sum}y_n }$
\end{theor}

\opr{Ряд (1) сходится абсолютно, если сходится $\rd |z_n|$ (2)}

{\bfseries Утверждение 3.}

Ряд (1) сходится $\LRA\ \forall\e>0\ \exists N\colon\forall n>N,\forall p\in\N\ \left|\ind{k=n+1}{n+p}{\sum}z_k\right|<\e$

{\bfseries Утверждение 4.}

Сходимость ряда (1) означает $z_n\to 0$

{\bfseries Утверждение 5.}

Из сходимости ряда (2) следует сходимость ряда (1).

{\bfseries Утверждение 6.}

Сходимость ряда (2) можно исследовать, применяя любой из признаков сходимости вещественных рядов с неотрицательными членами.

\section{Функции комплексной переменной. Предельное значение и неприменимость. Основные элементарные функции.}

\subsection{Основные понятия}

\opr{Пусть $E,F\in\CC$, $E,F$ -- непустые.

$f\colon E\to F$ $(w=f(z))$, если $\forall z\in E\ \exists w\in F\colon w=f(z)$}

\opr{ФКП $w=f(z)$ однолистна на $G\subset E$, если $\forall z_1,z_2\in G,z_1\neq z_2\ w_1\neq w_2$}

{\bfseries Замечание.}

Далее рассматриваются однозначные ФКП, если специально не оговорено противное.

Задание однозначной функции $w=f(z)$ на E означает задание пары вещественных функций двух действительных переменных: $\sys{u=u(x,y)\\v=v(x,y)}\ /w=u+iv,\ z=x+iy/,\ u,v,x,y\in\R$

\subsection{Предельное значение функции}

$w=f(z),\ z_0$ -- предельная точка множества $E$, $z_0\neq\infty$.

\opr{$l$ -- предельное значение $f(z)$ в точке $z_0$ /$\mlim{z\to z_0}f(z)=l$/ $\LRA\ \forall\e>0\ \exists\delta(\e)>0\colon$

$\forall z\in E\colon 0<|z-z_0|<\delta \then |f(z)-l|<\e$}

\begin{theor}
Пусть $l=a+ib\ (l\neq\infty)$.

$\mlim{z\to z_0}f(z)=l\ \LRA\ \sys{u(x,y)\to a \\ v(x,y)\to b}$ при $(x,y)\to(x_0,y_0)$
\end{theor}

\begin{proof}
$f=u_iv$

$\left.\begin{matrix}|u-a|\\|v-b|\end{matrix}\right]\leq\sqrt{(u-a)^2+(v-b)^2}=|f(z)-l|\leq|u-a|+|v-b|$
\end{proof}

{\bfseries Следствие:}

Из теоремы вытекают свойства пределов ФКП -- те же, что и в вещественном случае.

\begin{theor}[арифметические действия с пределами]
Пусть $\exists \mlim{z\to z_0}f(z)=A, \mlim{z\to z_0}g(z)=B \then \ldots$
\end{theor}

\subsection{Непрерывность}

\opr{$f(z)$ $\C$-непрерывна в т. $z_0$ ($z_0$ -- предельная точка $E$) $\LRA\ \exists \mlim{z\to z_0}f(z)=f(z_0)$}

В частности, если $z_0\neq\infty,\ f(z_0)\neq\infty$, то $\epsdelta\ \forall z\in E,\ |z-z_0|<\delta\ \then\ |f(z)-f(z_0)|<\e$

\opr{$f(z)$ непрерывна на $E$, если $f(z)$ непрерывна $\forall z\in E$}

\begin{theor}
$f(z)\ \C$-непрерывна в т. $z_0$ $\LRA$ $\sys{u(x,y)\\v(x,y)}$ -- $\R^2$-непрерывна в $(x_0,y_0)$ (следует из т. 4.1))
\end{theor}

\begin{theor}
Если $f,g\ \C$-непрерывны в т. $z_0$, то $f\pm g, f*g, \cfrac fg (g(z_0))$ $\C$-непрерывны в $z_0$
\end{theor}

\begin{theor}
Если $\zeta=f(z)$ непрерывна в $z_0\in E$, $w=g(\zeta)$ непрерывна в т. $\zeta_0=f(z_0)$, то $F(z)=g(f(z))$ непрерывна в т. $z_0$
\end{theor}

\opr{f(z) равномерно непрерывна на Е $\LRA\ \epsdelta \forall z_1,z_2\in E\ |z_1-z_2|<\delta\ \then\ |f(z_1)-f(z_2)|<\e$}

\begin{theor}[Кантор]
Пусть $E$ -- ограниченное замкнутое множество. Если $f$ непрерывна на $E\ \then\ f$ равномерно непрерывна на $E$.
\end{theor}

\subsection{Основные элементарные функции}

\begin{itemize}
\item[а)] $w=z^n,\ n>1,\ n\in\N$

Определена в $\C$, доопределим в $\CC\colon w(\infty)=\infty$

Неоднолистна ($n$-листна) в $\C$.

Непрерывна в $\CC$ (см. т. 4.4)
\item[б)] $w=\sqrt[n]{z}$

Определена в $\CC$

$n$-значная

Главная ветвь ($k=0$):

$w=\sqrt[n]{|z|}e^{\frac{i\arg z}{n}}$

$w=\arg z$

Определена при $z\neq0,z\neq\infty$

Непрерывна в $\C\smallsetminus\R^+$
\item[в)] $w=e^z$

$e^z=e^x(\cos y+i\sin y)$

$\nexists\mlim{z\to\infty}e^z$

Периодическая с основным периодом $T_0=2\pi i\ \then$ бесконечнолистна в $\C$

Непрерывна в $\C$
\item[г)] $w=\Ln z$ ($\LRA e^w=z$)

Бесконечнозначная.

$\Ln z=\ln|z|+i\Arg z=\ln|z|+i(\arg z+2\pi k),\ k\in\Z\ \forall z\in\C\smallsetminus\{0\}$

Главная ветвь: $\ln z=\ln|z|+i\arg z$

Непрерывна в $\C\smallsetminus\R^+$
\item[д)] Тригонометрические функции:

$\cos z=\frac12(e^{iz}+e^{-iz})$

$\sin z=\frac1{2i}(e^{iz}-e^{-iz})$

$\tg z=\cfrac{\sin z}{\cos z}$

$\ctg z=\cfrac{\cos z}{\sin z}$

Непрерывны на множестве определения.

Имеют место все формулы элементарной тригонометрии.
\item[е)] Обратные тригонометрические функции
\item[ж)] Гиперболические функции:

$\ch z=\frac12(e^z+e^{-z})$

$\sh z=\frac12(e^z-e^{-z})$

$\th z,\ \cth z$
\end{itemize}

\section{Дифференцируемость ФКП}

\subsection{Определение и свойства дифференцируемых функций}

$z_0,z\in E,\ z-z_0=\Delta z$

$f(z)$ -- однозначная

$\Delta f(z_0)=f(z)-f(z_0)=f(z_0+\Delta z)-f(z_0)$

\opr{Производная $f(z)$ в т. $z_0$ (по множеству $E$ в т. $z_0$) $f'(z_0)=\mlim{\mat{z\to z_0 \\ z\in E}}\cfrac{f(z)-f(z_0)}{z-z_0}=\mlim{\D z\to 0}\cfrac{\D f(z_0)}{\D z}$ (предел конечный)}

\opr{$f(z)$ дифференируема в т. $z_0$ $\LRA$ $\D f(z_0)=K\D z+\o(|\D z|)$, где $K\in\C$ и не зависит от $\D z$ (но может зависеть от $z_0$)}

$|\D z|=\sqrt{\D x^2+\D y^2}=\rho$, т.е. $\overline{o}(|\D z|)=\overline{o}(\rho)$

{\bfseries Утверждение 1.}

$f(z)$ дифференцируема в $z_0$ $\LRA$ $\exists f'(z_0)$, причём $K=f'(z_0)$

\opr{$f(z)$ дифференцируема на $E\LRA\ \exists f'(z)\ \forall z\in E$ }

{\bfseries Утверждение 2.}
\begin{itemize}
\item[а)] $f(z)=C\then f'(z)=0$
\item[б)] $(cf(z))'=cf'(z)$
\item[в)] $(f\pm g)'=f'\pm g'$
\item[г)] $(fg)'=f'g+fg'$
\item[д)] $\left(\cfrac fg\right)'=\cfrac{f'(z)g(z)-f(z)g'(z)}{g^2(z)}$
\end{itemize} при условии, что существуют производные в правых частях.

{\bfseries Утверждение 3.}

$\zeta=f(z)$ -- дифференцируема в т. $z_0\in Z$

$w=g(\zeta)$ определена на $G\supset f(E)$

$g(\zeta)$ дифференцируема в $\zeta_0=f(z_0)\then F(z)=g(f(z))$ дифференцируема в т. $z_0$, причём $F'(z_0)=\bigl.g'(\zeta_0)f'(z_0)\bigr|_{\zeta_0=f(z_0)}$

{\bfseries Утверждение 4.}

Пусть $w=f(z)$ -- взаимно-однозначное отображение $E$ на $F$, причём обратная функция $z=\phi(w)$ непрерывна. Тогда, если $\exists f'(z_0)\neq0$, то $\exists \phi'(w_0)=\cfrac{1}{f'(z_0)}$

\begin{proof}
1\forcenewline

$\cfrac{1}{\frac{w-w_0}{z-z_0}}=\cfrac{z-z_0}{w-w_0}=\cfrac{\f(w)-\f(w_0)}{w-w_0}$

$\mlim{\D w\to 0}\cfrac{z-z_0}{w-w_0}=\mlim{\D w\to0}\cfrac{\f(w)-\f(w_0)}{w-w_0}=\f'(w_0)=\mlim{\D w\to0}\cfrac{1}{\frac{w-w_0}{z-z_0}}=\mlim{\D z\to0}\cfrac{1}{\frac{w-w_0}{z-z_0}}=\cfrac{1}{\mlim{z\to z_0}\frac{\D w}{\D z}}=\cfrac{1}{f'(z_0)}$
\end{proof}

\subsection{Необходимое и достаточное условие дифференцируемости ФКП в точке}

\begin{theor}[5.1]
\forcenewline

$f(z)=u+iv$ дифференцируема ($\C$-дифференцируема) в т. $z_0=x_0+iy_0$ тогда и только тогда, когда \begin{enumerate}
\item $u(x,y),v(x,y)$ -- $\R^2$-дифференцируемы в $(x_0,y_0)$
\item $\sys{\cfrac{\dd u}{\dd x}=\cfrac{\dd v}{\dd y} \\ \cfrac{\dd u}{\dd y}=-\cfrac{\dd v}{\dd x}}$ в $(x_0,y_0)$ -- условия Коши-Римана.
\end{enumerate}
\end{theor}

\begin{proof}
\forcenewline
\begin{itemize}
\item[а)] Необходимость

$\D f(z_0)=f'(z_0)\D z+o(\rho), \rho \to 0, f'(z_0)=A+iB$

$\D f=\D u+i\D v$

$\D u(x_0,y0)+i\D v(x0,y0)=(A+iB)(\D x+i\D y)+\o_1(\rho)+i\o_2(\rho)$

$\D u=u(x_0+\D x,y_0+\D y)-u(x_0,y_0)$

Отделяем действительные и мнимые части:

$$\sys{\D u=A\D x-B\D y+\o_1(\rho) \\ \D v=B\D x+A\D y+\o_2(\rho)}_{\rho\to 0}$$

Заметим, что $\rho\to 0\ \LRA\ \sys{\rho_1 \to 0 \\ \rho_2 \to 0} \LRA \sys{\D x\to 0 \\ \D y\to 0} \then u(x,y),\ v(x,y)$ -- $\R^2$-дифференцируемы в т. $(x_0,y_0)$

Причём $A=\cfrac{\dd u}{\dd x}(x_0,y_0),\ -B=\cfrac{\dd u}{\dd y}(x_0,y_0)$

$B=\cfrac{\dd v}{\dd x}(x_0,y_0),\ A=\cfrac{\dd v}{\dd y}(x_0,y_0)$

\item[б)] достаточность -- TODO

% $u,v$ дифференцируемы и имеют место равенства. пишем условия дифф-сти, бла-бла-бла
\end{itemize}
\end{proof}
{\bfseries Следствие.}

$f'(z_0)=A+iB=\cfrac{\dd u}{\dd x}+i\cfrac{\dd v}{\dd x}=\cfrac{\dd u}{\dd x}-i\cfrac{\dd u}{\dd y}=\cfrac{\dd v}{\dd y}-i\cfrac{\dd u}{\dd y}=\cfrac{\dd v}{\dd y}+i\cfrac{\dd v}{\dd x}$

{\bfseries Задачи.}

1) Доказать, что условия Коши-Римана в полярной системе координат имеют вид:
$$\sys{\cfrac{\dd u}{\dd r}=\cfrac 1r\cfrac{\dd v}{\dd \f} \\ \cfrac{\dd v}{\dd r}=-\cfrac 1r\cfrac{\dd u}{\dd \f}}$$

2) Условия Коши-Римана $\LRA\ \cfrac{\dd f}{\dd \z}=0$ (если рассматривать $f(z,\z)$) (условие Даламбера-Эйлера)

\section{Аналитические функции и их свойства}

\subsection{Основные определения и свойства аналитических функций}

\opr{$f(z)$ -- аналитическая в обл. $D$ ($f\in A(D)$) $\LRA\ \forall z\in D\ \exists$ непрерывная $f'(z)$ }
\opr{$f\in A(z_0)$, если существует окрестность $U(z_0)\colon f\in A(U(z_0))$}
\opr{$f\in A(\bar{D})\ \LRA\ \exists D_1\colon \bar{D}\subset D_1,\ f\in A(D_1)$}

{\bfseries Утверждение 1.}

$f\in A(D)\then f\in \C(D)$

Если $f,g\in A(D)$, то
\begin{itemize}
\item[a)] $f\pm g\in A(D)$
\item[b)] $fg \in A(D)$
\item[c)] $\cfrac fg \in A(D\smallsetminus\{z\colon g(z)=0\})$
\end{itemize}

{\bfseries Утверждение 2.}

$f\in A(D), G\supset f(E), g\in A(G)\then g(f(z))\in A(D)$

{\bfseries Утверждение 3.}

$w=f(z)\in A(z_0), w_0=f(z_0), f'(z_0)\neq 0 \then z=\f(w)\in A(w_0)$

\subsection{Связь аналитичности с гармоничностью}

\opr{Функция $u(x,y)$ -- гармоническая в $D$ ($u\in H(D)$), если а) $u\in\C^2[D]$ и б) $\D u=\cfrac{\dd^2u}{\dd x^2}+\cfrac{\dd^2u}{\dd y^2}=0$ в $D$}

\opr{$u,v$ -- сопряжённые, если они удовлетворяют условиям Коши-Римана $$(1)\colon\sys{\cfrac{\dd u}{\dd x}=\cfrac{\dd v}{\dd y} \\ \cfrac{\dd u}{\dd y}=-\cfrac{\dd v}{\dd x}}$$}

{\bfseries Утверждение 4.}
\begin{itemize}
\item[а)] $u,v$ -- сопряжённые $\then\ u,v\in H(D)$
\item[б)] $f=u+iv\in A(D)\ \LRA\ u,v$ -- сопряжённые гармонические функции в $D$
\end{itemize}

\begin{proof}
\begin{itemize}
\item[а)] Возьмём уравнение (1) и дифференцируем =>
$\sys{\cfrac{\dd^2u}{\dd x^2}=\cfrac{\dd^2v}{\dd y\dd x} \\ \cfrac{\dd^2u}{\dd y^2}=-\cfrac{\dd^2v}{\dd x\dd y}}$
 => $\D u=0$ => $u\in H(D)$; аналогично для $v$
\item[б)] => $f\in A\LRA \sys{u,v\in D(D) \mbox{(дифференцируемые)} \\ \mbox{выполнено}\ (1)}$

<= очевидно
\end{itemize}
\end{proof}

\subsection{Задача определения аналитической функции по её части}

{\bfseries Утверждение 5.}

Пусть $u\in H(D)$. Тогда $\E$ единственная с точностью до аддитивной константы $v(x,y)$: $f=u+iv\in A(D)$. В частности, $v(x,y)=\mint{(x_0,y_0)}{(x,y)}-u'_ydx+u'_xdy+C$ (2)

\begin{proof}
$dv=v'_xdx+v'_ydy=-u'_ydx+u'_xdy$

Убедимся, что имеется полный дифференциал.

$=(u'_y)'_y=-u''_{yy},\ (u'_x)'_x=u''_{xx} \LRA\ u''_{xx}+u''_{yy}=0$, т.к. $u\in H(D)$

$\mint{(x_0,y_0)}{(x,y)}dv=v(x,y)-v(x_0,y_0)$ т.к. результат не зависит от выбора пути интегрирования.

В силу произвольности выбора $(x_0,y_0)\in D$ следует (2)
\end{proof}

{\bfseries Замечания.}
\begin{enumerate}
\item Аналогичное утверждение верно для $v(x,y)=\Im f$. При этом вместо формулы (2) будет что-то другое
\item Парную функцию можно находить не из (2), а непосредственно из системы (1).
\item Если $f(z)\neq 0$, то $f(z)=|f(z)|e^{i\Arg f(z)}$

$f\in A\ \LRA\ |f(z)|,\ \Arg f(z)$ -- сопряжённые гармонические функции -- неверно!
\end{enumerate}

\chapter{Конформные отображения}

\section{Геометрический смысл модуля и аргумента производной}

\subsection{Основные понятия конформных отображений}

\opr{Угол между гладкими кривыми, пересекающимися в точке $z_0$ -- это меньший угол между их касательными в этой точке. ($z_0\neq\infty$)}

\opr{Угол между пересекающимися в точке $z_0=\infty$ кривыми -- это угол в точке $\zeta_0=0$ между образами кривых при преобразовании $\zeta=\cfrac1z$}

\opr{Непрерывное отображение $f(z)$ называется конформным в точке $z_0$, если оно сохраняет углы между пересекающимися в этой точке гладкими кривыми (по величине и направлению).}

\opr{Отображение $f(z)$ называется конформным в области, если оно конформно во всякой точке этой области и взаимно-однозначно.}

$z_0\in\CC,\ \gamma$ -- гладкая кривая: $\gamma\colon z=\lambda{t},t\in[a,b]$

$z_0\in\gamma$, т.е. $\E\ t_0\colon z_0=\lambda{t_0}$

$\lambda'(t_0)$ -- вектор касательной к $\gamma$ в точке $z_0$

$\f=\arg \l'(t_0)$ -- угол наклона касательной к $\g$ в $z_0$ (т.е. угол между касательной и положительным направлением)

$z\in\g$

$\D z=z-z_0$

$\Gamma=f(\gamma),\ w_0=f(z_0),\ w=f(z);\ w_0,w\in\Gamma;\ \D w=w-w_0$

$\Gamma\colon w=\Lambda(t)=f(\l(t)),\ t\in[a,b]$

$\mlim{\D z\to0}\cfrac{|\D w|}{|\D z|}=k\neq0$ -- коэффициент линейного растяжения (сжатия) кривой $\g$ в точке $z_0$ при отображении $f(z)$

\subsection{Геометрический смысл модуля и аргумента $f'(z)$}

Пусть $f\in A(z_0),\ f'(z_0)\neq0$. Тогда можно показать, что $f$ отображает некоторую окрестность точки $z_0$ взаимно-однозначно на окрестность $w_0$.

\begin{theor}[7.1]
\forcenewline

Пусть $f\in A(z_0),\ f'(z_0)\neq0$, тогда
\begin{enumerate}
\item[а)] $|f'(z_0)|$ -- коэффициент растяжения в точке $z_0$ при отображении $f$
\item[б)] $\arg f'(z_0)$ -- угол поворота гладкой кривой, проходящей через $z_0$ при отображении $f$
\item[в)] отображение $f$ конформно в точке $z_0$
\end{enumerate}
\end{theor}

\begin{proof}
\forcenewline
\begin{enumerate}
\item[а)] Т.к. $\E f'(z_0)$, т.е. $\E \mlim{\D z\to0}\cfrac{\D w}{\D z}=f'(z)$. Тогда выполнено следующее равенство:
$$|f'(z_0)|=\mlim{\D z\to0}\left|\cfrac{\D w}{\D z}\right|=\mlim{\D z\to0}\cfrac{|\D w|}{|\D z|}=k(\neq0)$$
Причём это равенство независимо от выбора $\g$.
\item[б)] $\f=\arg \l'(z_)$

$\Phi=\arg\Lambda'(t_0)$

Тогда $\Lambda'(t_0)=f'(z_0)\l'(t_0)\then\arg\Lambda'(t_0)=\arg f'(z_0)+\arg\l'(t_0)$ (при надлежащем выборе промежутка для $\arg$)

$\Phi-\f=\arg f'(z_0)$
\item[в)] Рассмотрим $\g_1,\g_2$ -- две гладкие кривые, проходящие через $z_0$, $\Gamma_1,\Gamma_2$ -- их образы, проходящие через $w_0$

$\a=\arg f'(z_0)$

$\sys{\Phi_1-\f_1=\a \\ \Phi_2-\f_2=\a } \then \Phi_1-\Phi_2=\f_1-\f_2$
\end{enumerate}
\end{proof}

{\bfseries Замечание.}

Теперь ясно, что коэффициент сжатия не привязан к кривой.

\subsection{Основные свойства конформных отображений}

{\bfseries Утверждение 1.}

Если $f\in A(z_0), f'(z_0)\neq 0$, то отображение $f$ конформно в $z_0$.

{\bfseries Утверждение 2.}

Если $f$ однолистна в $D$, $f\in A(D), f'(z)\neq0 \forall z\in D$, то $f$ конформно отображает $D$.

\begin{theor}[7.1]
\forcenewline

Пусть $f\in A(D),\ f$ однолистна в $D$, $D$ -- область, тогда $G=f(D)$ -- область.
\end{theor}
{\bfseries Замечание.}

Требование $f'(z)\neq0$ можно опустить.
\begin{theor}[7.2]
\forcenewline

Пусть $D,G$ -- области в $(z), (w)$ соответственно с границами $\dd D, \dd G$ -- замкнутыми контурами и $f\in A(D)\cap C(\bar{D})$. Если $f\colon \dd D\leftrightarrow \dd G$ взаимно-однозначна с сохранением ориентации, то $f$ конформно отображает $D$ на $G$
\end{theor}
\begin{theor}[7.3, Риман]
\forcenewline

Пусть $D$ -- односвязная область в $\CC$, $\dd D$ содержит больше одной точки. Тогда существует конформное отображение $D$ на единичный круг (вообще говоря, не единственное).
\end{theor}

\section{Дробно-линейные функции}

\subsection{Определение, непрерывность}

\opr{$w=L(z),\ L(z)=\cfrac{az+b}{cz+d},\ a,b,c,d\in \C$ -- дробно-линейная функция. $\D=ad-bc\neq0$}

$z\neq-\cfrac dc\ \then\ \E L^{-1}\colon z=\cfrac{dw-b}{-cw+a}\in\{L\},\ \D^{-1}=\D\neq0$ (2)

$L\colon \C\smallsetminus\left\{-\cfrac dc\right\}\leftrightarrow \C\smallsetminus\left\{\cfrac ac\right\}$ взаимно-однозначно и взаимно непрерывно.

$\sys{w\left(-\cfrac dc\right)=\infty \\ w(\infty)=\cfrac ac} \then L\colon \CC \leftrightarrow \CC$

\subsection{Конформность отображения}

\begin{theor}
\forcenewline

\begin{enumerate}
\item $w=L(z)\in A\left(\C \smallsetminus\left\{-\cfrac dc\right\}\right)$
\item $L\colon \CC\leftrightarrow\CC$ конформно
\end{enumerate}
\end{theor}
\begin{proof}
\forcenewline
\begin{enumerate}
\item $w'=\cfrac{a(cz+d)-c(az+b)}{(cz+d)^2}=\cfrac {\D}{(cz+d)^2}\neq 0$
\item $\bsys{\E w'(z)\neq 0, z\neq-\cfrac dc \\ w=L(z) - \mbox{взаимно-однозначно}}\then \mbox{отображение\ конформно\ в\ }\C\smallsetminus\left\{-\cfrac dc\right\}$
Рассмотрим $z_0=-\cfrac dc$

Пусть гладкие кривые $\g_1, \g_2$ пересекаются в $z_0$ под углом $\f$.

$L(z_0)=w_0=\infty$

$L(\g_1)=\G_1,L(\g_2)=\G_2$

$\G_1,\G_2$ пересекаются в $w_0=\infty$ под углом пересечения их образов $C_1,C_2$ в т. $\zeta_0=0$ при отображении $\zeta=\cfrac 1w$.

$\zeta=\cfrac{cz+d}{az+b}\in\{L\}$ даёт конформное отображение в $z_0=-\cfrac dc$, т.е. угол между $C_1$ и $C_2$ равен $\f$ -- углу между $\G_1$ и $\G_2$.

Рассмотрение в $\infty$ сводится к аналогичному доказательству конформности $L^{-1}(w)$ в т. $w_0=\cfrac ac$
\end{enumerate}
\end{proof}

\subsection{Групповое свойство}

\begin{theor}[8.2]
\forcenewline

$\{L\}$ -- алгебраическая группа относительно операций суперпозиции
\end{theor}

\begin{proof}
\forcenewline
\begin{enumerate}
\item $\F L\in\{L\}\then L^{-1}\in\{L\}$
\item $\F L_1(\D_1),L_2(\D_2)\in\{L\}\then L_2(L_1(z))(\D=\D_1\D_2\neq0)\in\{L\}$
\end{enumerate}
\end{proof}

\subsection{Инвариантность двойного отношения}
\begin{theor}[8.3]
\forcenewline

$\F(z_1,z_2,z_3),(w_1,w_2,w_3)$, в каждой из которых нет совпадающих чисел, существует единственное дробно-линейное преобразование, что $w_k=L(z_k),\ k=1,2,3$, причём это $L$ можно получить из отношений
$$
\cfrac{z-z_1}{z-z_2}:\cfrac{z_3-z_1}{z_3-z_2}=\cfrac{w-w_1}{w-w_2}:\cfrac{w_3-w_1}{w_3-w_2}\ (3)
$$
\end{theor}

\begin{proof}
\forcenewline

$L_1(z)=\cfrac{z-z_1}{z-z_2}:\cfrac{z_3-z_1}{z_3-z_2}$

$L_2(w)=\cfrac{w-w_1}{w-w_2}:\cfrac{w_3-w_1}{w_3-w_2}$

$(3) \LRA L_1(z)=L_2(w) \LRA w=L^{-1}_2(L_1(z))$ по т. 8.2

$w=L(z),\ L\in\{L\}$

Единственность: Пусть существует $F\colon F(z_k)=w_k,\ k=1,2,3$

$L^{-1}$ -- дробно-линейная

$\cfrac{az_k+b}{cz_k+d}=z_k$

$cz_k^2+(d-a)z_k-b=0$, $k=1,2,3 \then $ квадратный многочлен с тремя корнями тождественно равен нулю, $\sys{c=0 \\ d=a \\ b=0}$, т.е. $L(z)=\cfrac{az+b}{cz+d}\equiv z \LRA F=L$
\end{proof}

{\bfseries Замечание.}

В (3) -- равенство т.н. двойных отношений, т.е. инвариантность относительно дробно-линейного преобразования.

\subsection{Круговое свойство}

\begin{theor}[8.4]
\forcenewline

Дробно-линейное преобразование переводит множество прямых либо окружностей в (возможно, другое) множество прямых либо окружностей.
\end{theor}
\begin{proof}
\forcenewline
\begin{enumerate}
\item Для линейного отображения $w=l(z)$ -- очевидно.
\item Рассмотрим преобразование $w=\Lambda(z)\equiv \cfrac 1z$.

Ранее было доказано, что уравнение
$$
(4)\ Az\z+\bar{B}z+B\z+D=0\ (B\bar{B}-AD>0)
$$
это уравнение окружности ($A\neq0$) или прямой ($A=0$).

$w=\cfrac 1z \LRA z=\cfrac 1w \LRA \bar{z}=\cfrac {1}{\bar{w}} \then (4)$

$Dw\bar{w}+Bw+\bar{B}\bar{w}+A=0\ (5)$

$B\bar{B}-AD>0$

т.е. (5) описывает множество прямых либо окружностей.
\item $L(z)=\cfrac{az+b}{cz+d}=\cfrac ac \cfrac {z+\frac ba+\frac dc-\frac dc}{z+\frac dc}=\cfrac ac\left(1+\cfrac{\frac ba-\frac dc}{z+\frac dc}\right)=\cfrac ac\left(1+\cfrac{bc-ad}{a(cz+d)}\right)$ ($a\neq 0,\ c\neq 0$)

$\bsys{w_1=l_1=cz+d \\ \L=\cfrac {1}{w_1}(=w_2) \\ w=l_2(w_2)}\then L=l_2\L l_1(z)$
\end{enumerate}
\end{proof}

\subsection{Сохранение симметрии}

\opr{$z_1, z_2$ -- симметричное отношение прямой $l$, если $l$ -- срединный перпендикуляр к отрезку с концами в $z_1, z_2$}
\opr{$z_1, z_2$ симметричны (сопряжены) относительно окружности $\g$, если
\begin{enumerate}
\item $z_1, z_2$ лежат на одном луче, исходящем из центра $z_0$ окружности $\g$
\item $|z_1-z_0||z_2-z_0|=R^2$
\end{enumerate}
Обозначение: $z_1=z_2^*$}

{\bfseries Утверждение 1 (свойства симметрии).}

Пусть $z_1, z_2$ симметричны относительно прямой либо окружности.
\begin{enumerate}
\item[а)] $(z^*)^*=z$
\item[б)] $z=z^* \LRA z\in l/\g$, иначе они расположены по разные стороны $l/\g$
\item[в)] $\g\colon$ Если $z\to z_0$, то $z^*\to\infty$, т.е. $z_0$ и $\infty$ симметричны (сопряжены) относительно окружности любого радиуса.
\end{enumerate}

{\bfseries Лемма.}

$z_1, z_2$ симметричны относительно прямой либо окружности тогда и только тогда, когда любая обобщённая окружность (т.е. окружность либо прямая), проходящая через $z_1, z_2$, ортогональна к $l(\g)$
\begin{proof}
\forcenewline
\begin{enumerate}
\item В случае прямой доказательство очевидно.
\item Рассмотрим произвольную окружность, проходящую через две эти точки. Проведём из $z_0$ касательную к этой окружности, $M$ -- точка касания. Из элементарной геометрии следует, что $OM^2=|z-z_0||z^*-z_0|=R^2 \then OM=R \then \g\perp k$

... доказательство геометрическое, см. учебник
\end{enumerate}
\end{proof}

\begin{theor}[8.5]
\forcenewline

Пусть $z,z^*$ симметричны относительно обобщённой окружности $\g$

$\Gamma=L(\g), w=L(z), w_1=L(z^*) \then w_1=w^*$ (относительно $\Gamma$)
\end{theor}

\begin{proof}
\forcenewline

$\g,z,z^*$

$L\colon \Gamma,w,w_1$

Рассмотрим в плоскости $(w)$ произвольную окружность $K\ni w,w_1$.

Пусть $k=L^{-1}(K)$, тогда по лемме $k$ -- прямая/окружность

$z,z^*\in k\then \g\perp k$. Тогда, в силу конформности $L$, $\Gamma\perp K \then w_1=w^*$
\end{proof}

\section{Степенная функция и обратная к ней}

\subsection{Степенная функция}

\opr{Степенная функция: $w=z^n, n\in \N, n>1$ (1)}
{\bfseries Утверждение 1.}

Степенная функция:
\begin{enumerate}
\item Непрерывна в $\CC$
\item Аналитична в $\CC$
\item Конформна в любой точке $z\in\C,z\neq0$
\item Конформно отображает $\{\a<\arg z<\b\}\ (0<\b-\a<\frac{2\pi}{n}$, в частности, $D_k=\left\{\cfrac{2(k-1)\pi}{n}<\arg z<\cfrac{2\pi k}{n}\right\}\to G=\C\smallsetminus\R^+$
\end{enumerate}
\begin{proof}
\forcenewline
\begin{enumerate}
\item $w(\infty)=\infty$ -- доопределим до непрерывности. В т. $z\neq\infty$ непрерывность очевидна
\item $\F z\in\C\ \E w'=nz^{n-1}$ непрерывна в $\C\then w\in A(\C)$
\item $w'=0\LRA z=0$

$\forall z\neq0\ w'\neq0\then$ конформна в т. $z$
\item конф. в обл.

Область однолистна

$z_1\neq z_2\then w_1\neq w_2$

$z_1^n=z_2^n\LRA |z_1|^ne^{i(\arg z_1)n}=|z_2|^ne^{i(\arg z_2)n}\LRA\sys{|z_1|=|z_2| \\ \arg z_1-\arg z_2=\cfrac{2\pi k}{n}}\then \{\a<\arg z<\b\}, \b-\a\leq \cfrac{2\pi}{n}$ -- множество однолистности

$\sys{z=|z|e^{i\arg z}\\w=z^n=|w|e^{i\arg w}=|z|^ne^{i(\arg z)n}}\then\sys{|w|=|z|^n \\ \arg w=n\arg z}$
\end{enumerate}
\end{proof}

\subsection{Функция $w=\sqrt[n]{z}$}

\opr{$w=\sqrt[n]{z}\LRA z=w^n,n\in N, n>1$}
\opr{Однозначная аналитическая в $D$ функция $f(z)$ -- регулярная однозначная ветвь многозначной функции $F(z)$ в $D$, если $\F z\in D$ значение $f(z)$ совпадает с одним из значений $F(z)$.}
{\bfseries Утверждение 2.}

Функция $w=\sqrt[n]{z}\ \F z\in\CC,z\neq 0;\infty$ имеет $n$ значений. В области $G=\C\smallsetminus\R^+$ допускает выделение $n$ однозначных регулярных ветвей $w_k$, причём $w_k\in A(G)$

$w_k'=\cfrac{1}{n(\sqrt[n]{z})^{n-1}_k}$, $w_k$ конформно отображает $G$ на $D_k$=$\left\{\cfrac{2\pi(k-1)}{n}<\arg w<\cfrac{2\pi}{n}k\right\}$

\begin{proof}
\forcenewline

$w(0)=0, w(\infty)=\infty$

$w=\sqrt[n]{z}=\sqrt[n]{|z|}e^{\cfrac{i\arg z+2\pi k}{n}}$

Фиксируем $k$. Выделяем однозначную ветвь $w_k=\left(\sqrt[n]{z}\right)_k$

По теореме о производной обратной функции

$w_k'=\cfrac{1}{nw^{n-1}}=\cfrac{1}{n\left(\sqrt[n]{z}\right)^{n-1}_k}\neq 0$ (4) $\F z\in\C\smallsetminus\R^+$ непрерывна в $G$.

Из (4) следует конформность отображения $w_k(z)\ \F z\in G$. Т.к. отображение взаимно-однозначно, то $w_k$ конформно в области $G$.
\end{proof}

\section{Показательная функция и обратная к ней}

\subsection{Показательная функция}

\opr{$e^z=e^x(\cos y+i\sin y)$}
{\bfseries Утверждение.}
\begin{enumerate}
\item $e^z\in A(\C)$
\item Осуществляет конформное отображение $\F z\in\C$
\item Конформно отображает $\{y<\Im z<y+2\pi\}$. В частности, $D_k=\{2\pi <\Im z<2\pi(k+1)$ отображается конформно на $G=\C\smallsetminus\R^+$
\end{enumerate}
\begin{proof}
\forcenewline
\begin{enumerate}
\item $(e^z)'=e^z\neq0\ \F z\in\C$
\end{enumerate}
\end{proof}

\subsection{Логарифмическая функция}

\opr{$w=\Ln z \LRA z=e^W$}

$\Ln z=\ln|z|+i\Arg z=\ln |z|+i(\arg z+2\pi k), k\in\Z$ (2)

{\bfseries Утверждение 2.}

Логарифмическая функция, определённая (2), $\F z\neq0$ является бесконечнозначной.

На множестве $G=\C\smallsetminus\R^+$ допускает выделение регулярных однозначных ветвей.

$w_k=(\Ln z)_k, w_k\in A(G)$, конформно отображает $G\to D_k$

$w_k'=\cfrac 1z\ \F z\in\C, z\neq0$

\chapter{Интегрирование функций комплексных переменных}

\section{Определение и свойства интегралов функций комплексных переменных}

\subsection{Изначальное определение}

Пусть $\g$ -- кусочно-гладкая простая кривая без особых точек.

$\g\colon\sys{z=z(t)\\t\in[\a,\b]}\ (1)$

$A=z(\a),\ B=z(\b)$

$f(z)$ -- однозначная ФКП $z\in\g$

$T\colon z_0=A,z_1,\ldots,z_n=B$

$z_k\in\g$

$\g=\overset{n}{\underset{k=1}{\cup}}\g_k$

$\D z_k=z_k-z_{k-1}$

$|\D z_k|=|z_k-z_{k-1}|=s_k$

$\l(T)=\underset{k}{\max}s_k$

$\zeta_k\in\g_k$

$\{\zeta_k\}$

Интегральные суммы:

$S(f,T)=\msum{k=1}{n}f(\zeta_k)\D z_k\ (2)$

$S_1(f,T)=\msum{k=1}{n}f(\zeta_k)|\D z_k|\ (3)$

$\mlim{\l(T)\to 0}S(f,T)=\mint{\g}{} f(z)dz\ (4)$

$\mlim{\l(T)\to 0}S_1(f,T)=\mint{\g}{} f(z)|dz|=\mint{\g}{} f(z)ds\ (5)$

Если существуют конечные пределы (4), (5), не зависящие от способа разбиения и выбора $\{\zeta_k\}$ -- это интеграл и интеграл первого рода $f(z)$ по $\g$.

\begin{theor}[11.1]
\forcenewline

Пусть $\g$ -- гладкая кривая, $f=u+iv$ -- непрерывная на $\g$ функция. Тогда существуют интегралы (4), (5), причём справедливы равенства:

$\mint{\g}{}f(z)dz=\mint{\g}{}(udx-vdy)+i\mint{\g}{}(vdx+udy)\ (6)$

$\mint{\g}{}f(z)dz=\mint{\a}{\b}f(z(t))z'(t)dt\ (7)$

$\mint{\g}{}f(z)|dz|=\mint{\g}{}uds+i\mint{\g}{}vds\ (8)$

$\mint{\g}{}f(z)|dz|=\mint{\a}{\b}f(z(t))|z'(t)|dt\ (9)$
\end{theor}
\begin{proof}
\forcenewline

(6): $S(f,T)=\msum{k=1}{n}f(\zeta_k)\D z_k=\msum{k=1}{n}(u_k+iv_k)(\D x_k+i\D y_k)=\msum{k=1}{n}(u_k\D x_k-v_k\D y_k)+i\msum{k=1}{n}(v_k\D x_k+u_k\D y_k)$

$|\D z_k|\to 0 \LRA\sys{\D x_k\to 0 \\ \D y_k\to 0}\then \mlim{\l(T)\to 0}S(f,T)=\mint{\g}{}(udx-vdy)+i\mint{\g}{}(vdx+udy)$

(7): используя параметризацию (1), получаем:

$(7)=\mint\a\b(u(x(t),y(t))x'(t)-v(x(t),y(t))y'(t))dt+i\mint\a\b(u(x(t),y(t))y'(t)+v(x(t),y(t))x'(t))dt=\mint\a\b(u+iv)(x'+iy')=\mint\a\b f(z(t))z'(t)dt$

(8): $S_1(f,T)=\msum{k=1}{n}f(\zeta_k)|\D z_k|=\msum{k=1}{n}(u_k+iv_k)s_k=\msum{k=1}{n}u_ks_k+i\msum{k=1}{n}v_ks_k$

(9): $|z_k|=s_k=\sqrt{\D x_k^2+\D y_k^2}$

$ds=\sqrt{dx^2+dy^2}=\sqrt{(x'(t))^2+(y'(t))^2}dt=|z'(t)|dt$

При $\l(T)\to 0$ получаем требуемое равенство.
\end{proof}

\begin{theor}[11.2]
\forcenewline
\begin{enumerate}
\item[а)] $\mint{\g(A\to B)}{}f(z)dz=-\mint{\g^{-1}(B\to A)}{}f(z)dz$

$\mint{\g}{}f(z)|dz|$ не зависит от направления интегрирования
\item[б)] Если $\g_1,\g_2$ составляют $\g$ без наложения, то $\mint{\g}{}f(z)dz=\mint{\g_1}{}f(z)dz+\mint{\g_2}{}f(z)dz$, причём интегралы в левой и правой частях существуют одновременно.
\item[в)] $\F a,b\in\C,\ f,g$ интегрируемы по $\g\then \mint{\g}{}(af(z)+bg(z))dz=a\mint{\g}{}f(z)dz+b\mint{\g}{}g(z)dz$
\item[г)] $f(z)$ интегрируема на $\g \then |f(z)|$ интегрируема на $\g$ и $\left|\mint{\g}{}f(z)dz\right|\leq\mint{\g}{}|f(z)||dz|\ (10)$
\end{enumerate}
\end{theor}
\begin{proof}
\forcenewline
\begin{enumerate}
\item[а)] Следует из теории вещественных криволинейных интегралов
\item[б)] Следует из теории вещественных криволинейных интегралов
\item[в)] Следует из теории вещественных криволинейных интегралов
\item[г)] $f\in C(\g)\then |f|\in C(\g)$, т.е. $|f(z)|$ интегрируема на $\g$. (10) следует из неравенства:

$|S(f,T)|=|\msum{k=1}{n}f(\zeta_k)\D z_k| \leq |f(\zeta_k)||\D z_k|=S_1(|f|, T)$ и предельного перехода.
\end{enumerate}
\end{proof}
{\bfseries Следствие 1.}

$|f(z)|\leq M \F z\in\g$, длина $\g=L\then |\mint{\g}{}f(z)dz|\leq ML$
\begin{proof}
$\left|\mint{\g}{}f(z)dz\right|\leq\mint{\g}{}|f(z)||dz|\leq M\mint{\g}{}ds=ML$
\end{proof}
{\bfseries Следствие 2.}

Интеграл и интеграл первого рода от $f(z)$ по $\g$ существует и имеет те же свойства, если $\g$ -- кусочно-гладкая кривая и $f(z)$ -- кусочно-непрерывная

{\bfseries Замечание.}

Формула среднего значения для интеграла ФКП, вообще говоря, не верна.

Контрпример: $\g\colon[0,2\pi]\subset\R,\ f(z)=e^{iz}$

$\mint{\g}{}f(z)dz=\mint{0}{2\pi}e^{ix}dx=\mint{0}{2\pi}\cos xdx+i\mint{0}{2\pi}\sin xdx=0$

Предположим, что $\E\zeta\colon\mint{\g}{}e^{iz}=e^{i\zeta}\mint{\g}{}dz\neq0$ -- противоречие.

\section{Интегральная теорема Коши}

\subsection{Случай односвязной области}

\begin{theor}
Пусть $f\in A(D),\ \g$ -- замкнутый контур в $D$. Тогда $\moint\g f(z)dz=0$ (1)
\end{theor}
\begin{proof}
Из теоремы 11.1: $\moint\g f(z)dz=\moint\g udx-vdy+i\moint\g vfx+udy=I_1+I_2$

$G=\Int \g$

$f\in A(D)\then \E f'(z)\in C(D)$

Используем формулу Грина:

$I_1=\moint\g udx-vdy=\underset{\bar{G}}{\iint}(-\cfrac{\dd v}{\dd x}-\cfrac{\dd u}{\dd y})dxdy=0$ (по усл. К/Р)

$I_2=\moint\g vdx+udy=\underset{\bar{G}}{\iint}(\cfrac{\dd u}{\dd x}-\cfrac{\dd v}{\dd y})dxdy=0$ (по усл. К/Р)
\end{proof}

{\bfseries Следствия.}
\begin{enumerate}
\item $f\in A(D)\then\moint{\dd D}f(z)dz=0$
\item $f\in A(D)\then\F z_1,z_2\in D\  \mint{z_1}{z_2}f(z)dz$ не зависит от пути интегрирования (лежащего в $D$)
\end{enumerate}

\subsection{Случай неодносвязной области}

\begin{theor}
Пусть $D$ -- $n$-связная область с границей $\dd G=\G\cup\g_1\cup\ldots\cup\g_{n-1}$, ориентированной положительно относительно $D$, и $f\in A(\bar{D})$. 

Тогда $\moint{\dd D}f(z)dz=0$, т.е. $\moint{\G}f(z)dz+\msum{k=1}{n-1}\moint{\g_k}f(z)dz=0$
\end{theor}
\begin{proof}
Проведём доказательство для двусвязной области, далее -- по индукции.

$A\in\G,\ B\in\g_1$

Сделаем разрез по $AB$, тогда область односвязна. По теореме 12.1

$\moint{\G}f(z)dz+\moint{AB}f(z)dz+\moint{\g_1}f(z)dz+\moint{BA}f(z)dz=0$, но $\moint{AB}f(z)dz=-\moint{BA}f(z)dz\then\mint{\dd D}{}f(z)dz=0$
\end{proof}

{\bfseries Следствие.}
В условиях т. 12.2

$\moint{\G}f(z)dz=\msum{k=1}{n-1}\moint{\g_k}f(z)dz$, (4)

где во всех интегралах интегрирование в одном и том же направлении.

\section{Интегральная формула Коши}

\subsection{И.Ф.К.}

\begin{theor}
Пусть $f\in A(D),\ \g$ -- замкнутый контур. Тогда $\F z_0\in\Int\g\subset D:$

$f(z_0)=\cfrac{1}{2\pi i}\moint{\g}\cfrac{f(z)}{z-z_0}dz$
\end{theor}
\begin{proof}
Так как $z_0$ внутри $D$, то существует $\g_\rho\colon|z-z_0|=\rho,\ \g_\rho\subset\Int\g$, тогда

$\moint{\g}\cfrac{f(z)}{z-z_0}dz=\moint{\g_\rho}\cfrac{f(z)}{z-z_0}$

$\moint{\g_\rho}\cfrac{dz}{z-z_0}=2\pi i$

$\moint{\g_\rho}\cfrac{f(z)}{z-z_0}dz-2\pi if(z_0)=\moint{\g_\rho}\cfrac{f(z)}{z-z_0}-\moint{\g_\rho}\cfrac{z_0}{z-z_0}dz=\moint{\g_\rho}\cfrac{f(z)-f(z_0)}{z-z_0}dz=$

$=\{z\in\g_\rho \LRA z=z_0+\rho e^{i\f}\}=\mint{0}{2\pi}\cfrac{f(z_0+\rho e^{i\f})-f(z_0)}{\rho e^{i\f}}i\rho e^{i\f}d\f=i\mint{0}{2\pi}(f(z)-f(z_0))d\f$

$\left|\moint{\g_\rho}\cfrac{f(z)}{z-z_0}dz-2\pi if(z_0)\right|=\left|i\mint{0}{2\pi}(f(z)-f(z_0))d\f\right|\leq\mint{0}{2\pi}\left|f(z)-f(z_0)\right|d\f<\mint{0}{2\pi}\cfrac{\varepsilon}{2\pi}d\f=\cfrac{\varepsilon}{2\pi}2\pi=\varepsilon$

Т.о. доказано, что существует $\mlim{\rho\to0}\moint{\g_\rho}\cfrac{f(z)}{z-z_0}dz=2\pi if(z_0)$ -- не зависит от $\rho$
\end{proof}

{\bfseries Следствие 1 (формула среднего значения).}

Пусть $\g_R=\{z\colon|z-z_0|=R\},\ \overline{\Int\g_R}\subset D,\ f\in A(D)$. Тогда 

$f(z_0)=\cfrac{1}{2\pi}\mint{0}{2\pi}f(z_0+Re^{i\f})d\f$

$f(z_0)=\cfrac{1}{2\pi R}\moint{\g_R}f(z)ds$
\begin{proof}
По (1) $f(z_0)=\cfrac{1}{2\pi i}\moint{\g_R}\cfrac{f(z)}{z-z_0}dz=\{z=z_0+Re^{i\f}\}=\cfrac{1}{2\pi i}\mint{0}{2\pi}\cfrac{f(z_0+Re^{i\f})}{Re^{i\f}}iRe^{i\f}d\f=\cfrac{1}{2\pi}\mint{0}{2\pi}f(z_0+Re^{i\f})d\f$
\end{proof}

{\bfseries Следствие 2.}

$\moint{\g}\cfrac{f(z)}{z-z_0}dz=\sys{2\pi if(z_0),\ z_0\in\Int \g \\ 0, z_0\in\Ext\g}$

\subsection{Принцип max модуля аналитической функции}

\begin{theor}
Пусть $f\in A(D),\ f\not\equiv const$. Тогда $\underset{z\in D}{\max}|f(z)|$ не может достигаться во внутренней точке $D$.
\end{theor}
\begin{proof}
Пусть $\E z_0\in D\colon\underset{z\in D}{\max}|f(z)|=|f(z_0)|=M \then \E K_0=\{|z-z_0|\leq R\}\subset D$

По (3) $M=|f(z_)|=\cfrac{1}{2\pi}\left|\mint{0}{2\pi}f(z_0+Re^{i\f})d\f\right|\leq\cfrac{1}{2\pi}\mint{0}{2\pi}|f(z_0+Re^{i\f}|d\f\leq\cfrac{1}{2\pi}M2\pi=M$
$\LRA$
$\mint{0}{2\pi}|f(z)|d\f=2\pi M$
$\LRA$
$\mint{0}{2\pi}(M-|f(z)|)d\f=0$

$g(\f)=M-|f(z_0+Re^{i\f})|\geq 0$ -- действительная непрерывная функция параметра $\f$ на $[0,2\pi]$

$\then M-|f(z)|=0\ \F z\in\g_R \then |f(z)|=M\ \F z\in\g_R$. То же верно для $\F \g_r,\ r<R \then |f(z)|\equiv M\ \F z\in K_0$

Покажем теперь, что $|f(z)|\equiv M\ \F z\in D$

Для любой $z^*\in D$ существует непрерывная кривая $l\subset D$, соединяющая $z_0$ с $z^*$.

Рассмотрим т. $z_1=\g_R\cap l\in D$

$\E K_{R_1}=\{|z-z_1|\leq R_1\}\subset D$

$|f(z_1)|=M\then |f(z)|=M\ \F z\in K_{R_1}$

$z_2=\g_{R_1}\cap l,\ \E K_{R_2}\ldots$

За конечное число шагов получаем $K_{R_n}=\{|z-z_n|\leq R_n\}\subset D$

$|f(z_n)|=M,\ z^*\in K_{R_n} \then |f(z^*)|=M$

Если $\rho(l,\dd D)=\d$, то можно взять все $R_k>\cfrac\d2>0$

Теперь докажем, что $f(z)\equiv const$ в $D$.

$|f(z)|^2=u^2+v^2=M^2$

$\sys{2uu_x'+2vv_x'=0 \\ 2uu_y'+2vv_y'=0}$

$v_x'=-u_y',\ v_y'=u_x'$

$\sys{uu_x'-vu_y'=0 \\ uu_x'+vu_y'=0}$ -- линейна относительно $(u_x', u_y')$.

$\D =u^2+v^2=M^2$

$M=0\LRA |f(z)|\equiv 0\LRA f(z)\equiv 0$

$M>0\then\sys{u'_x=0 \\ u'_y=0}\then u=const\then$(из условий К-Р) $v=const\then f=const$
\end{proof}
{\bfseries Следствие 1.}

Пусть $D$ ограничено, $\dd D$ -- граница, $f\in A(D)\cap C(\bar{D})$. Тогда $\mmax{\bar{D}}|f(z)|=\mmax{\dd D}|f(z)|$
\begin{proof}
$f$ непрерывна в $\bar{D}$, следовательно, $|f|$ непрерывен в $\bar{D}$. $D$ -- ограниченная замкнутая область, следовательно, $f$ достигает своё максимальное значение. Но это значение не может достигаться внутри области, следовательно, оно достигается на границе.
\end{proof}
{\bfseries Следствие 2.}

Пусть $D$ ограничено, $f_1,f_2\in A(D)\cap C(\bar{D})$. Если $f_1(z)=f_2(z)\F z\in\dd D$, то $f_1(z)\equiv f_2(z)\F z\in D$
\begin{proof}
$f(z)=f_1(z)-f_2(z)$ и применяем следствие 1.
\end{proof}
{\bfseries Следствие 3.}
Пусть $f\in  A(D),f\neq const,f\neq 0\F z\in D$. Тогда $\mmin{D}|f(z)|$ не может достигаться во внутренней точке $D$
\begin{proof}
$\cfrac{1}{f(z)}\in A(D);\ \max\left|\cfrac{1}{f(z)}\right|=\cfrac{1}{\min|f(z)|}$
\end{proof}

\section{Интеграл типа Коши. Бесконечная дифференцируемость аналитической функции.}

\subsection{Интеграл типа Коши}

\opr{$g(z)=\cfrac 1{2\pi i}\mint{\g}{}\cfrac{f(\zeta)}{\zeta-z}d\zeta,\ \g$ -- кусочно-гладкая кривая, $f\in C(\g)$}

\begin{theor}
Пусть $\g$ -- кусочно-гладкая кривая, $f\in C(\g)$. Тогда
\begin{enumerate}
\item[а)] $g(z)\in A(\C\smallsetminus\g)$
\item[б)] $\F n\in N\ \E g^{(n)}(z)=\cfrac{n!}{2\pi i}\mint{\g}{}\cfrac{f(\zeta)}{(\zeta-z)^{n+1}}d\zeta$
\end{enumerate}
\end{theor}
\begin{proof}
\forcenewline
\begin{enumerate}
\item[а)] следует из б) ($n=1,2$) ($\then$ непрерывна $f'(z)\F z\in \C\smallsetminus\g$)
\item[б)] индукция по $n$

$n=1$ Докажем, что $g'=\cfrac{1}{2\pi i}\mint{\g}{}\cfrac{f(\zeta)}{(\zeta-z)^{2}}d\zeta$

$\F z\notin\g\ \E U(z),\ \rho(\overline{U(z)},\g)=d>0$

Возьмём $z+\D z\in U(z)$

$$\cfrac1{\D z}(g(z+\D z)-g(z))-\cfrac{1}{2\pi i}\mint{\g}{}\cfrac{f(\zeta)}{(\zeta-z)^{2}}d\zeta=$$

(ИФК: $f(z_0)=\cfrac{1}{2\pi i}\moint{\g}{}\cfrac{f(z)}{z-z_0}dz$)

$$=\cfrac1{2\pi i\D z}\mint{\g}{}\left(\cfrac{f(\zeta)}{\zeta-z-\D z}-\cfrac{f(\zeta)}{\zeta-z}\right)d\zeta-\cfrac{1}{2\pi i}\mint{\g}{}\cfrac{f(\zeta)}{(\zeta-z)^{2}}d\zeta=\cfrac1{2\pi i}\mint{\g}{}f(\zeta)\left(\underset{=B_1}{\underbrace{\cfrac1{\D z}\left(\cfrac1{\zeta-z-\D z}-\cfrac1{\zeta-z}\right)-\cfrac1{(\zeta-z)^2}}}\right)d\zeta$$

$$B_1=\cfrac1{\D z(\zeta-z)}\left(\cfrac1{1-\frac{\D z}{\zeta-z}}-1\right)-\cfrac1{\zeta-z}^2\underset{\D z\to 0}{=}\cfrac1{\D z(\zeta-z)}\left(1+\cfrac{\D z}{\zeta-z}+O\left(\left(\cfrac{\D z}{\zeta-z}\right)^2\right)-1\right)-\cfrac1{(\zeta-z)^2}=O\left(\cfrac{\D z}{(\zeta-1)^3}\right)$$
\{списать у Тани\} %TODO

Переход $(n-1)\to n$

\{тоже списать у Тани\}
\end{enumerate}
\end{proof}

\subsection{}

\begin{theor}
$f\in A(D)\then \F z\in D\ \F n\in\N\ \E f^{(n)}(z)=\cfrac{n!}{2\pi i}\moint{\g}\cfrac{f(\zeta)}{(\zeta-z)^{n+1}}d\zeta$, где $\g$ -- произвольный замкнутый контур.
\end{theor}
\begin{proof}
$\F z\in D,\F \G \then$ инт. Коши $f(z)=\cfrac{1}{2\pi i}\moint{\g}\cfrac{f(\zeta)}{\zeta-z}d\zeta$

инт. Коши $\then$ интеграл типа Коши $\then$ 13.1
\end{proof}
{\bfseries Следствие 1 (оценка Коши для производных)}

$K_R=\{\zeta\colon|\zeta-z|\leq R\},\ \g_R=\{\zeta\colon|\zeta-z|=R\}$

$f\in A(\bar{K_R}),\ \mmax{\zeta\in\g_R}|f(\zeta)|=M_R$

Тогда $|f^{(n)}(z)|\leq\cfrac{n!M_R}{R^n}$
\begin{proof}
По (5) $|f^{(n)}(z)|=|\cfrac{n!}{2\pi i}\mint{\g_R}{}\cfrac{f(\zeta)}{(\zeta-z)^{n+1}}d\zeta|\leq \cfrac {n!}{2\pi}\mint{\g_R}{}\cfrac{|f(\zeta)|}{|\zeta-z|^{n+1}}d\zeta\leq\cfrac{n!M_r}{2\pi}\cfrac1{R^{n+1}}\mint{\g_R}{}d\zeta=\cfrac{M_Rn!}{R^{n+1}}$
\end{proof}

\subsection{Теорема Лиувилля и её следствия}

\opr{Однозначная функция $f(z)$ называется целой, если $f\in A(\C)$}
\begin{theor}[Лиувилль]
Если $f\in A(\C),\ |f(z)|\leq M\ \F z\in \C$, то $f=const$
\end{theor}
\begin{proof}
$\F z\in\C$, рассмотрим $\g_R$ с центром в точке $z$. По (6) $|f'(z)|\leq \cfrac MR\underset{R\to\infty}{\to} 0 \then |f'(z)|=0 \LRA f'(z)=0$

$f'(z)=u'_x+iv'_x=0\LRA \sys{u'_x=0 \\ v'_x=0} \then \sys{v'_y=0 \\ u'_y=0}$
\end{proof}
{\bfseries Следствие (основная теорема высшей алгебры).}

Многочлен $P_n(z)=a_0z^n+\ldots+a_n,\ a_k\in\C$ имеет хотя бы один корень в $\C$.
\begin{proof}
От противного. Пусть $P_n(z)\neq0\ \F z\in\C$. Рассмотрим $f(z)=\cfrac 1{P_n(z)}$. $f\in A(\C),\ f(z)$ ограничена в $C$.

Из того, что $|P_n(z)|\underset{z\to\infty}{\to}\infty$, следует $|f(z)|\underset{z\to\infty}{\to}0$

$|f|\leq M$, тогда по т. Лиувилля $f=const$ -- противоречие.
\end{proof}

\section{Первообразная и неопределённые интегралы. Теорема Морера}

\subsection{ }

\begin{theor}
$f\in C(D)$, для любого замкнутого контура $\g\subset D$

$\moint{\g}f(\zeta)d\zeta=0$ $(1)$

Тогда

$F(z)=\mint{z_0}{z}f(\zeta)d\zeta$ $(2)$

$F(z)\in A(D)\ \F z_0\in D$, причём $F'(z)=f(z)$
\end{theor}
\begin{proof}
Из условия (1) следует, что (2) не зависит от пути интегрирования, соединяющего $z_0$ и $z$, т.е. $F(z)$ -- однозначная функция $z$.
Зафиксируем $z_0$, для любого $z\in D$, возьмём $z+\D z\in U(z)$, причём $z$ и $z+\D z$ соединяет отрезок.
$$
\cfrac{1}{\D z}(F(z+\D z)-F(z))-f(z) = \cfrac{1}{\D z}\left( \mint{z_0}{z+\D z}f(\zeta)d\zeta-\mint{z_0}{z}f(\zeta)d\zeta \right)-f(z) = \left( \mint{z}{z+\D z}d\zeta=\D z\right)=
$$
\end{proof}

\chapter{Ряды аналитических функций}

\section{Равномерно сходящиеся функциональные ряды}

\subsection{Основные определения}

\opr{ФР -- $\msum{n=1}{\infty}f_n(z)\ (1),\ f_n(z)$ -- однозначные функции.}

\opr{ФР $(1)$ сходится в точке $z_0\LRA$ сходится числовой ряд $\msum{n=1}{\infty}f_n(z_0)$}

\opr{Множество точек, где сходится $(1)$ -- множество сходимости ФР -- $E$}

$\F z_0\in E\ \E$ сумма (2).

На $E$ определена однозначная функция $f(z)$

\opr{ФР $(1)\underset{E_0}{\rra}f(z)\LRA \F \e>0\ \E N\ \F n>N\ \F z\in E_0\ \left|\msum{k=1}{n}\{f_k(z)-f(z)\}\right|<\e$}

\subsection{Непрерывность и интегрируемость суммы ряда}

{\bfseries Утверждение.}

Пусть $\msum{n=1}{\infty}f_n(z)\rrae{E_0}f(z), |g(z)|\leq M\ \F z\in E_0$. Тогда $\msum{n=1}{\infty}f_n(z)g(z)\rrae{E_0}f(z)g(z)$
\begin{proof}
$|\msum{n=1}{\infty}f_n(z)g(z)-f(z)g(z)|=|\msum{n=1}{\infty}f_n(z)-f(z)||g(z)|<\e$
\end{proof}

\begin{theor}[16.1]
Пусть $f_n\in C(E_0)$, ФР $\rrae{E_0}f(z)$. Тогда $f(z)\in C(E_0)$.
\end{theor}
\begin{proof}
Аналогично вещественному случаю.
\end{proof}

\begin{theor}[16.2]
Пусть $\g\subset D$ -- кусочно-гладкая, $f_n(z)\in C(\g)$, ФР$\rrae{\g}f(z)$. Тогда $\mint{\g}{}\left(\msum{n=1}{\infty}f_n(z)\right)dz=\msum{n=1}{\infty}\mint{\g}{}f_n(z)dz$
\end{theor}
\begin{proof}
По т. 16.1 $f(z)\in C(\g)$, т.е. все интегралы существуют.

$\left|\msum{k=1}{n}\mint{\g}{}f_k(z)dz-\mint{\g}{}f(z)dz\right|=\left|\mint{\g}{}\left(\msum{k=1}{n}f_k(z)-f(z)\right)\right|\leq\mint{\g}{}\left|\msum{k=1}{n}f_k(z)-f(z)\right|dz\leq\cfrac \e L\mint{\g}{}dz=\e$
\end{proof}

\opr{ФР сходится равномерно внутри области $D$, если он равномерно сходится на любом компакте из $D$}

{\bfseries Замечание.}

В т. 16.1 при $E_0=D$~(область) условие равномерной сходимости на $D$ можно заменить на условие равномерной сходимости внутри $D$.

\section{Теоремы Вейерштрасса}

\begin{theor}[Первая теорема Вейерштрасса]
Пусть $\F n\ f_n\in A(D),\ \msum{n=1}{\infty}f_n(z)\rrae{D}f(z)$. Тогда \begin{enumerate}
\item[а)] $f(z)\in A(D)$
\item[б)] $\F k\in \N\ f^{(k)}(z)=\msum{n=1}{\infty}f_n^{(k)}(z)\ \F z\in D$
\end{enumerate}
\end{theor}
\begin{proof}
\begin{enumerate}
\item[а)] $\F z\in D\ \E \bar{U}(z)\subset D.\ \F$ замкнутой $\g\subset U(z),\ z\in int(\g)$ $\moint{\g}f(z)dz=\moint{\g}\left(\msum{n=1}
{\infty}f_n(z)\right)dz=\msum{n=1}{\infty}\moint{\g}f_n(z)dz=0$

$f(z)\in C(\bar{U}(z))\then f\in A(U(z))$, т.к. $z$ -- произвольная, то $f\in A(D)$
\item[б)] $\msum{n=1}{\infty}\cfrac{k!}{2\pi i}\cfrac{f_n(\zeta)}{(\zeta-z}^{k+1}$ (2)

$\zeta\in\g_p=\{\zeta\colon|\zeta-z|=g\}$

Из равномерной сходимости на $\g_p$ исходного ряда следует равномерная сходимость на $\g_p$ ряда (2).

Т.к. $\left|\cfrac{k!}{2\pi i}\cfrac1{\zeta-z}^{k+1}\right|=\cfrac{k!}{2\pi}\cfrac1{|\zeta-z|^{k+1}}=\cfrac{k!}{2\pi\rho^{k+1}}=\cfrac{k!}{M}$

По т. 16.2 $\moint{\g_p}\cfrac{k!}{2\pi}\cfrac{f(\zeta)}{(\zeta-z)^{k+1}}d\zeta=\msum{n=1}{\infty}\cfrac{k!}{2\pi i}\moint{\g}\cfrac{f_n(\zeta)}{(\zeta-z)^{k+1}}d\zeta=\msum{n=1}{\infty}f_n^{(k)}(z)$, т.е. $f^{(k)}(z)=\msum{n=1}{\infty}f_n^{(k)}(z)\ \F z\in D$
\end{enumerate}
\end{proof}

\begin{theor}[Вторая теорема Вейерштрасса]
Пусть $D$ -- ограниченная область с границей $\dd D$, $f_n\in A(D)\cap C(\bar{D})\ \F n$ и $\msum{n=1}{\infty}f_n(z)\rrae{\dd D}f(z)$.
Тогда \begin{enumerate}
\item[а)] $\rsum f_n(z)\rrae{D}f(z)$
\item[б)] $f\in A(D)\cap C(\bar{D})$
\end{enumerate}
\end{theor}
\begin{proof}
\begin{enumerate}
\item[а)] $g_{n,p}(z)=\msum{k=n+1}{n+p}f_k(z)\then g_{n,p}(z)\in A(D)\cap C(\bar{D})$

По следствию теоремы 13.2 $\mmax{z\in\bar{D}}|g_{n,p}(z)|=\mmax{z\in\dd D}|g_{n,p}(z)|<\e\ \F n>N\ \F p\in\N\ \F z\in\bar{D}$

Из р.сх. ряда (1) $\F \e>0\ \E N\colon\F n>N\ \F p\in\N$ на $\dd D\ \g_{n,p}(z)|<\e$ 
\item[б)] Т.к. имеем равномерную сходимость на $\bar{D}$, то по т.16.1 $\F n\ f_n()\in C(\bar{D})\ f(z)\in C(\bar{D})$. По т.16.3 $f\in A(D)$
\end{enumerate}
\end{proof}

\section{Степенные ряды}

\subsection{Множество сходимости степенного ряда}

\opr{$\ssum C_n(z-z_0)^n$}

\begin{theor}[Коши-Адамар]
Пусть $l=\overline{\mlim{n\to\infty}}\sqrt[n]{|C_n|}$. Тогда \begin{enumerate}
\item[а)] $l=0\then(1)$ сходится абсолютно на $\C$
\item[б)] $l=+\infty\then(1)$ сходится только в $z=z_0$
\item[в)] $0<l<+\infty\then\F z\colon|z-z_0|<\cfrac1l\ (1)$ сходится абсолютно, а $\F z\colon|z-z_0|>\cfrac1l\ (2)$ расходится.
\end{enumerate}
\end{theor}

{\bfseries Замечание.}

Если $\E \mlim{n\to\infty}\left|\cfrac{C_n}{C_{n+1}}\right|=R'$, то $R=R'$

$\cfrac1R=\overline{\mlim{n\to\infty}}\sqrt[n]{|C_n|}$

\subsection{Равномерная сходимость степенного ряда}

\begin{theor}[Абель]
Пусть в $z_1\neq z_0$ (1) сходится. Тогда \begin{enumerate}
\item[а)] $\F z\colon|z-z_0|<|z_1-z_0|$ (1) сходится абсолютно.
\item[б)] $\F z\colon|z-z_0|\leq\rho$, где $0<\rho<|z_1-z_0|$, ряд сходится равномерно на $\bar{K}_\rho$
\end{enumerate}
\end{theor}
\begin{proof}
\begin{enumerate}
\item[а)] Т.к. (1) сходится в $z_0$, $z_1$ лежит в $\bar{K}_R$ (по т. 17/1) $\then z\colon|z-z_0|<|z_1-z_0|$

$z\in K_R\then$ (по 17.1) сх. в $z$ абсолютно.
\item[б)] $|z-z_0|=|(z_1-z_0)|\left|\cfrac{z-z_0}{z_1-z_0}\right|\leq|z_1-z_0|\cfrac{\rho}{|z_1-z_0|}=|z_1-z_0|q,\ 0<q<1$

Т.к. $\cfrac{|z-z_0|}{|z_1-z_0|}\leq\cfrac{\rho}{|z_1-z_0|}<1$

Тогда $|C_n(z-z_0)^n|=|C_n||z-z_0|^n\leq|C_n||z_1-z_0|q^n$

Т.к. $\ssum C_n(z_1-z_0)^n$ сходится, то $|C_k(z_1-z_0)^n|<1\ \F n\geq n_0\then |C_n(z-z_0)^n|<q^n$

$\F z\in K_\rho\ \ssum q^nCx\then$равномерная сходимость (1) в $K_\rho$ по пр. Вейерштрасса.
\end{enumerate}
\end{proof}

{\bfseries Замечание.}

Из т. 17.2 следует, что ряд (1) равномерно сходится внутри $K_R$.

{\bfseries Следствия.}
\begin{enumerate}
\item $f(z)\in A(K_R)$
\item $\F\g\subset K_R$ возможно почленное интегрирование.
\item $\F k\in\N$ можно почленно дифференцировать (1) в $K_R\ k$ раз. При этом все полученные ряды имеют радиус сходимости $R$.
\begin{proof}
Возможно дифференцирование $\then$ достаточно доказать неизменяемость $R$ для $k=1$.

$f'(z)=\rsum C_n n(z-z_0)^{n-1}=\ssum C_{n+1}(n+1)(z-z_0)^n$

$R^{-1}_1=\overline{\mlim{n\to\infty}}\sqrt[n]{|C_{n+1}(n+1)|}=\overline{\mlim{n\to\infty}}\sqrt[n]{n+1}\sqrt[n]{|C_n+1|}=\overline{\mlim{n\to\infty}}\sqrt[n]{|C_n+1|}=\overline{\mlim{n\to\infty}}\sqrt[n]{|C_n|}=R^{-1}$
\end{proof}
\item $C_0=f(z_0),\ C_n=\cfrac{f^{(n)}(z_0)}{n!}$ (2), где $f(z)$ -- сумма (1) ($R>0$)

$f(z)=\ssum C_n(z-z_0)^n$
\begin{proof}
$c_0=f(z_0)$

$f'(z)=\rsum nC_n(z-z_0)^{n-1}$

$z=z_0\then f'(z_0)=C_1$

$\ldots$

$f^{(k)}(z)=\msum{n=k}{\infty}C_nn(n-1)\ldots(n-k+1)(z-z_0)^{n-k}$

$z=z_0\then f^{(k)}(z_0)=k!C_k$
\end{proof}
\opr{Ряд (1) с коэффициентами вида (2) -- ряд Тейлора функции $f(z)$}
\item Пусть два ряда, сходящиеся в $K_R$, имеют одинаковые суммы в $K_R$. Тогда эти ряды совпадают.
\end{enumerate}

\subsection{Теорема Тейлора}

\subsection{Методы разложения в степенной ряд}

Они те же, что и в вещественном случае.
\begin{enumerate}
\item Использование формул для коэффициентов ряда Тейлора.

$C_n=\cfrac{f^{(n)}(z_0)}{n!}\ (n=0,1,\ldots)$
\item Использование основных разложений
\item Использование почленного интегрирования и дифференцирования.
\end{enumerate}

\section{Теорема единственности. Понятие аналитического продолжения}

\subsection{Нули аналитической функции}

\opr{$z_0$ -- нуль функции $f(z)$, если $f(z_0)=0$. ВНЕЗАПНО, блджад.}

Если $f\in A(z_0)$, то $\E K_R=\{|z-z_0|<R\}$, где $f(z)=\ssum c_n(z-z_0)^n$ (1)

\opr{$z_0$ -- нуль порядка $k$ аналитической функции $f(z)$, если $c_0=\ldots=c_{k-1}=0,c_k\neq0$ в разложении $(1)$.}

Нуль первого порядка называют простым нулём.

{\bfseries Утверждение.}

$z_0$ -- нуль $k$-го порядка тогда и только тогда, когда $f(z_0)=\ldots=f^{(k-1)}(z_0)=0,\ f^{(k)}(z_0)\neq 0$, что эквивалентно $f(z)=(z-z_0)^k\f(z)$, где $\f\in A(z_0),\ \f(z_0)\neq0$
\begin{proof}
Первое очевидно следует из разложения в ряд Тейлора.

$f(z)=c_k(z-z_0)^k+c_{k+1}(z-z_0)^{k+1}+\ldots=(z-z_0)^k\underset{\f(z)}{\underbrace{(c_k+c_{k+1}(z-z_0)+\ldots)}}$
\end{proof}

\subsection{Основная теорема.}

\begin{theor}
\forcenewline

$f\in A(D),\ D_0\subset D$

$D_0$ имеет хотя бы одну предельную точку в $D$.

Если $f(z)=0\ \F z\in D_0$, то $f(z)\underset{z\in D}{\equiv}0$
\end{theor}

\begin{proof}
\forcenewline

Пусть $a$ -- предельная точка множества $D_0,\ a\in D$. 

Рассмотрим разложение $f(z)$ в степенной ряд: $f(z)=\ssum c_n(z-a)^n,\ z\in K_0=\{|z-a|<R_0\}\subset D$ (2)

$\E \{z_n\}\in D_0\cap K_0,\ z_n\neq a,\ \mlim{n\to\infty}z_n=a$

Из (2) следует, что $0=f(z_n)=\ssum c_n(z_n-a)^n$. Т.к. $f(z)$ непрерывна, то из $z_n\to a\then \underset{=0}{f(z_n)}\to f(a)\then f(a)=0\then c_0=0$ в (2), т.е. $f(z)=(z-a)\underset{\f_1(z)}{\underbrace{(c_1+c_2(z-a)+\ldots)}}$, где $\f_1\in A(K_0)$.

Имеем $\F n\ 0=f(z_n)=(z_n-a)\f_1(z_n) \then \f_1(z_n)=0\ \F n$, по непрерывности $\f_1(z_n)\to\f_1(a)=0\then c_1=0$, et cetera.

Итого $c_i=0\ \F i\then f(z)=0\ \F z\in K_0$.

Покажем, что $\F z^*\in D\ f(z^*)=0$.

Т.к. $D$ -- связное множество, существует кривая $\g\subset D$, соединяющая $a$ и $z^*$.

$a_1=\dd K_0\cap\g$

$\E\ K_1=\{|z-a_1|<R_1\}\subset D$

$\E\ \{z'_n\}\in K_0,\ z'_n\to a_1$

$f(z'_n)=0\then f(z)=0\ \F z\in K_1$

Т.о. за конечное число шагов $\E K_n=\{|z-a_n|<R_n\}\subset D,\ \F z\in K_n\ f(z)=0$
\end{proof}

{\bfseries Следствие 1.}

Существует не более одной аналитичной в $D$ функции, принимающей заданные значения на $D_0\subset D,\ D_0$ имеет предельную точку в $D$.
\begin{proof}
1. Может не существовать такой функции.

2. Пусть $f_1(z),f_2(z)\in A(D),\ f_1(z)=f_2(z)\ \F z\in D_0$. Тогда применим т. 18.1 к функции $f(z)=f_1(z)-f_2(z)\then f(z)\equiv 0$ в $D$.
\end{proof}

{\bfseries Следствие 2.}

$f\in A(D),\ f\not\equiv0,\ E\subset D$ -- ограниченная, замкнутая.

Тогда на $E\ f(z)$ имеет не более конечного числа нулей.

{\bfseries Следствие 3.}

Целая функция имеет в $C$ не более чем счётное множество нулей. Если их количество бесконечно, то $z=\infty$ -- их предельная точка.

\subsection{Понятие об аналитическом продолжении}

Пусть $f_1\in A(D_1),\ f_2\in A(D_2),\ D_1\cap D_2=D_{12}\neq\emptyset$. Если $f_1(z)=f_2(z)\ f(z)\in D_{12}\then$
% $f_1\equiv f_2,\ z\in D_1\cup D_2$
$f=\sys{f_1(z),&z\in D_1 \\ f_2(z),&z\in D_2\smallsetminus D_1}$

\opr{$f_2$ -- аналитическое продолжение $f_1(z)$ с $D_1$ на $D_2$ (а $f_1$ -- аналитическое продолжение $f_2(z)$ с $D_2$ на $D_1$).}

\chapter{Ряды Лорана и особые точки ФКП}

\section{Ряды Лорана}

\subsection{Основные понятия}

\opr{Ряд Лорана -- $\lsum c_n(z-z_0)^n,\ c_n\in\C\ (1)$

$\lsum c_n(z-z_0)^n=\ssum c_n(z-z_0)^n$ (правильная часть) $+\rsum c_{-n}(z-z_0)^{-n}\ (2)$ (главная часть)}

\opr{Ряд $(1)$ сходится $\LRA$ сходятся оба ряда $(2)$}

\subsection{Сумма ряда Лорана и её свойства}

\begin{theor}
\forcenewline
Пусть $R^{-1}=\overline{\mlim{n\to\infty}}\sqrt[n]{|c_n|},\ r=\overline{\mlim{n\to\infty}}\sqrt[n]{|c_{-n}|},\ 0\leq r<R\leq +\infty$.

Тогда ряд $(1)$ сходится абсолютно, равномерно в кольце $K_{r,R}=\{r<|z-z_0|<R\}\ \F z\in K_{r,R}$, сумма $(1)\ f(z)\in A(K_{r,R})$
\end{theor}
\begin{proof}
\forcenewline
$f_1(z)=\ssum c_n(z-z_0)^n\ (3)$

Из т. 17.1 следует абсолютная сходимость $(3)$ в $\{|z-z_0|<R\}$

По теореме 17.2 $(3)$ сходится равномерно внутри $\{|z-z_0|<R\}\then f_1(z)\in A\{|z-z_0|<R\}$.

$f_2(z)=\rsum c_{-n}(z-z_0)^{-n}\ (4)$

$\zeta=\cfrac1{z-z_0}\then f_2(z)=\rsum c_{-n}\zeta^n\ (5)$

$(5)$ сходится абсолютно $\F \zeta\colon|\zeta|<\cfrac1r\LRA|z-z_0|>r$ сходится абсолютно $(4)$.

$(5)$ равномерно сходится внутри $\{|\zeta|<\cfrac1r\}$

$(z>0)\then z\neq z_0\then (4)$ сходится равномерно внутри $\{|z-z_0|>r\}$

$g(\zeta)\in A(\{|\zeta|<\cfrac1r\}$ -- сумма $(5)\then f_2(z)=g(\cfrac1{z-z_0})\in A\{|z-z_0|>r\}$

Ряд $(1)$ сходится $\LRA$ сходятся оба ряда $(2)$.

Случай $r=0$ рассмотреть самостоятельно.
\end{proof}

{\bf Следствие.}

$f(z)$ можно почленно дифференцировать в $K_{r,R}$ и интегрировать $\F\g\subset K_{r,R}$.

\subsection{Теорема Лорана.}

\begin{theor}[17.2]
\forcenewline
$f\in A(K_{r,R})\then f(z)=\lsum c_n(z-z_0)^n\ \F z\in K_{r,R}$, причём разложение единственно.
\end{theor}
\begin{proof}
\forcenewline
Рассмотрим $f\in A(K_{r,R}$. Возьмём точку $z\in K_{r,R}$.

$\E r_1,R_1\colon r<r_1<R_1<R,\ z\in K_{r_1,R_1}$

$\E \rho\colon \g_\rho=\{\zeta\colon|\zeta-z|=\rho\}\subset K_{r_1,R_1}$

\forcenewline
\forcenewline
ИТК для составного контура:

$\moint{\g_{R_1}(ccw)}\cfrac{f(\zeta)}{\zeta-z}d\zeta+\moint{\g_{R_2}(cw)}\cfrac{f(\zeta)}{\zeta-z}d\zeta+\moint{\g_\rho(cw)}\cfrac{f(\zeta)}{\zeta-z}d\zeta=0$

$\cfrac1{2\pi i}\moint{\g_\rho(ccw)}\cfrac{f(\zeta)}{\zeta-z}d\zeta=\cfrac1{2\pi i}\moint{\g_{R_1}(ccw)}\cfrac{f(\zeta)}{\zeta-z}d\zeta-\cfrac1{2\pi i}\moint{\g_{R_2}(ccw}\cfrac{f(\zeta)}{\zeta-z}d\zeta=f(z)$

$\zeta\in\g_{R_1}\colon \cfrac1{\zeta-z}=\cfrac1{\zeta-z_0-(z-z_0)}=\cfrac1{\zeta-z_0}\cfrac1{1-\cfrac{z-z_0}{\zeta-z_0}}=\cfrac1{\zeta-z_0}\rsum\cfrac{(z-z_0)^n}{(\zeta-z_0)^n}$

$|\zeta-z_0|>|z-z_0|\then \cfrac{|z-z_0|}{|\zeta-z_0|}=\cfrac{|z-z_0|}{R_1}\equiv q_1<1$

$\cfrac1{2\pi i}\moint{\g_{R_1}}\cfrac{f(\zeta)}{\zeta-z}d\zeta=
\cfrac1{2\pi i}\moint{\g_{R_1}}f(\zeta)\rsum\cfrac{(z-z_0)^n}{(\zeta-z_0)^{n+1}}d\zeta=
\rsum\underset{=c_n}{\underbrace{\left(\cfrac1{2\pi i}\moint{\g_{R_1}}\cfrac{f(\zeta)}{(\zeta-z_0)^{n+1}}d\zeta\right)}}(z-z_0)^n=
\rsum c_n(z-z_0)^n\equiv f_1(z)$

$\zeta\in\g_{r_1}\colon |\zeta-z_0|<|z-z_0|$

$\cfrac1{\zeta-z}=\cfrac1{\zeta-z_0-(z-z_0)}=\cfrac1{\zeta-z_0}\cfrac1{1-\cfrac{z-z_0}{\zeta-z_0}}=\cfrac1{\zeta-z_0}\rsum\cfrac{(z-z_0)^n}{(\zeta-z_0)^n}$

$\cfrac{|\zeta-z_0|}{|z-z_0|}=\cfrac{r_1}{|z-z_0|}\equiv q_2<1$

Единственность:

Предположим, что $f(z)$ соответствуют два ряда: $\lsum c_n(z-z_0)^n=\lsum c'_n(z-z_0)^n$

Домножим обе части на $(z-z_0)^{-k-1}\ \F k\in\Z$

$\lsum c_n(z-z_0)^{n-k-1}=\lsum c'n(z-z_0)^{n-k-1}$

$\moint{\g}\lsum c_n(z-z_0)^{n-k-1}=\moint{\g}\lsum c'n(z-z_0)^{n-k-1}$

$\moint{\g}(z-z_0)^{n-k-1}dz=\sys{0,& \F n-k-1\neq-1\\2\pi i,&n-k-1=-1\LRA n=k}$

Т.о. $\F k\in\Z$ $c_k=c'_k$
\end{proof}

\section{Изолированные особые точки однозначной ФКП и их классификация}

\subsection{Основное определение}

Пусть $f(z)$ -- однозначная, $f\in A(\dot{K}_R(z_0))$.

$\dot{K}_R(z_0)=K_{0,R}=\{z\colon0<|z-z_0|<R\},\ z_0\neq\infty;\ \dot{K}_R(\infty)=\{z\colon B<|z|<\infty\}$
\opr{$z_0$ -- правильная точка $f(z)$, если $f\in A(z_0)$.}
\opr{$z_0$ -- изолированная особая точка $f(z)$, если $f\not\in A(z_0)$}
$f(z)=\lsum c_n(z-z_0)^n=f_{\mbox{пр}(z)}+f_{\mbox{гл}}(z),\ z\in\dot{K}$
\opr{$z_0$ -- устранимая особая точка, если $f_{\mbox{гл}}(0)\equiv0$.}
\opr{$z_0$ -- полюс $f(z)$, если $f_{\mbox{гл}}(z)$ содержит конечное число членов.}
\opr{$z_0$ -- полюс $f(z)$ порядка $k$, если $c_k\neq0,\ c_n=0\ \F n>k$.}
\opr{$z_0$ -- существенная особая точка $f(z)$, если $f_{\mbox{гл}}(z)$ содержит бесконечное число членов.}

\subsection{Поведение ФКП в окрестности изолированной особой точки}

$z_0$ -- изолированная точка $f(z)$.
\begin{theor}[20.1]
$z_0$ -- УОТ $\LRA f(z)$ ограничена в $\dot{K}(z_0)$.
\end{theor}
\begin{proof}
\forcenewline
а) $z_0$ -- УОТ $\then f(z)=\ssum c_n(z-z_0)^n\then S(z)\in A(z_0),\ \E \mlim{z\to z_0}f(z)=c_0\neq\infty$

$S(z)=f(z)$ в $K \then f(z)$ огр. в $K$.

б) $|f(z)|\leq M\ \F z\in\dot{K}(z_0)$

$c_n=\cfrac{1}{2\pi i}\moint{\gamma}\cfrac{f(\zeta)}{(\zeta-z)^{n+1}}d\zeta$

$\g=\g_\rho=\{\zeta\colon|\zeta-z_0|=\rho\}\in\dot{K}$

$|c_n|=\cfrac{1}{2\pi}\moint{\gamma}\cfrac{|f(\zeta)|}{|\zeta-z|^{n+1}}|d\zeta|\leq \cfrac{M}{2\pi\rho^{n+1}}\moint{\g_\rho}ds=\cfrac{M}{\rho^n}\to0 \F0<\rho<\rho_0$

Т.е. $c_n=0,\ n<0\LRA f_{\mbox{гл}}=0$
\end{proof}

{\bf Следствие.}

$z_0$ -- УОТ $\LRA$ существует конечный $\mlim{z\to z_0}f(z)$.

\begin{theor}[20.2]
$z_0$ -- полюс $f(z)\LRA\mlim{z\to z_0}f(z)=\infty$
\end{theor}
\begin{proof}
а) $z_0$ -- полюс $\then z\in\dot{K}(z_0)$

$f(z)=\cfrac{c_{-k}}{(z-z_0)^k}+\ldots+c_0+c_1(z-z_0)+\ldots=\cfrac{1}{(z-z_0)^k}(c_{-k}+c_{-k+1}(z-z_0)+\ldots)=\cfrac{1}{(z-z_0)^k}\f(z),\ c_{-k}\neq 0$

$\f(z)\in A(z_0),\ \f(z_0)=c_{-k}\neq0$

б) $\mlim{z\to z_0}f(z)=\infty\then \E$ окрестность т. $z_0$, в которой $f(z)\neq0$.

В этой окрестности $U$ рассмотрим $g(z)=\sys{\cfrac1{f(z)},&\ z\neq z_0 \\ 0, &z=z_0},\ \mlim{z\to z_0} g(z)=0$

$g\in A(z_0)\then$ в $U$

$g(z)=b_k(z-z_0)^k+\ldots=(z-z_0)^k(b_k+b_{k+1}(z-z_0)+\ldots)=(z-z_0)^k\f_1(z)$

$f_1(z)\in A(z_0)$

$f_1(z_0)=b_k\neq 0$

$f(z)=\cfrac{1}{g(z)}=\cfrac1{(z-z_0)^k}\cfrac1{\f_1(z)}=\cfrac1{(z-z_0)^k}(a_0+a_1(z-z_0)+\ldots)$ -- ряд Лорана с конечным числом членов в главной части.
\end{proof}

{\bf Замечание.}

$z_0\neq \infty$ -- полюс $\LRA f(z)=\cfrac{\f(z)}{(z-z_0)^k},\ \z\in \dot{U}(z_0),\ \f\in A(z_0),\ \f(z_0)\neq 0$

{\bf Следствие.}

$z_0$ -- полюс порядка $k\ f(z) \LRA z_0$ -- нуль порядка $k\ g(z)=\sys{\cfrac1{f(z)},&z\neq z_0 \\ 0, &z=z_0}$

\begin{theor}[20.3]
$z_0$ -- СОТ $f(z) \LRA \not\E\mlim{z\to z_0}f(z)$
\end{theor}
\begin{proof}
Доказательство следует из т. 20.1, 20.2 и определений.
\end{proof}

\begin{theor}[20.3, Сохоцкий-Вейерштрасс]
Пусть $z_0$ -- СОТ $f(z)$. Тогда $\F A\in\CC\ \E\{z_n\},\ z_n\lra{n\to\infty}z\colon f(z_n)\lra{n\to\infty}A$
\end{theor}
\begin{proof}
а) $A=\infty$

Утверждение теоремы следует из неограниченности $f(z)$ в $\dot{K}(z_0)$.

б) $A\neq\infty$

Пусть $\not\exists\{z_n\}\subset \dot{U}(z_0)\colon z_n\lra{n\to\infty}z_0$

$\then \E \e>0\colon\E \dot{U}=\{0<|z-z_0|<\delta\}\colon\F z\in \dot{U}\ |f(z)-A|\geq \e$

$g(z)=\cfrac1{f(z)-A}\in A(\dot{U}$

$|g(z)|\leq\cfrac1{|f(z)-A|}\leq\cfrac1\e,\ z\in\dot{U}$

$\then z_0$ -- УОТ $g(z) \LRA \E$ конечный $\mlim{z\to z_0}g(z)=b_0$

$1^\circ)\ b_0=0\then \E\mlim{z\to z_0}f(z)=\infty\then z_0$ -- полюс $f(z)$ -- противоречие.

$2^\circ)\ b_0\neq0\then\E$ конечный $\mlim{z\to z_0}f(z)=\cfrac1{b_0}+A \LRA z_0$ -- УОТ $f(z)$ -- противоречие.
\end{proof}

\chapter{Теория вычетов}

\section{Понятие вычета. Вычисление вычета}

\subsection{Определение и основное утверждение}

\opr{$f(z)$ -- однозначная, $f\in A(\dot{K}(z_0))$

$\cfrac1{2\pi i}\moint{\g}f(z)dz=\mres{z=z_0}f(z)$ -- вычет $f(z)$ относительно т. $z_0$

($\g\subset\dot{K},\ z_0\in\Int\g$, интегрирование в положительном направлении)}

\begin{theor}[21.1]
\forcenewline
$\mres{z=z_0}f(z)=\sys{c_{-1},&z_0\neq\infty \\-c_{-1},&z_0=\infty}$
\end{theor}
\begin{proof}
\forcenewline
а) $z_0\neq\infty$

$f\in A(\dot{K})\then f(z)=\lsum c_n(z-z_0)^n$

$c_n=\cfrac1{2\pi i}\moint{\g}\cfrac{f(\zeta)}{(\zeta-z_0)^{n+1}}d\zeta$

$n=-1\then c_{-1}=\cfrac1{2\pi i}\moint{\g}f(\zeta)d\zeta$

б) $z_0=\infty$

$f(z)=\lsum c_nz^n$

$\moint{\g}f(z)dz=\moint{\g}\left(\lsum c_nz^n\right)=\lsum c_n\moint{\g}z^ndz=$

$\moint{\g}z^ndz=\sys{2\pi i,&n=-1 \\ 0,&n\neq-1}$

$=-2\pi ic_{-1}$
\end{proof}

{\bf Следствие.}

$z_0\neq\infty$ -- правильная или УОТ $f(z)\then\mres{z_0}f(z)=0$

\subsection{Вычисление вычетов}

{\bf Утверждение 1.}

$z_0\neq\infty$ -- простой полюс $\then \mres{z_0}f(z)=\mlim{z\to z_0}f(z)(z-z_0)\ (3)$

\begin{proof}
\forcenewline
$f(z)=\cfrac{c_{-1}}{z-z_0}+c_0+c_1(z-z_0)+\ldots$

$(z-z_0)f(z)=c_{-1}+c_0(z-z_0)+\ldots$
\end{proof}

{\bf Утверждение 2.}

$f(z)=\cfrac{\f(z)}{\psi(z)}$, где $\f,\psi\in A(z_0),\ \psi(z_0)=0,\ \psi'(z_0)\neq0,\ \f(z_0)\neq 0$

Тогда $\mres{z_0}f(z)=\cfrac{\f(z_0)}{\psi'(z_0)}\ (4)$

\begin{proof}
$z_0$ -- простой полюс $f(z)$. По (3):

$\mres{z_0}=\mlim{z\to z_0}\cfrac{\f(z)}{\psi(z)}(z-z_0)=\mlim{z\to z_0}\cfrac{\f(z)}{\frac{\psi(z)-\psi(z_0)}{z-z_0}}=\cfrac{\mlim{z\to z_0}\f(z)}{\mlim{z\to z_0}\frac{\psi(z)-\psi(z_0)}{z-z_0}}=\cfrac{\f(z_0)}{\psi'(z_0)}$
\end{proof}

{\bf Утверждение 3.}

$z_0$ -- полюс порядка $k\ f(z)$. Тогда $\mres{z_0}f(z)=\cfrac{1}{(k-1)!}\mlim{z\to z_0}\left(f(z)(z-z_0)^k\right)^{(k-1)}$

\begin{proof}
$f(z)=\cfrac{c_{-k}}{(z-z_0)^k}+\ldots+\cfrac{c_{-1}}{z-z_0}+c_0+c_1(z-z_0)+\ldots$

$(z-z_0)^kf(z)=c_{-k}+c_{-(k-1)}(z-z_0)+\ldots+c_0(z-z_0)^k+\ldots$

$\cfrac{\dd^{k-1}}{\dd z^{k-1}}\left((z-z_0)^kf(z)\right)=(k-1)!c_{-k}+k(k-1)\ldots2c_0(z-z_0)+\ldots$

$\mlim{z\to z_0}\left((z-z_0)^kf(z)\right)^{(k-1)}=(k-1)!c_{-1}$
\end{proof}

{\bf Утверждение 4.}

$z_0=\infty$ -- нуль $f(z)$ порядка $k\geq2\then\mres{z_0=\infty}f(z)=0$
\begin{proof}
$f(z)=\cfrac{\f(z)}{z^k},\ \f\in A(K_\infty),\ \f(\infty)=\mlim{z\to\infty}\f(z)\neq0$

$\f(z)=c_0+\cfrac{c_1}z+\ldots,\ c_0\neq\infty\then f(z)=\cfrac{c_0}{z_k}+\ldots\then c_{-1}=0$
\end{proof}

\subsection{Теорема о полной сумме вычетов}

\begin{theor}[21.3]
\forcenewline
Пусть $z_1,\ldots,z_n$ -- особые точки $f(z)$ ($z_i<\infty$). Тогда $\msum{k=1}{n}\mres{z_k}f(z)+\mres{z_0=\infty}f(z)=0$
\end{theor}

\begin{proof}
$z_1,\ldots,z_n$ -- изолированные особые точки $f(z)$

$\E R>0\colon\underset{k}{\max}|z_k|<R,\ \g_R\colon\{|z|=R\}$

$\E\g_k\subset\int\g_R\colon z_k\in\int\g_k$

ИТК для составного контура:
$$
\moint{\g_R,CCW}f(z)dz=\msum{k=1}{n}\moint{\g_k,CCW}f(z)dz
$$
$$
\underset{\mres{z=\infty}f(z)}{\underbrace{\cfrac{1}{2\pi i}\moint{\g_R,CW}f(z)dz}}+\msum{k=1}{n}\underset{\mres{z=z_k}f(z)}{\underbrace{\cfrac{1}{2\pi i}\moint{\g_k,CCW}f(z)dz}}=0
$$
\end{proof}

\section{Основная теорема теории вычетов. Вычисление контурных интервалов}

\subsection{Основная теорема теории вычетов}

\begin{theor}[22.1]
Пусть $f(z)$ -- аналитическая в $\bar{D}$ -- ограниченной области за исключением особых точек $z_1,\ldots,\z_n$, не лежащих на границе $\dd D$. Тогда
$$
\moint{\dd D}f(z)dz=2\pi i\msum{k=1}{n}\mres{z=z_k}f(z) \leqno (1)
$$
\end{theor}
\begin{proof}
Так как о.т. конечное число и они изолированны, то $\E\{\g_k\},\ \g_k\subset D,\ z_k\in\Int\g_k$.

По ИТК для составного контура
$$
\moint{\dd D}f(z)dz=\msum{k=1}{n}\moint{\g_k}f(z)dz=2\pi i\msum{k=1}{n}\mres{z_k}f(z)
$$
Дальше -- очевидно.
\end{proof}
{\bf Замечание.}

$D$ может быть неодносвязной, тогда $\dd D$ -- полная граница.
\begin{theor}[22.$1^*$]
Пусть $f\in A(D)$, за исключением конечного числа о.т., $\g$ -- замкнутый контур, $\g\subset D$, не проходящий через о.т. Тогда
$$
\moint{\g}f(z)dz=2\pi i\msum{z_k\in\Int\g}{}\mres{z_k}f(z) \eqno (2)
$$
\end{theor}

\subsection{Вычисление интегралов вида $\int_0^{2\pi}R(\cos\f,\sin\f)d\f$}

$\cos\f=\cfrac{e^{i\f}+e^{-i\f}}{2}$

$\sin\f=\cfrac{e^{i\f}-e^{-i\f}}{2i}$

$\g\colon\sys{z=e^{i\f}\\\f\in[0,2\pi]}$

Пример.

$\mint{0}{\pi}\cfrac{d\f}{1+a\cos\f},\ -1<a<1$

$$\mint{0}{\pi}\cfrac{d\f}{1+a\cos\f}=\left\{\begin{matrix}z=e^{i\f}\\dz=ie^{i\f}d\f\\d\f=\frac{dz}{iz}\end{matrix}\right\}=\cfrac12\mint{-\pi}{\pi}\cfrac{d\f}{1+a\cos\f}=\cfrac12\moint{|z|=1}\cfrac{dz}{iz(1+\frac{a}{2}(z+\frac1z))}=\cfrac1i\moint{\g}\cfrac{dz}{az^2+2z+a}$$

$f(z)=\cfrac1{az^2+2z+a}$

$az^2+2z+a=0$

$z_{1,2}=\cfrac{-1\pm\sqrt{1-a^2}}{a}$

$|z_1^-|>1,\ |z_1^+|<1$

$z_1=\cfrac{-1+\sqrt{1-a^2}}{a}\in\Int\g$ -- полюс

$\mres{z_1}f(z)=\left.\cfrac1{2az+2}\right|_{z_1}=\cfrac1{2(az_1+1)}=\cfrac1{2\sqrt{1-a^2}}$

$$
\cfrac1i\moint{\g}\cfrac{dz}{az^2+2z+a}=2\pi i\cfrac1i\cfrac1{2\sqrt{1-a^2}}=\cfrac\pi{\sqrt{1-a^2}}
$$

{\bf Утверждение.}

$\mint{0}{2\pi}R(\cos\f,\sin\f)d\f=2\pi\msum{|z_k|<1}{}\mres{z_k}R_1(z)$, где $R_1(z)=\cfrac1zR(\cfrac12(z+\cfrac1z),\cfrac1{2i}(z-\cfrac1z))$

\section{Вычисление несобственных интегралов}

\subsection{Лемма}

\begin{lemma}[1]
Пусть $\bar{D}=\{|z|\geq R_0>0,\ \Im z\geq0\}$.

$C_R=\{|z|=R,\ 0\leq\arg z\leq\pi\}$

$f\in A(\bar{D}),\ M_f(R)=\mmax{z\in C_R}|f(z)|=\bar{\bar{o}}(\cfrac1R)$

Тогда $\mint{C_R}{}f(z)dz\lra{R\to\infty}0$.
\end{lemma}
\begin{proof}
$\left|\mint{C_R}{}f(z)dz\right|=\left.\sys{z=Re^{i\f}\\0\leq\f\leq\pi}\right\}=\left|\mint{0}{\pi}f(Re^{i\f})Rie^{i\f}d\f\right|\leq\mint{0}{\pi}\bar{\bar{o}}(\cfrac1R)Rd\f=\mint{0}{\pi}\bar{\bar{o}}(1)d\f$
\end{proof}

\subsection{Вычисление интегралов вида $\int_{-\infty}^{+\infty}\frac{P_m(z)}{Q_n(z)}dz$}

\begin{theor}[23.1]
Пусть $f$ -- аналитическая в $\{\Im z\geq0\}$ за исключением о.т. $z_1,\ldots,z_n$, не лежащих на вещественной прямой, $\underset{R\to\infty}{M_f(R)=\bar{\bar{o}}(\cfrac1R)}$. Тогда
$$
\mint{-\infty}{+\infty}f(x)dx=2\pi i\msum{k=1}{n}\mres{z_k}f(z)\eqno(1)
$$
(интеграл в (1) понимается в смысле v.p.)
\end{theor}
\begin{proof}
Так как о.т. конечное число, то $\E R_0>0\colon f(z)\in A(\bar{D})$.

$\g_R=C_R\cup[-\pi,\pi],\ R>R_0$

Рассмотрим $\moint{\g}f(z)dz=\mint{-R}{R}f(z)dz+\mint{C_R}{}f(z)dz=2\pi i\msum{k=1}{n}\mres{z_k}f(z)\ \F R>R_0$

При $R\to\infty$ $v.p.\mint{-\infty}{+\infty}f(x)dx=2\pi i\msum{k=1}{n}\mres{z_k}f(z)$
\end{proof}
{\bf Следствия.}

Пусть $f(z)=\cfrac{P_m(z)}{Q_n(z)},\ n-m\geq2,\ Q_n(x)\neq0$.

Тогда $\mint{-\infty}{+\infty}\cfrac{P_m(z)}{Q_n(x)}dx=2\pi i\msum{k=1}{n}\mres{z_k}R_{m,n}(z)\ |\ Q_n(z_k)=0,\ \Im z_k>0$

\subsection{Лемма Жордана}

$\bar{D}=\{|z|\geq R_0\}\cap\{\Im z\geq 0\}$

$C_R=\{|z|=R,0\leq\arg z\leq\pi\}$

$M_f(R)=\mmax{z\in C_R}|f(z)|$
\begin{lemma}[Жордан]
Пусть $f(z)\in C(\bar{D}),\ \underset{R\to\infty}{M_f(R)=\bar{\bar{o}}(1)}$

Тогда $\mint{C_R}{}f(z)e^{iaz}dz\lra{R\to\infty}0\ (a>0)$
\end{lemma}
\begin{proof}
$z\in C_R\LRA\sys{z=Re^{i\f}\\\f\in[0,\pi]}$

$|e^{iaz}|=|e^{iaRe^{i\f}}|=|e^{iaR(\cos\f+i\sin\f)}|=|e^{iaR\cos\f}||e^{-aR\sin\f}|=e^{-aR\sin\f}$

$|\mint{C_R}{}f(z)e^{iaz}dz|=|\mint{0}{\pi}f(Re^{i\f})e^{iaRe^{i\f}}Rie^{i\f}d\f|\leq\mint{0}{\pi}M_f(R)e^{-aR\sin\f}df=RM_f(R)\mint{0}{\pi}e^{-aR\sin\f}d\f=$

$\left.=2RM_f(R)\mint{0}{\frac\pi2}e^{-aR\sin\f}d\f\leq\{\sin\f\geq\frac2\pi\f,\ \f\in[0,\frac\pi2]\}\leq2RM_f(R)\mint{0}{\pi}e^{\frac{-2aR\f}{\pi}}d\f=2RM_f(R)\cfrac{\pi}{2aR}e^{\frac{-2aR\f}{\pi}}\right|^0_{\frac\pi2}$
\end{proof}

\subsection{Теорема 23.2}

\begin{theor}
Пусть $f(z)$ -- аналитическая на $\{\Im z\geq0\}$ за исключением точек $z_1,\ldots,z_n$, не лежащих на вещественной оси; $M_f(R)=\bar{\bar{o}}(1),\ R\to\infty$.

Тогда $\mint{-\infty}{+\infty}f(z)\cos axdx=\Re A$

$\mint{-\infty}{+\infty}f(z)\sin axdx=\Im A$

$A=2\pi i\msum{k=1}{n}\mres{z_k}\left(f(z)e^{iaz}\right)$
\end{theor}
\begin{proof}
Рассмотрим $\g_R=C_R\cup[-R,R],\ R>R_0>\mmax{1\leq k\leq n}|z_k|$

$F(z)=f(z)e^{iaz}\in A(\bar{D})$

По основной теореме о вычетах

$\moint{\g_R}F(z)dz=2\pi i\msum{k=1}{n}\mres{z_k}F(z)=A$

$\mint{-R}{R}F(x)dx+\underset{\lra{R\to\infty}0}{\underbrace{\mint{C_R}{}F(z)dz}}=A\ \F R>R_0$

$R\to\infty$

$v.p.\mint{-\infty}{+\infty}f(x)e^{iax}=A$

$v.p.\mint{-\infty}{+\infty}f(x)(\cos ax+i\sin ax)dx=A$
\end{proof}
{\bf Следствие.}

Если $f(x)=\cfrac{P_m(x)}{Q_n(x)},\ \underset{x\in\R}{Q_n(x)\neq0},\ n-m\geq1$, то

$\mint{-\infty}{+\infty}\cfrac{P_m(x)}{Q_n(x)}\cos axdx=\Re A$

$\mint{-\infty}{+\infty}\cfrac{P_m(x)}{Q_n(x)}\sin axdx=\Im A$
\end{document}

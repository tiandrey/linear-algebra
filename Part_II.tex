\chapter{*-Направленный отрезок и свободный вектор-*}
\chapter{*-Линейные операции над векторами-*}
\section{*-Сложение-*}
\section{*-Умножение на число-*}
\section{*Векторы как элементы вещественного линейного пространства*}
Пусть $\V\neq\varnothing$, и на $\V$ задано соответствие:

$\forall\ (a,b)\in\underbrace{\V\times\V}_{\scriptsize\mbox{Декартово произведение}}\longmapsto{}c\in\V$ называется алгебраической операцией или внутренним законом композиции.

$c=a*b$

$c=a\cdot{}b$

$c=a+b$ --- абстрактное сложение.

$\forall\ \alpha\in\mathbb{R},\ a\in\V\longmapsto{}b\in\V$ --- внешний закон композиции

$b=\alpha{}a$ --- абстрактное умножение.

\begin{opred}
Множество с введенными на нём внутренним и внешним законом композиции называется \mybold{вещественным линейным пространством $\mathbb{V}$}, если эти операции обладают следующими свойствами:

$\forall\ a,b,c\in\V, \alpha,\beta\in\mathbb{R}$
\begin{enumerate}
\item $a+b=b+a$
\item $(a+b)+c=a+(b+c)$
\item $\exists\ \T\in\V:\ a+\T=a$
\item $\exists\ (-a)\in\V:\ a+(-a)=\T$
\item $1a=a$
\item $\alpha(\beta{}a)=(\alpha\beta)a$
\item $(\alpha+\beta)a=\alpha{}a+\beta{}a$
\item $\alpha(a+b)=\alpha{}a+\alpha{}b$
\end{enumerate}
\end{opred}
\begin{opred}
$\V$ --- \mybold{векторное пространство}, если \begin{itemize}
\item элементы $\V$ --- векторы
\item $\V$ --- вещественное линейное пространство
\item $\T$ --- нулевой вектор
\item $(-a)$ --- противоположный вектор
\end{itemize}
\end{opred}
\begin{theor}
Простейшие свойства:\begin{enumerate}
\item $\T$ определен однозначно, $\forall\ a\in\V\ (-a)$ определен однозначно
\item $\forall\ a\in\V\colon 0a=\T$, $\forall\ \alpha\in\mathbb{R}\colon \alpha\T=\T$
\item из равенства $\alpha{}a=\T$ следует : $\left[\begin{aligned}
&\alpha=0 \\
&a=\T
\end{aligned}\right.$
\item $\forall\ a\in\V\colon -a=(-1)a$
\end{enumerate}
\end{theor}
\begin{proof}
\begin{enumerate}
\item Пусть $\exists\ \T_1,\T_2\in\V$

$\forall\ a\in\V\colon{}a+\T_1=a,\ a+\T_2=a$

$\T_1=\T_1+\T_2=\T_2+\T_1=\T_2$

Пусть $\exists\ a\in\V\colon\exists\ \(-a)_1,\(-a)_2\in\V$

$\T=a+(-a)_1=a+(-a)_2$

$(-a)_1+(a+(-a)_1)=(-a)_1+(a+(-a)_2)$

$\underbrace{((-a)_1+a)}_{\scriptsize\T}+(-a)_1=\underbrace{((-a)_1+a)}_{\scriptsize\T}+(-a)_2$

\item $a+0a=1a+0a=(1+0)a=1a=a$, т. о., $a+\underbrace{0a}_{\scriptsize\T}=a$

$\alpha\T=\alpha(0a)=(\alpha0)a=0a=\T$

\item $\alpha{}a=\T\Rightarrow\left[\begin{aligned}
& \alpha=0 \\
& \alpha\neq0\ \Rightarrow\ \frac1\alpha(\alpha{}a)=\frac1\alpha\T\Rightarrow a=\T
\end{aligned}\right.$
\item $(-1)a+a=(-1)a+1a=(-1+1)a=0a=\T$; по единственности $(-a)\ (-1)a=-a$
\end{enumerate}
\end{proof}
\chapter{*-Линейная зависимость-*}
\section{*-Теорема о линейной зависимости-*}
\section{*-Геометрический смысл линейной зависимости-*}
\chapter{*-Ранг матрицы-*}
\section{*-Арифметическое линейное пространство-*}
\section{*-Понятие ранга-*}
\section{*-Теорема о базисном миноре-*}
\section{*-Следствия из теоремы о базисном миноре-*}
\section{*-Метод Гаусса вычисления ранга-*}

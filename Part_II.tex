\chapter{*Направленный отрезок и свободный вектор*}
\begin{opred}
\mybold{Направленный отрезок} --- упорядоченная пара точек на плоскости.
\end{opred}
\begin{opred}
\mybold{Длина} $\vec{AB}$ - длина отрезка $AB$, если $A$ и $B$ различны, и число 0 в противном случае.
\end{opred}
\begin{opred}
$\vec{AB}$ \mybold{параллелен} прямой(плоскости), если $\left[\begin{aligned}
& A=B \\
& A\neq B,\ (AB)\ ||\ \mbox{прямой(плоскости)}
\end{aligned}\right.$
\end{opred}
\begin{opred}
Направленные отрезки $\vec{A_1B_1},\vec{A_2B_2},\vec{A_3B_3}$ называются коллинеарными (компланарными), если существует прямая(плоскость), которой они параллельны.
\end{opred}
\begin{remark}
Нулевой направленный отрезок коллинеарен и компланарен любому направленному отрезку.
\end{remark}
\begin{opred}
Два ненулевых направленных отрезка $\vec{A_1B_1}$ и $\vec{A_2B_2}$ называются \mybold{сонаправленными (противоположно направленными)}, если они коллинеарны и лучи $[A_1;B_1)$ и $[A_2,B_2)$ сонаправлены (противоположно направлены).
\end{opred}
\begin{opred}
2 напрaвленных отрезка $\vec{AB}$ и $\vec{CD}$ называются \mybold{равными}, если

$\left[\begin{aligned}
&\vec{AB}=\vec{0},\ \vec{CD}=\vec{0}\\
&\vec{AB}\neq\vec{0},\ \vec{CD}\neq\vec{0},\ |\vec{AB}|=|\vec{CD}|,\ \vec{AB}\uparrow\uparrow\vec{CD}
\end{aligned}\right.$
\end{opred}
\begin{remark}
Можно дать компактное эквивалентное определение: 2 направленных отрезка называются равными, если середины обычных отрезков $AD$ и $BC$ совпадают.
\end{remark}
\begin{opred}
\mybold{Свободный вектор (просто вектор)} --- множество всех равных между собой направленных отрезков. Если говорят, что вектор порождается направленным отрезком $\vec{AB}$, то пишут $\bar{a}=\vec{AB}$ и $\bar{a}=\{$все направленнык отрезки $\vec{CD}=\vec{AB}\}$. Все нулевые направленные отрезки называются нулевым вектором и обозначаются $\bar{0}$.

Для свободных векторов вводятся все термины, связанные с направленными отрезками: длина, параллельность прямой и плоскости, коллинеарность и компланарность, сонаправленность и противоположнонаправленность.
\end{opred}
\chapter{*Линейные операции над векторами*}
\section{*Сложение*}
\begin{opred}
Суммой $\bar{a}$ и $\bar{b}$ называется $\bar{c}$, определяемый по следующим правилам: (тут рисунок с правилом треугольника)
\end{opred}
\begin{theor}
Операция сложения векторов обладает следующими свойствами: \begin{enumerate}
\item $\va+\vb=\vb+\va,\ \forall\ \va,\vb$
\item $(\va+\vb)+\vc=\va+(\vb+\vc)$
\item $\forall\ \va\colon\va+\vo=\va$
\item $\forall\ \va\ \exists\ (-\va)\colon\va+(-\va)=\vo$
\end{enumerate}
\end{theor}
\begin{proof}
Доказывается построением.
\end{proof}
\begin{opred}
\mybold{Разностью} $\vb$ и $\va$ называется $\vx\colon\va+\vx=\vb$
\end{opred}
\begin{remark}
Для любых $\va,\vb$ разность всегда существует и определена единственным образом.
\end{remark}
\begin{proof}\begin{enumerate}
\item $\vx=\vb+(-\va)\colon\va+(\vb+(-\va))=\vb$
\item Пусть $\vc$ --- тоже разность $\vb$ и $\va$

$\vc=\vc+\vo=\vc+(\va+(-\va))=(\vc+\va)+(-\va)=\vb+(-\va)$
\end{enumerate}\end{proof}
\section{*Умножение на число*}
\begin{opred}
Вектор $\vb=\alpha\va$:\begin{enumerate}
\item $|\vb|=|\alpha||\va|$
\item Если $\vb\neq\vo$, то если $\alpha>0\colon\va\uparrow\uparrow\vb$, иначе $\va\uparrow\downarrow\vb$
\end{enumerate}
\end{opred}
\begin{theor}
Операция умножения вектора на число обладает следующими свойствами:
\begin{enumerate}
\item $\forall\ \va\colon1\va=\va$
\item $\alpha(\beta\va)=(\alpha\beta)\va$
\item $(\alpha+\beta)\va=\alpha\va+\beta\va$
\item $\alpha(\va+\vb)+\alpha\va+\alpha\vb$
\end{enumerate}
\end{theor}
\begin{proof}
\begin{enumerate}
\item По определению
\item По определению
\item По геометрическим соображениям
\item По геометрическим соображениям
\end{enumerate}
\end{proof}
\section{*Векторы как элементы вещественного линейного пространства*}
Пусть $\V\neq\varnothing$, и на $\V$ задано соответствие:

$\forall\ (a,b)\in\underbrace{\V\times\V}_{\scriptsize\mbox{Декартово произведение}}\longmapsto{}c\in\V$ называется алгебраической операцией или внутренним законом композиции.

$c=a*b$

$c=a\cdot{}b$

$c=a+b$ --- абстрактное сложение.

$\forall\ \alpha\in\mathbb{R},\ a\in\V\longmapsto{}b\in\V$ --- внешний закон композиции

$b=\alpha{}a$ --- абстрактное умножение.

\begin{opred}
Множество с введенными на нём внутренним и внешним законом композиции называется \mybold{вещественным линейным пространством $\mathbb{V}$}, если эти операции обладают следующими свойствами:

$\forall\ a,b,c\in\V, \alpha,\beta\in\mathbb{R}$
\begin{enumerate}
\item $a+b=b+a$
\item $(a+b)+c=a+(b+c)$
\item $\exists\ \T\in\V:\ a+\T=a$
\item $\exists\ (-a)\in\V:\ a+(-a)=\T$
\item $1a=a$
\item $\alpha(\beta{}a)=(\alpha\beta)a$
\item $(\alpha+\beta)a=\alpha{}a+\beta{}a$
\item $\alpha(a+b)=\alpha{}a+\alpha{}b$
\end{enumerate}
\end{opred}
\begin{opred}
$\V$ --- \mybold{векторное пространство}, если \begin{itemize}
\item элементы $\V$ --- векторы
\item $\V$ --- вещественное линейное пространство
\item $\T$ --- нулевой вектор
\item $(-a)$ --- противоположный вектор
\end{itemize}
\end{opred}
\begin{theor}
Простейшие свойства:\begin{enumerate}
\item $\T$ определен однозначно, $\forall\ a\in\V\ (-a)$ определен однозначно
\item $\forall\ a\in\V\colon 0a=\T$, $\forall\ \alpha\in\mathbb{R}\colon \alpha\T=\T$
\item из равенства $\alpha{}a=\T$ следует : $\left[\begin{aligned}
&\alpha=0 \\
&a=\T
\end{aligned}\right.$
\item $\forall\ a\in\V\colon -a=(-1)a$
\end{enumerate}
\end{theor}
\begin{proof}
\begin{enumerate}
\item Пусть $\exists\ \T_1,\T_2\in\V$

$\forall\ a\in\V\colon{}a+\T_1=a,\ a+\T_2=a$

$\T_1=\T_1+\T_2=\T_2+\T_1=\T_2$

Пусть $\exists\ a\in\V\colon\exists\ (-a)_1,(-a)_2\in\V$

$\T=a+(-a)_1=a+(-a)_2$

$(-a)_1+(a+(-a)_1)=(-a)_1+(a+(-a)_2)$

$\underbrace{((-a)_1+a)}_{\scriptsize\T}+(-a)_1=\underbrace{((-a)_1+a)}_{\scriptsize\T}+(-a)_2$

\item $a+0a=1a+0a=(1+0)a=1a=a$, т. о., $a+\underbrace{0a}_{\scriptsize\T}=a$

$\alpha\T=\alpha(0a)=(\alpha0)a=0a=\T$

\item $\alpha{}a=\T\Rightarrow\left[\begin{aligned}
& \alpha=0 \\
& \alpha\neq0\ \Rightarrow\ \frac1\alpha(\alpha{}a)=\frac1\alpha\T\Rightarrow a=\T
\end{aligned}\right.$
\item $(-1)a+a=(-1)a+1a=(-1+1)a=0a=\T$; по единственности $(-a)\ (-1)a=-a$
\end{enumerate}
\end{proof}
\chapter{*Линейная зависимость*}
\begin{opred}
Рассмотрим линейную комбинацию элементов:

$\alpha_1a_1+\alpha_2a_2+\cdots+\alpha_ka_k$

Если $\alpha_1=\alpha_2=\ldots=\alpha_k=0$, то эта линейная комбинация называется \mybold{тривиальной}. Если $\exists\ a_j\neq0$, то эта линейная комбинация --- \mybold{нетривиальная}.
\end{opred}
\begin{opred}
Система векторов $a_1,a_2,\ldots,a_k\in\V$ называется \mybold{линейно зависимой}, если существует нетривиальная линейная комбинация этих векторов, равная $\T$.
\end{opred}
\begin{opred}
Система векторов $a_1,a_2,\ldots,a_k\in\V$ называется \mybold{линейно независимой}, если только тривиальная линейная комбинация этих векторов равна $\T$.
\end{opred}
\begin{remark}
\begin{enumerate}
\item $k=1$

Система л/з $\Leftrightarrow$ $a_1=\T$
\item Если в системе векторов есть $\T$, то эта система л/з.
\end{enumerate}
\end{remark}
\section{*Теоремы о линейной зависимости*}
\begin{theor}Система из более, чем одного вектора линейно зависима тогда и только тогда, когда хотя бы один вектор линейно выражается через остальные (является их линейной комбинацией).
\end{theor}
\begin{proof}
$\Longleftarrow$ $a_k=\beta_1a_1+\cdots+\beta_{k-1}a_{k-1}$

$\underbrace{\beta_1a_1+\cdots+\beta_{k-1}a_{k-1}+(-1)a_k}_{\scriptsize\mbox{нетривиальная ЛК}}=\T$

$\Longrightarrow$ $\exists\ \alpha_1,\ldots,\alpha_k\in\mathbb{R},\ a_j\neq0$

$\alpha_1a_1+\ldots+\alpha_ja_j+\ldots+\alpha_ka_k=\T$

$a_j=\sum\limits_{i\neq{}j}(-\frac{\alpha_i}{\alpha_j})a_i$
\end{proof}
\begin{theor}
Если в системе $a_1,\ldots,a_k$ есть л/з подсистема, то и вся система л/з.
\end{theor}
\begin{proof}
Пусть $a_1,\ldots,a_s\ (s<k)$ --- л/з \then $\exists$ нетривиальная л/к $\alpha_1a_1+\cdots+\alpha_sa_s+0a_{s+1}+\cdots+0a_k=\T$
\mybold{Следствие:} если система $a_1,\ldots,a_k$ л/нез, то любая её подсистема л/нез.
\end{proof}
\begin{theor}
Система $a_1,\ldots,a_k$ л/нез тогда и только тогда, когда любой вектор, являющийся их линейной комбинацией, выражается через них единственным образом.
\end{theor}
\begin{proof}
$\Longrightarrow$ Предположим противное: $\exists\ b\in\V\colon b=\alpha_1a_1+\cdots+\alpha_ka_k;\ b=\beta_1a_1+\cdots+\beta_ka_k\ (\exists\ \alpha_j\neq\beta_j)$ \then $\T=(\alpha_1-\beta_1)a_1+\cdots+(\alpha_k-\beta_k),\ \alpha_j-\beta_j\neq0$ \then система л/з \then противоречие

$\Longleftarrow$ Нулевой вектор заведомо является линейной комбинацией этих векторов: $\T=0\alpha_1+\cdots+0\alpha_k$; это тривиальная комбинация \then система л/нез
\end{proof}
\section{*Геометрический смысл линейной зависимости*}
\begin{theor}
2 геометрических вектора л/з тогда и только тогда, когда они коллинеарны
\end{theor}
\begin{proof}
$\Longrightarrow$ $\bar{a}_2=\beta\bar{a}_1$ \then $\bar{a}_1,\bar{a}_2$ --- коллинеарны

$\Longleftarrow$ а) $\left[\begin{aligned}
&\bar{a}_1=\bar{0} \Rightarrow \bar{a}_1=0\bar{a}_2 \\
&\bar{a}_2=\bar{0} \Rightarrow \bar{a}_2=0\bar{a}_1
\end{aligned}\right.$

б) $\bar{a}_1,\bar{a}_2\neq\bar{0}$

$\bar{a}_1\uparrow\uparrow\bar{a}_2\ \Rightarrow\ \cfrac{\bar{a}_1}{|\bar{a}_1|}=\cfrac{\bar{a}_2}{|\bar{a}_2|} \Rightarrow\ \bar{a}_1=\cfrac{|\bar{a}_1|}{|\bar{a}_2|}\bar{a}_2$

$\bar{a}_1\uparrow\downarrow\bar{a}_2\ \Rightarrow\ \cfrac{\bar{a}_1}{|\bar{a}_1|}=-\cfrac{\bar{a}_2}{|\bar{a}_2|} \Rightarrow\ \bar{a}_1=-\cfrac{|\bar{a}_1|}{|\bar{a}_2|}\bar{a}_2$
\end{proof}
\begin{theor}
3 геометрических вектора л/з тогда и только тогда, когда они компланарны
\end{theor}
\begin{proof}
$\Longrightarrow$ $\bar{a}_1=\underbrace{\beta_2\bar{a}_2+\beta_3\bar{a}_3}_{\scriptsize{\mbox{параллелен плоскости, определяемой }\bar{a}_2\mbox{ и }\bar{a}_3}}$

$\Longleftarrow$ а) один вектор $\bar{0}=\bar{a}_1$ \then $\bar{a}_1=0\bar{a}_2+0\bar{a}_3$

б) все вектора $\neq\bar{0}$

$\bar{a}_3=\vec{AD}+\vec{AB}=\alpha_1\bar{a}_1+\alpha_2\bar{a}_2$ \then $\bar{a}_1,\bar{a}_2,\bar{a}_3$ --- л/з
\end{proof}
\begin{theor}
4 геометрических вектора всегда л/з
\end{theor}
\begin{proof}
а) ${a}_1,\bar{a}_2,\bar{a}_3$ --- компланарны \then они л/з \then вся система л/з

б) ${a}_1,\bar{a}_2,\bar{a}_3$ --- некомпланарны.

$\bar{a}_4=\vec{AB}+\vec{AC}+\vec{AD}=\alpha_1\bar{a}_1+\alpha_2\bar{a}_2+\alpha_3\bar{a}_3$ \then ${a}_1,\bar{a}_2,\bar{a}_3,\bar{a}_4$ --- л/з
\end{proof}
\chapter{*Ранг матрицы*}
\section{*Арифметическое линейное пространство*}
\begin{opred}
Элементами \mybold{арифметического линейного пространства} являются упорядоченные совокупности из $n$ вещественных чисел. Обозначается это так: $\mathbb{R}^n$ Нетрудно убедиться, что арифметические линейные пространства --- вещественные линейные пространства $(\mathbb{R}^{1\times{}n},\mathbb{R}^{n\times1})$
\end{opred}
\section{*Понятие ранга*}
\begin{opred}
Рангом ненулевой матрицы $A$ называется максимальный размер её ненулевых миноров. Нулевая матрица считается матрицей ранга 0. Обозначается так: $rang\ A=rk\ A=rg\ A$
\end{opred}
\begin{opred}
В любой ненулевой матрице ранга $r$ \mybold{базисный минор} --- любой ненулевой минор порядка $r$.
\end{opred}
\begin{opred}
\mybold{Базисные строки(столбцы)} --- строки(столбцы) матрицы, в которых расположен базисный минор.
\end{opred}
\section{*Теорема о базисном миноре*}
\begin{theor}(строчный вариант)

В ненулевой матрице \rmatrix{A}{m}{n}:\begin{enumerate}
\item базисные строки л/нез
\item любая строка матрицы $A$ линейно выражается через базисные строки
\end{enumerate}
\end{theor}
\begin{remark}
Любая строка матрицы $A\ (a_{i1},a_{i2},\ldots,a_{in})$ --- элемент арифметического линейного пространства~$\mathbb{R}^n$
\end{remark}
\begin{proof}
\rmatrixof{A}{a_{ij}}{m}{n}

Пусть базисный минор --- $M_r$.
\begin{enumerate}
\item Предположим противное, т.е. $a'_1,a'_2,\ldots,a'_k$ --- л/з \then $\exists\ a'_k=\sum\limits_{i\neq k}\alpha_i a'_i$ \then $k$--я строка минора линейно выражается через другие строки базового минора \then $M_r=0$
\item Нужно показать: $a'_k=\sum\limits^n_{i=1}\beta_ia'_i$
\begin{enumerate}
\item $1\leq k\leq r$

$a'_k=\underbrace{\cdots}_{\scriptsize\mbox{коэф}=0}+1a'_k+\underbrace{\cdots}_{\scriptsize\mbox{коэф}=0}$
\item $k>r$

\mybold{Окаймим минор}:
$$
\det\begin{bmatrix}
a_{11} & \cdots & a_{1r} & a_{1j} \\
\vdots & \ddots & \vdots & \vdots \\
a_{r1} & \cdots & a_{rr} & a_{rj} \\
a_{k1} & \cdots & a_{kr} & a_{kj}
\end{bmatrix}=0\mbox{ для } \forall\ j
$$
Почему: если $1\leq j\leq r$, то в матрице 2 одинаковых столбца; если $j>r$ и детерминант не равен нулю, то ранг равен $r+1$, что противоречит условию.

$0=\underbrace{a_{1j}A_1+a_{2j}A_2+\cdots+a_{rj}Ar+a_{kj}M_r}_{\scriptsize\begin{aligned}&\mbox{определяется только первыми r} \\ & \mbox{столбцами вспомогательной матрицы,} \\ & \mbox{а потому не зависит от j}\end{aligned}}$

$a_{kj}=-\cfrac{A_1}{M_r}a_{1j}-\cfrac{A_3}{M_r}a_{2j}-\cdots-\cfrac{A_r}{M_r}a_{rj}$

$a'_k=-\cfrac{A_1}{M_r}a'_1-\cfrac{A_3}{M_r}a'_2-\cdots-\cfrac{A_r}{M_r}a'_r$
\end{enumerate}
\end{enumerate}
\end{proof}
\section{*Следствия из теоремы о базисном миноре*}
\begin{opred}
Матрица называется \mybold{вырожденной}, если её $\det$ равен 0, и \mybold{невырожденной} --- в противном случае.
\end{opred}
\begin{theor}
\mybold{Критерий невырожденности.}

Определитель матрицы равен нулю тогда и только тогда, когда её строки (столбцы) л/нез.
\end{theor}
\begin{proof}$\Longrightarrow$ Пусть $|A|\neq0$ и её строки л/з \then существует строка, линейно выражающаяся через другие \then $|A|=0$

$\Longleftarrow$ Пусть $|A|=0$ \then $rg A\leq n-1$ \then базисными строками будут не все строки $A$ \then существует строка, являющаяся линейной комбинацией остальных \then строки $A$ л/з.
\end{proof}
\begin{theor}
\mybold{Основная теорема о линейной зависимости.}

Пусть $\V$ --- ВЛП; рассмотрим 2 системы векторов: $\{a_1,\ldots,a_k\},\{b_1,\ldots,b_m\},\ m>k$. Тогда если любой $b_j$ линейно выражается через векторы первой системы, то вторая система --- л/з.

\mybold{Другая формулировка}: Если большая система линейно выражается через меньшую, то большая --- л/з.
\end{theor}
\begin{proof} Пусть

$\begin{matrix}
& b_1&=&\alpha_1a_1+\cdots+\alpha_ka_k & \\
& b_2&=&\beta_1a_1+\cdots+\beta_ka_k & \\
& &\vdots & &\\
& b_{m-1}&=&\gamma_1a_1+\cdots+\gamma_ka_k & \\
& b_m&=&\delta_1a_1+\cdots+\delta_ka_k &
\end{matrix}$

$m\left\{\lefteqn{\phantom{\begin{matrix}\alpha_1 \\ \beta_1 \\ \vdots \\ \gamma_1 \\ \delta_1 \end{matrix}}}\right.
\underbrace{
\begin{matrix}
\alpha_1 & \ldots & \alpha_k \\
\beta_1  & \ldots & \beta_k  \\
\vdots   & \vdots & \vdots   \\
\gamma_1 & \ldots & \gamma_k \\
\delta_1 & \ldots & \delta_k
\end{matrix}}_{k}$

$m>k$ \then базисных строк не больше $k$ \then существует строка (пускай последняя), которая линейно выражается через остальные.

$(\gamma_1,\ldots,\gamma_k)=\lambda_1(\alpha_1,\ldots,\alpha_k)+\lambda_2(\beta_1,\ldots,\beta_k)+\cdots+\lambda_{m-1}(\delta_1,\ldots,\delta_k)$

$\lambda_1b_1+\lambda_2b_2+\cdots+\lambda_{m-1}b_{m-1}=\delta_1a_1+\delta_2a_2+\cdots+\delta_ka_k=b_m$ \then л/з.
\end{proof}
\begin{theor}
\mybold{Другое определение ранга}

Ранг ненулевой матрицы равен максимальному числу её л/нез строк и/или столбцов.
\end{theor}
\begin{proof}
Пусть $rg A=r\geq1$. Докажем, что $r$ равно максимальному количеству л/нез строк, т.е.: \begin{enumerate}
\item $\exists\ r$ л/нез строк
\item Любой набор из большего количества строк л/з.
\end{enumerate}
\begin{enumerate}
\item $r$ базисных строк л/нез по теореме о базисном миноре
\item Возьмём $k>r$ строк. Все они линейно выражаются через $r$ базисных \then в силу основной теоремы о линейной зависимости выбранные $k$ строк л/з.
\end{enumerate}
\end{proof}
\mybold{Следствие}. $rg\ A^T=rg\ A$
\begin{theor}
Пусть $A,B$ --- две матрицы с одинаковым количеством столбцов (строк). Пусть любая строка $B$ линейно выражается через строки $A$. Тогда $rg\ B\leq rg\ A$
\end{theor}
\begin{proof}
Пусть $rg\ A=r$ и $rg B>r$.

Рассмотрим в $B$ базисные строки (их больше $r$). Они все линейно выражаются через строки $A$, а строки $A$ линейно выражаются через $r$ базисных строк $A$ \then базисные строки $B$ линейно выражаются через $r$ базисных строк $A$ \then базисные строки $B$ л/з, что противоречит теореме о базисном миноре.
\end{proof}
\begin{theor}
$rg\ AB\leq min(rg\ A,rg\ B)$
\end{theor}
\begin{proof}
$\left\{\begin{aligned}
& \text{Столбцы $AB$ являются л/к столбцов $A$} \\
& \text{Строки $AB$ являются л/к строк $B$}
\end{aligned}\right.$ \then $\left\{\begin{aligned}
& rg\ AB\leq rg\ A \\
& rg\ AB\leq rg\ B
\end{aligned}\right.$ \then $rg\ AB\leq min(rg\ A,rg\ B)$
\end{proof}
\begin{theor}
Если $B$ --- квадратная невырожденная матрица, то $\left\{\begin{aligned}
& rg\ AB=rg\ A \\
& rg\ BA=rg\ A
\end{aligned}\right.\ \forall\ A$
\end{theor}
\begin{proof}
$rg\ AB\leq rg\ A$

$A=(AB)B^{-1}$

$rg\ [(AB)B^{-1}]\leq rg\ AB$

$rg\ AB=rg\ A$
\end{proof}
\section{*Метод Гаусса вычисления ранга*}
\begin{theor}
Элементарные преобразования не меняют ранг матрицы.
\end{theor}
\begin{proof}
$A\epofs\tilde{A}=SA$

$A\epofc\tilde{\tilde{A}}=AS$

Заметим, что $S$ --- невырожденная.

I тип: $|S|=-1$

II тип: $|S|=\lambda$

III тип: $|S|=1$

$S$ --- невырожденная, квадратная \then ЭП не меняют ранг.
\end{proof}
\mybold{Метод Гаусса вычисления ранга.}

$A\epofs B$ --- верхняя ступенчатая матрица, $rg\ A=rg\ B$

Ранг верхней ступенчатой матрицы равен числу её ненулевых строк.

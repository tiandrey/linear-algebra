\chapter{*-Направленный отрезок и свободный вектор-*}
\chapter{*-Линейные операции над векторами-*}
\section{*-Сложение-*}
\section{*-Умножение на число-*}
\section{*Векторы как элементы вещественного линейного пространства*}
Пусть $\V\neq\varnothing$, и на $\V$ задано соответствие:

$\forall\ (a,b)\in\underbrace{\V\times\V}_{\scriptsize\mbox{Декартово произведение}}\longmapsto{}c\in\V$ называется алгебраической операцией или внутренним законом композиции.

$c=a*b$

$c=a\cdot{}b$

$c=a+b$ --- абстрактное сложение.

$\forall\ \alpha\in\mathbb{R},\ a\in\V\longmapsto{}b\in\V$ --- внешний закон композиции

$b=\alpha{}a$ --- абстрактное умножение.

\begin{opred}
Множество с введенными на нём внутренним и внешним законом композиции называется \mybold{вещественным линейным пространством $\mathbb{V}$}, если эти операции обладают следующими свойствами:

$\forall\ a,b,c\in\V, \alpha,\beta\in\mathbb{R}$
\begin{enumerate}
\item $a+b=b+a$
\item $(a+b)+c=a+(b+c)$
\item $\exists\ \T\in\V:\ a+\T=a$
\item $\exists\ (-a)\in\V:\ a+(-a)=\T$
\item $1a=a$
\item $\alpha(\beta{}a)=(\alpha\beta)a$
\item $(\alpha+\beta)a=\alpha{}a+\beta{}a$
\item $\alpha(a+b)=\alpha{}a+\alpha{}b$
\end{enumerate}
\end{opred}
\begin{opred}
$\V$ --- \mybold{векторное пространство}, если \begin{itemize}
\item элементы $\V$ --- векторы
\item $\V$ --- вещественное линейное пространство
\item $\T$ --- нулевой вектор
\item $(-a)$ --- противоположный вектор
\end{itemize}
\end{opred}
\begin{theor}
Простейшие свойства:\begin{enumerate}
\item $\T$ определен однозначно, $\forall\ a\in\V\ (-a)$ определен однозначно
\item $\forall\ a\in\V\colon 0a=\T$, $\forall\ \alpha\in\mathbb{R}\colon \alpha\T=\T$
\item из равенства $\alpha{}a=\T$ следует : $\left[\begin{aligned}
&\alpha=0 \\
&a=\T
\end{aligned}\right.$
\item $\forall\ a\in\V\colon -a=(-1)a$
\end{enumerate}
\end{theor}
\begin{proof}
\begin{enumerate}
\item Пусть $\exists\ \T_1,\T_2\in\V$

$\forall\ a\in\V\colon{}a+\T_1=a,\ a+\T_2=a$

$\T_1=\T_1+\T_2=\T_2+\T_1=\T_2$

Пусть $\exists\ a\in\V\colon\exists\ (-a)_1,(-a)_2\in\V$

$\T=a+(-a)_1=a+(-a)_2$

$(-a)_1+(a+(-a)_1)=(-a)_1+(a+(-a)_2)$

$\underbrace{((-a)_1+a)}_{\scriptsize\T}+(-a)_1=\underbrace{((-a)_1+a)}_{\scriptsize\T}+(-a)_2$

\item $a+0a=1a+0a=(1+0)a=1a=a$, т. о., $a+\underbrace{0a}_{\scriptsize\T}=a$

$\alpha\T=\alpha(0a)=(\alpha0)a=0a=\T$

\item $\alpha{}a=\T\Rightarrow\left[\begin{aligned}
& \alpha=0 \\
& \alpha\neq0\ \Rightarrow\ \frac1\alpha(\alpha{}a)=\frac1\alpha\T\Rightarrow a=\T
\end{aligned}\right.$
\item $(-1)a+a=(-1)a+1a=(-1+1)a=0a=\T$; по единственности $(-a)\ (-1)a=-a$
\end{enumerate}
\end{proof}
\chapter{*Линейная зависимость*}
\begin{opred}
Рассмотрим линейную комбинацию элементов:

$\alpha_1a_1+\alpha_2a_2+\cdots+\alpha_ka_k$

Если $\alpha_1=\alpha_2=\ldots=\alpha_k=0$, то эта линейная комбинация называется \mybold{тривиальной}. Если $\exists\ a_j\neq0$, то эта линейная комбинация --- \mybold{нетривиальная}.
\end{opred}
\begin{opred}
Система векторов $a_1,a_2,\ldots,a_k\in\V$ называется \mybold{линейно зависимой}, если существует нетривиальная линейная комбинация этих векторов, равная $\T$.
\end{opred}
\begin{opred}
Система векторов $a_1,a_2,\ldots,a_k\in\V$ называется \mybold{линейно независимой}, если только тривиальная линейная комбинация этих векторов равна $\T$.
\end{opred}
\begin{remark}
\begin{enumerate}
\item $k=1$

Система л/з $\Leftrightarrow$ $a_1=\T$
\item Если в системе векторов есть $\T$, то эта система л/з.
\end{enumerate}
\end{remark}
\section{*Теоремы о линейной зависимости*}
\begin{theor}Система из более, чем одного вектора линейно зависима тогда и только тогда, когда хотя бы один вектор линейно выражается через остальные (является их линейной комбинацией).
\end{theor}
\begin{proof}
$\Longleftarrow$ $a_k=\beta_1a_1+\cdots+\beta_{k-1}a_{k-1}$

$\underbrace{\beta_1a_1+\cdots+\beta_{k-1}a_{k-1}+(-1)a_k}_{\scriptsize\mbox{нетривиальная ЛК}}=\T$

$\Longrightarrow$ $\exists\ \alpha_1,\ldots,\alpha_k\in\mathbb{R},\ a_j\neq0$

$\alpha_1a_1+\ldots+\alpha_ja_j+\ldots+\alpha_ka_k=\T$

$a_j=\sum\limits_{i\neq{}j}(-\frac{\alpha_i}{\alpha_j})a_i$
\end{proof}
\begin{theor}
Если в системе $a_1,\ldots,a_k$ есть л/з подсистема, то и вся система л/з.
\end{theor}
\begin{proof}
Пусть $a_1,\ldots,a_s\ (s<k)$ --- л/з \then $\exists$ нетривиальная л/к $\alpha_1a_1+\cdots+\alpha_sa_s+0a_{s+1}+\cdots+0a_k=\T$
\mybold{Следствие:} если система $a_1,\ldots,a_k$ л/нез, то любая её подсистема л/нез.
\end{proof}
\begin{theor}
Система $a_1,\ldots,a_k$ л/нез тогда и только тогда, когда любой вектор, являющийся их линейной комбинацией, выражается через них единственным образом.
\end{theor}
\begin{proof}
$\Longrightarrow$ Предположим противное: $\exists\ b\in\V\colon b=\alpha_1a_1+\cdots+\alpha_ka_k;\ b=\beta_1a_1+\cdots+\beta_ka_k\ (\exists\ \alpha_j\neq\beta_j)$ \then $\T=(\alpha_1-\beta_1)a_1+\cdots+(\alpha_k-\beta_k),\ \alpha_j-\beta_j\neq0$ \then система л/з \then противоречие

$\Longleftarrow$ Нулевой вектор заведомо является линейной комбинацией этих векторов: $\T=0\alpha_1+\cdots+0\alpha_k$; это тривиальная комбинация \then система л/нез
\end{proof}
\section{*Геометрический смысл линейной зависимости*}
\begin{theor}
2 геометрических вектора л/з тогда и только тогда, когда они коллинеарны
\end{theor}
\begin{proof}
$\Longrightarrow$ $\bar{a}_2=\beta\bar{a}_1$ \then $\bar{a}_1,\bar{a}_2$ --- коллинеарны

$\Longleftarrow$ а) $\left[\begin{aligned}
&\bar{a}_1=\bar{0} \Rightarrow \bar{a}_1=0\bar{a}_2 \\
&\bar{a}_2=\bar{0} \Rightarrow \bar{a}_2=0\bar{a}_1
\end{aligned}\right.$

б) $\bar{a}_1,\bar{a}_2\neq\bar{0}$

$\bar{a}_1\uparrow\uparrow\bar{a}_2\ \Rightarrow\ \cfrac{\bar{a}_1}{|\bar{a}_1|}=\cfrac{\bar{a}_2}{|\bar{a}_2|} \Rightarrow\ \bar{a}_1=\cfrac{|\bar{a}_1|}{|\bar{a}_2|}\bar{a}_2$

$\bar{a}_1\uparrow\downarrow\bar{a}_2\ \Rightarrow\ \cfrac{\bar{a}_1}{|\bar{a}_1|}=-\cfrac{\bar{a}_2}{|\bar{a}_2|} \Rightarrow\ \bar{a}_1=-\cfrac{|\bar{a}_1|}{|\bar{a}_2|}\bar{a}_2$
\end{proof}
\begin{theor}
3 геометрических вектора л/з тогда и только тогда, когда они компланарны
\end{theor}
\begin{proof}
$\Longrightarrow$ $\bar{a}_1=\underbrace{\beta_2\bar{a}_2+\beta_3\bar{a}_3}_{\scriptsize{\mbox{параллелен плоскости, определяемой }\bar{a}_2\mbox{ и }\bar{a}_3}}$

$\Longleftarrow$ а) один вектор $\bar{0}=\bar{a}_1$ \then $\bar{a}_1=0\bar{a}_2+0\bar{a}_3$

б) все вектора $\neq\bar{0}$

$\bar{a}_3=\vec{AD}+\vec{AB}=\alpha_1\bar{a}_1+\alpha_2\bar{a}_2$ \then $\bar{a}_1,\bar{a}_2,\bar{a}_3$ --- л/з
\end{proof}
\begin{theor}
4 геометрических вектора всегда л/з
\end{theor}
\begin{proof}
а) ${a}_1,\bar{a}_2,\bar{a}_3$ --- компланарны \then они л/з \then вся система л/з

б) ${a}_1,\bar{a}_2,\bar{a}_3$ --- некомпланарны.

$\bar{a}_4=\vec{AB}+\vec{AC}+\vec{AD}=\alpha_1\bar{a}_1+\alpha_2\bar{a}_2+\alpha_3\bar{a}_3$ \then ${a}_1,\bar{a}_2,\bar{a}_3,\bar{a}_4$ --- л/з
\end{proof}
\chapter{*-Ранг матрицы-*}
\section{*-Арифметическое линейное пространство-*}
\section{*-Понятие ранга-*}
\section{*-Теорема о базисном миноре-*}
\section{*-Следствия из теоремы о базисном миноре-*}
\section{*-Метод Гаусса вычисления ранга-*}

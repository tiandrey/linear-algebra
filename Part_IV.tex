\chapter{Координаты вектора-}
На  прямой: любой ненулевой вектор $\forall\bar{e}_1\neq\T$ \then $\bar{x}=\alpha_1\bar{e}_1$

$\bar{x}=\cfrac{|\bar{x}|}{|\bar{e}_1|}\bar{e}_1$

Величина вектора $\bar{x}$ на оси, определяемой $\bar{e}_1$:
$\left\{\begin{aligned}
 |\bar{x}| & ,\ \bar{x}\SN\bar{e}_1 \\
-|\bar{x}| & ,\ \bar{x}\PN\bar{e}_1 \\
         0 & ,\ \bar{x}=\T
\end{aligned}\right\}\equiv(\bar{x})$

На плоскости: выбраны $\bar{e}_1,\bar{e}_2$ \then $\bar{x}=\alpha_1\bar{e}_1+\alpha_2\bar{e}_2$

В пространстве: выбраны $\bar{e}_1,\bar{e}_2,\bar{e}_3$ \then $\bar{x}=\alpha_1\bar{e}_1+\alpha_2\bar{e}_2+\alpha_3\bar{e}_3$

Покажем, что во всех этих равенствах коэффициенты определяются однозначно.
\begin{theor}
Если $e_1,e_2,\ldots,e_n\in\V$ линейно независимы, то для $\forall$ вектора $x\in\V$ представление в виде их линейной комбинации определяется однозначно.
\end{theor}
\begin{proof}
\begin{enumerate}
\item $\bar{x}\mapsto\alpha_1$
\item $\bar{x}\mapsto(\alpha_1;\alpha_2)$
\item $\bar{x}\mapsto(\alpha_1;\alpha_2;\alpha_3)$
\end{enumerate}	
\end{proof}
\chapter{-Координаты точки-}	
\chapter{-Скалярное произведение-}
\chapter{-Векторное и смешанное произведения-}
\chapter{Преобразование афинной системы координат}
Пусть известны координаты точки $M(x,y,z)$ в старой СК. Найти её координаты $M(x',y',z')$.

Зададим параметры --- положение новой СК относительно старой, а именно $O'(\alpha,\beta,\gamma)$ -- в старой СК.

Координаты $\bar{e}_i': \begin{pmatrix}
c_{11} & c_{12} & c_{13} \\
c_{21} & c_{22} & c_{23} \\
c_{31} & c_{32} & c_{33}
\end{pmatrix} \equiv C$

\begin{opred}
$C$ --- \mybold{матрица перехода}.
\end{opred}

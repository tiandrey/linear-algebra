\chapter{Определенный интеграл}
\section{Основные понятия}
\begin{opred}
\mybold{Разбиением отрезка $[a;b]$} называется набор $\underset{0\leq k\leq n}{\{x_k\}}=\{x_0,x_1,\ldots,x_n\}$, где $x_0=a$, $x_n=b$, $x_0<x_1<\ldots<x_n$.
\end{opred}
\begin{opred}
\mybold{Диаметром,} или \mybold{мелкостью разбиения $\{x_k\}$} называется число $d=d(\{x_k\})=\underset{1\leq k\leq n}{max}\{\Delta x_k\}$, $\Delta x_k=x_k-x_{k-1}$.
\end{opred}
\begin{opred}
\mybold{Размеченным разбиением} называется разбиение, в котором зафиксированы точки $\underset{1\leq k\leq n}{\{\xi_k\}}$, где $\xi_k\in[x_{k-1},x_k]$.
\end{opred}
\begin{opred}
Разбиение $\{y_m\}$ называется \mybold{измельчением разбиения $\{x_k\}$}, если $\{x_k\}\subset\{y_m\}$.
\end{opred}
\begin{opred}
Разбиение $\{z_j\}$ называется \mybold{объединением разбиений $\{x_k\}$ и $\{y_m\}$}, если $\{z_j\}=\{x_k\}\cup\{y_m\}$.
\end{opred}
\begin{opred}
Пусть на $[a;b]$ задана $f(x)$. \mybold{Интегральной суммой для $f(x)$ на отрезке $[a;b]$, составленной по размеченному разбиению $(\{x_k\},\{\xi_k\})$} называется выражение вида $$\sigma_f=\sigma_f(\{x_k\},\{\xi_k\}):=\sum\limits^n_{k=1}f(\xi_k)\Delta x_k$$
\end{opred}
\begin{opred}
Число $A$ наывается \mybold{пределом интегральных сумм $f(x)$ на $[a;b]$ при $d\mapsto0$}, где $d$ - мелкость разбиений, если $\forall\ \epsilon>0\ \exists\ \delta=\delta(\epsilon)>0$, что для любого размеченного разбиения $(\{x_k\},\{\xi_k\})$, мелкость которого $d<\delta$, выполнено неравенство: $|\sigma_f\rr-A|<\epsilon$.
$$A=\lim\limits_{d\mapsto0}\sigma_f\rr$$
\end{opred}
\begin{opred}
\mybold{Определенным интегралом} $f(x)$ на отрезке $[a;b]$ называется предел интегральных сумм этой функции на этом отрезке при $d\mapsto0$.
$$\int\limits^b_af(x)dx:=\lim\limits_{d\mapsto0}\sum\limits^n_{k=1}f(\xi_k)\Delta x_k$$
\end{opred}
\begin{theor}
Если у $f(x)$ существует предел интегральных сумм, то этот предел -- единственный.
\end{theor}
\begin{proof}
Предположим противное. Пусть существует 2 предела: $A_1<A_2,\ A_2-A_1=\alpha>0$.
По определению предела, для $\forall\ \epsilon=\frac\alpha3\ \exists\ \delta_1,\delta_2$, что:
\begin{enumerate}
\item для $\forall\ \rr,\ d<\delta_1\colon|\sigma_f\rr-A_1|<\epsilon$
\item для $\forall\ \rr,\ d<\delta_2\colon|\sigma_f\rr-A_2|<\epsilon$
\end{enumerate}
Тогда для любого размеченного разбиения $\rr$, у которого $d\leq min(\delta_1,\delta_2)$, будет выполнено и 1), и 2) \then $\sigma_f\rr$ попадает одновременно в 2 непересекающихся интервала. Противоречие.
\end{proof}
\begin{theor}
Если существует $\int\limits_a^bf(x)dx$, то обязательно $f(x)$ ограничена на $[a;b]$.
\end{theor}
\begin{proof}
Предположим противное. Пусть $f(x)$ неограничена на $[a;b]$. Тогда для любого разбиения $\{x_k\}\ f(x)$ будет неограничена на хотя бы одном отрезке $\xko$ этого разбиения. Выберем последовательность разбиений $\underset{0\leq k\leq n}{\{x_k^m\}}$ с мелкостью $d_m=d_m(\{x_k^m\})<\frac1m$. В каждом из этих разбиений выберем отрезок $\xmko$, где $f(x)$ не ограничена. Теперь подберем разметку так, чтобы интегральные суммы $\sigma_f\rr$ были больше $m$. Выберем $\xi_k$ на всех отрезках $\xmk$, кроме $\xmko$, произвольным образом. А на $\xko$ выберем $\xi_{k_0}$ так, чтобы $|f(\xi_{k_0})|>\frac{m+|\sum\limits_{k\neq k_0}f(\xi_k)\Delta x_k)|}{\Delta x^m_{k_0}}$.
Тогда вспомним неравенство: $|a+b|\geq|a|-|b|$ ($|a|=|(a+b)-b|\leq|a+b|+|b|$ \then $|a|-|b|\leq|a+b|$, ЧТД).

$|\sigma_f\rr|=|f(\xi_k^m)\Delta x_{k_0}^m+\sum\limits_{k\neq k_0}f(\xi^m_k)\Delta x^m_k|\geq|f(\xi_k^m)\Delta x_{k_0}^m|-|\sum\limits_{k\neq k_0}f(\xi^m_k)\Delta x^m_k|>m$ \then $\underset{d<\frac1m}{\sigma_f}>m$ \then при $m\mapsto\infty$ получим противоречие.
\end{proof}
\begin{opred}
Пусть $f(x)$ ограничена на $[a;b]$, и задано размеченное разбиение $\rr$ \then $f(x)$ ограничена на каждом отрезке $\xk$ \then  существует $\underset{x\in\xk}{sup}\{f(x)\}=M_k$, $\underset{x\in\xk}{inf}\{f(x)\}=m_k$. \mybold{Верхней (нижней) интегральной суммой (суммой Дарбу)} $f(x)$ по разбиению $\{x_k\}$ на $[a;b]$  называется выражение:

$\vs:=\sum\limits_{k=1}^nM_k\Delta x_k$

$\ns:=\sum\limits_{k=1}^nm_k\Delta x_k$
\end{opred}
\begin{theor}
6 свойств сумм Дарбу.
\begin{enumerate}
\item $\forall\ \rr \ns\leq\sigma_f\rr\leq\vs$
\item $\forall\ \epsilon>0\ \exists$ разметка $\{\xi_k\}$ данного разбиения $\{x_k\}$, что $\sigma_f\rr-\ns<\epsilon$, $\vs-\sigma_f\rr<\epsilon$
\item При ризмельчении разбиения $\ns$ не может уменьшиться, $\vs$ -- увеличиться.
\item При добавлении к разбиению $\{x_k\}$ $q$ новых точек $\vs$ может уменьшиться не более чем на $(M-m)qd$, $d$ -- мелкость $\{x_k\}$. Аналогично для $\ns$.
\item Пусть $\{x_k\},\ \{y_j\}$ -- 2 разбиения $[a;b]$. $\vs,\ \ns$ и $\vs',\ \ns'$ -- их суммы Дарбу. Тогда $\ns\leq\vs'$, $\vs\geq\ns'$.
\item В силу 5), $\exists\ sup\{\ns\}=\underline{I}$ -- нижний интеграл Дарбу, $\exists\ inf\{\vs\}=\overline{I}$ -- верхний интеграл Дарбу, причем $\ns\leq\underline{I}\ \leq\ \overline{I}\leq\vs$.
\end{enumerate}
\end{theor}
\begin{proof}
\begin{enumerate}
\item Для любого разбиения $\{x_k\}$ и любой разметки $\{\xi_k\}$ $m_k\leq f(\xi_k)\leq M_k$ \then $\ns=\sum\limits^n_{k=1}m_k\Delta x_k\leq\sum\limits^n_{k=1}f(\xi_k)\Delta x_k=\sigma_f\rr\leq\sum\limits^n_{k=1}M_k\Delta x_k=\vs$
\item По определению $sup$, $\forall\ \epsilon>0$ на каждом $\xk\ \exists\ \xi_k$, что $M_k-f(\xi_k)<\frac\epsilon{b-a},\ 1\leq k\leq n$ \then $\vs-\sum\limits^n_{k=1}f(\xi_k)\Delta x_k=\sum\limits^n_{k=1}(M_k-f(\xi_k))\Delta x_k<\frac\epsilon{b-a}\sum\limits^n_{k=1}\Delta x_k=\epsilon$. Аналогично для $\ns$.
\item Достаточно доказать, что $\vs$ не увеличивается, а $\ns$ не уменьшается, при добавлении к разбиению $\{x_k\}$ 1 новой точки.

Пусть новая точка $\eta$ добавлена между $x_{k_0-1}$ и $x_{k_0}$. Расмотрим суммы Дарбу.

$\vs=M_{k_0}\Delta x_{k_0}+\sum\limits_{k\neq k_0}M_k\Delta x_k$

$\vs'=M_{k_0}^1(\eta-x_{k_0-1})+M_{k_0}^2(x_{k_0}-\eta)+\sum\limits_{k\neq k_0}M_k\Delta x_k$

Сравним эти два выражения. Заметим, что $M_{k_0}\Delta x_{k_0}=M_{k_0}(\eta-x_{k_0-1})+M_{k_0}(x_{k_0}-\eta)$. Причём очевидно, что $M_{k_0}\geq\underset{x\in[x_{k_0-1};\eta]}{sup}\{f(x)\}=M_{k_0}^1$ и $M_{k_0}\geq\underset{x\in[\eta;x_{k_0}]}{sup}\{f(x)\}=M_{k_0}^2$ \then $\vs\geq\vs'$. Аналогично для $\ns$.
\item Докажем, что при добавлении 1 новой точки к разбиению $\{x_k\}\ \vs$ может уменьшиться не более, чем на $(M-n)d$, где $M=\underset{x\in[a;b]}{sup}\{f(x)\},\ m=\underset{x\in[a;b]}{inf}\{f(x)\},\ d$ -- мелкость разбиения $\{x_k\}$.

Аналогично доказательству 3), пусть добавлена новая точка $\eta$ между $x_{k_0-1}$ и $x_{k_0}$. Рассмотрим разность $\vs-\vs'$:

$\vs-\vs'=M_{k_0}\Delta x_{k_0}-(M^1_{k_0}(\eta-x_{k_0-1})+M^2_{k_0}(x_{k_0}-\eta))=(M_{k_0}-M^1_{k_0})(\eta-x_{k_0-1})+(M_{k_0}-M^2_{k_0})(x_{k_0}-\eta)\leq(M-m)((\eta-x_{k_0-1})+(x_{k_0}-\eta))=(M-m)\Delta x_{k_0}\leq(M-m)d$
\item Пусть даны 2 любых разбиения: $\{x_k\},\ \{y_j\}$; $\{z_m\}=\{x_k\}\cup\{y_j\}$. Пусть $\vs,\ns$ -- суммы Дарбу для $\{x_k\}$, $\vs',\ns'$ -- для $\{y_j\}$, $\vs'',\ns''$ -- для $\{z_m\}$.

$\ns\leq\ns''\leq\vs''\leq\vs'$, $\ns'\leq\ns''\leq\vs''\leq\vs$.
\item Докажем, что $\ns\leq\nis\ \leq\ \vis\leq\vs$.

Предположим противное. $\vis<\nis,\ \nis-\vis=\alpha>0$. По определению $sup,\ inf$ для $\frac\alpha3\ \exists\ \ns$, что $\nis-\frac\alpha3<\ns\leq\nis;\ \exists\ \vs$, что $\vis\leq\vs<\vis+\frac\alpha3$ \then $\vs<\vis+\frac\alpha3<\nis-\frac\alpha3<\ns$ \then противоречие.
\end{enumerate}
\end{proof}
\begin{opred}
$f(x)$ называется \mybold{интегрируемой по Риману на $[a;b]$}, если $\exists\ \int\limits^b_af(x)dx$. Также используется запись $f\in \mathbb{R}[a;b]$.
\end{opred}
\section{2 критерия интегрируемости функции по Риману}
\begin{theor}Критерий интегрируемости в терминах сумм Дарбу

Для того, чтобы $f(x)$, ограниченная на $[a;b]$, была интегрируема по Риману, необходимо и достаточно, чтобы $\forall\ \epsilon>0\ \exists\ \{x_k\}$, что $\vs_f-\ns_f<\epsilon$.
\end{theor}
\begin{proof}\mybold{Необходимость.}

Пусть $\exists\ I=\int\limits^b_af(x)dx=\underset{d\mapsto0}{\lim}\ \sum\limits^n_{k=1}f(\xi_k)\Delta x_k$ \then по определению $\lim,\ \forall \epsilon>0\ \exists\ \delta=\delta(\frac\epsilon4)>0$, что для любого разбиения $\{x_k\}$ и любой его разметки $\{\xi_k\}$, если $d(\{x_k\})<\delta$ \then $|\sigma_f-I|<\frac\epsilon4$.

По 2 свойству из теоремы 3, $\exists\ \{\xi_k'\},\ \exists\ \{\xi_k''\}$, что при даном разбиении $\{x_k\}$

$\sigma_f'=\sum\limits^n_{k=1}f(\xi_k')\Delta x_k,\ \sigma_f''=\sum\limits^n_{k=1}f(\xi_k'')\Delta x_k$, удовлетворяющие неравенствам:

$\sigma_f'-\ns<\frac\epsilon4,\ \vs-\sigma_f''<\frac\epsilon4$.

При этом выполнены и неравенства:

$|\sigma_f'-I|<\frac\epsilon4,\ |\sigma_f''-I|<\frac\epsilon4$ \then $|\vs-\ns|\leq|\vs-\sigma_f''|+|\sigma''-I|+|I-\sigma'|+|\sigma'-\ns|<\frac\epsilon4+\frac\epsilon4+\frac\epsilon4+\frac\epsilon4=\epsilon$
\end{proof}
Для доказательства достаточности нам потребуется доказать следующее утверждение:
\begin{lemma}Основная лемма Дарбу.

$\nis=\underset{d\mapsto0}{\lim}\{\ns\},\ \vis=\underset{d\mapsto0}{\lim}\{\vs\}$
\end{lemma}
\begin{proof}
По определению $\vis=inf\{\vs\}$ \then $\forall\ \epsilon>0\ \exists$ разбиение $\{x_k\}$, что $\vs$ удовлетворяет неравенству:

$\vis\leq\vs<I+\frac\epsilon2$

Пусть в $\{x_k\}$ имеется $q+1$ точка. Рассмотрим теперь разбиение $\{y_j\}$ с мелкостью

$d(\{y_j\})<\frac\epsilon{2(M+m)(q-1)}=\delta(\epsilon)>0$

Рассмотрим разбиение $\{z_m\}=\{y_j\}\cup\{x_k\}$. В $\{z_m\}$, по сравнению с $\{y_j\}$, добавлено не более, чем $q-1$ точек. Обозначим верхние суммы Дарбу: $\vs'$ - для $\{y_j\}$, $\vs''$ - для $\{z_m\}$.

$\vs''\leq\vs$, $\vs''<\vs'$

$\vs'-\vs''\leq(M-m)(q-1)d<(M-m)(q-1)\frac\epsilon{2(M-m)(q-1)}=\frac\epsilon2$

$I\leq\vs''<I+\frac\epsilon2$ \then $\vis\leq\vs'<\vis+\epsilon$ \then $\vis=\underset{d\mapsto0}{\lim}\ \vs$
\end{proof}
\begin{proof}\mybold{Достаточность.}

Дано: $\forall\ \epsilon>0\ \exists\ \{x_k\}$, что $\vs-\ns<\epsilon$ \then $\ns\leq\nis\leq\vis\leq\vs$ \then $\nis=\vis=I$

По основной лемме Дарбу $\underset{d\mapsto0}{\lim}\ \ns=\underset{d\mapsto0}{\lim}\ \vs=I$

По свойству 1 теоремы 3, $\ns\leq\sigma_f(\{x_k\},\{\xi_k\})\leq\vs$ \then $\exists\ \lim\ \sigma_f(\{x_k\},\{\xi_k\})=I$, т. е. $f\in\mathbb{R}[a;b]$
\end{proof}
\begin{theor}Критерий интегрируемости в теминах верхних и нижних интегралов Дарбу

$f\in\mathbb{R}[a;b]\ \Leftrightarrow\ \nis=\vis$
\end{theor}
\begin{proof}\mybold{Необходимость.}

Пусть $f\in\mathbb{R}[a;b]$ \then по теореме 4 $\forall\ \epsilon\ \exists\ \{x_k\}$, что $\vs-\ns<\epsilon\ \Rightarrow\ \nis=\vis$

\mybold{Достаточность.}

Пусть $\nis=\vis$. По основной лемме Дарбу $\underset{d\mapsto0}{\lim}\ \ns=\underset{d\mapsto0}{\lim}\ \vs=I$. По теореме 3 $\ns\leq\sigma_f\leq\vs$. По теореме о двух милиционерах $\underset{d\mapsto0}{\lim}\ \sigma_f=I$
\end{proof}

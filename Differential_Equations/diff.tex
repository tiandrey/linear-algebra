\documentclass[draft]{report}
\usepackage[utf8]{inputenc}
\usepackage[russian]{babel}
\usepackage{amsmath}
\usepackage{amsthm}
\usepackage{amsfonts}
\usepackage{amssymb}
\usepackage{ulem}
\textwidth=190mm % 210
\oddsidemargin=-15.4mm
\textheight=267mm % 297
\topmargin=-15.4mm
\headheight=10mm
\hoffset=0mm
\voffset=-20mm
\parindent=0in

\newcommand{\sys}[1]{\left\{\begin{matrix}#1\end{matrix}\right.}
\renewcommand{\t}{\ \Rightarrow\ }
\renewcommand{\l}{\lambda}

\begin{document}
Понижение порядка.
\begin{enumerate}
\item Если в уравнение $y$ входит только начиная с $k$-й степени, то понижается заменой $z=y^k$
\item Если в уравнение не входит $x$, то считаем $y$ независимой переменной, а $y'=p(y)$ - искомой функцией
\end{enumerate}
{\bfseries Уравнения, не разрешенные относительно производной}
\begin{enumerate}
\item Если можно разрешить относительно $y$, то заменяем $y'=p(y)$
\item Если уравнение однородно относительно $y$, то заменяем $y'=z(x)y$
\end{enumerate}Линейные уравнения.
$$y'+a(x)y=b(x)\t\sys{y_o'+a(x)y_o=0\\C->C(x)}$$
Иногда можно поменять местами $x$ и $y$:
$$y=(2x+y^3)y'\t ydx-(2x+y^3)dy=0\t\cfrac{dx}{dy}-\cfrac2yx=y^2\t x'-\cfrac2yx=y^2$$
{\bfseries Уравнение Бернулли.}
$$y'+a(x)y=b(x)y^n\t\sys{z=\cfrac{1}{y^{n-1}}\\dz=(1-n)y^{-n}dy\\dy=\cfrac{1}{1-n}y^ndz}\t\cfrac{1}{1-n}y^nz'+a(x)y=b(x)y^n\t z'+(1-n)a(x)z=(1-n)b(x)\t PROFIT$$
{\bfseries Уравнение Эйлера.}
$$a_0x^ny^(n)+\ldots+a_ny=f(x)\t\sys{x=e^t,\ x>0 \\ x=-e^t,\ x<0}\t a_0\l(\l-1)\ldots(\l-n+1)+\ldots+a_{n-1}\l+a_n=0$$
{\bfseries Разложение по параметру.}
$$\sys{y'=f(x,y,\mu)\\y(x_0,\mu)=y_0(\mu)}\t y(x,\mu)=a_0(x)+a_1(x)\mu+a_2(x)\mu^2+\ldots\t\sys{a_0'=\ldots\\a_1'=\ldots\\a_2'=\ldots\\\ldots}$$
{\bfseries Функция Грина.}
$$\sys{a_0(x)y''+a_1(x)y'+a_2(x)y=f(x),\ x_0\leq x\leq x_1 \\ \alpha y'(x_0)+\beta y(x_0)=0,\ \gamma y'(x_1)+\delta y(x_1)=0}$$
Обратите внимание, что краевые условия {\bfseries нулевые}. Если в задаче они ненулевые, нужно сделать замену.

Решение такой краевой задачи можно найти с помощью функции Грина. Для начала найдем нетривиальные решения однородного уравнения $y_1,y_2$, удовлетворяющие соответственно первому и второму краевому условию.
$$G(x,s)=\sys{a(s)y_1(x)\ (x_0\leq x\leq s),\\b(s)y_2(x)\ (s\leq x\leq x_1).},\ \mbox{причем\ a\ и\ b\ такие,\ что}\ \sys{by_2(s)=ay_1(s)\\by_2'(s)=ay_1'(s)+\cfrac{1}{a_0(s)}}$$
Когда $a$ и $b$ найдены, решение выражается следующей формулой:
$$y(x)=\int\limits^{x_1}_{x_0}G(x,s)f(s)ds=\int\limits^{x}_{x_0}b(s)y_2(x)f(s)ds+\int\limits^{x_1}_{x}a(s)y_1(x)f(s)ds$$
{\bfseries Собственные решения}

Собственно, так сказать, соответственно, вооот, собственным значением задачи
$$\sys{a_0(x)y''+a_1(x)y'+a_2(x)y=\l y,\ x_0\leq x\leq x_1 \\ \alpha y'(x_0)+\beta y(x_0)=0,\ \gamma y'(x_1)+\delta y(x_1)=0}$$
называется такое число $\l$, при котором уравнение имеет решение $y(x)\not\equiv0$, удовлетворяющее краевым условиям. Это решение называется \sout{соответственной} собственной функцией.
\end{document}

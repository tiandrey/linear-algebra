\chapter{Постановка задачи}
\begin{opred}
\mybold{Системой линейных алгебраических уравнений(СЛАУ) с $n$ неизвестными $x_1,\ldots,x_n$} будем называть систему:

$\begin{cases}
a_{11}x_1+\cdots+a_{1n}x_n=b_1 \\
\cdots \\
a_{m1}x_1+\cdots+a_{mn}x_n=b_m
\end{cases}\ (1)$
\end{opred}
\begin{opred}
\mybold{Решение (1)} - это упорядоченная совокупность чисел $(c_1,\ldots,c_n)$, которые после подстановки $x_1=c_1,\ldots,x_n=c_n$ обращают каждое уравнение в тождество.
\end{opred}
\begin{opred}
Говорят, что система (1) \mybold{совместна}, если у неё есть хотя бы одно решение, и \mybold{несовместна} --- в противном случае.
\end{opred}
\begin{opred}
Говорят, что система (1) \mybold{определённая}, если у неё ровно одно решение, и \mybold{неопределённая} --- если у неё больше, чем одно решение.
\end{opred}
\begin{opred}
\mybold{Исследовать и решить систему (1)} значит
\begin{enumerate}
\item Выяснить, совместна ли она
\item В случае совместности описать множество всех решений
\end{enumerate}
\end{opred}
$\begin{pmatrix}
a_{11} & \cdots & a_{1n} \\
\vdots & \ddots & \vdots \\
a_{m1} & \cdots & a_{mn}
\end{pmatrix}=A$ --- матрица коэффициентов

$\begin{pmatrix}
x_1    \\
\vdots \\
x_m
\end{pmatrix}=x$ --- столбец неизвестных

$\begin{pmatrix}
b_1    \\
\vdots \\
b_m
\end{pmatrix}=b$ --- столбец ответов

$Ax=b$ --- матричная форма записи

$a_1x_1+\cdots+a_mx_m=b$ --- векторная форма записи
\begin{theor}(О структуре множества решений СЛАУ)

СЛАУ либо не имеет решений, либо имеет единственное решение, либо имеет бесконечное количество решений.
\end{theor}
\begin{proof}
Пусть СЛАУ имеет более 1 решения. Значит, существуют как минимум 2 различных столбца $x^{(1)},\ x^{(2)}$, что $Ax^{(1)}=b,\ Ax^{(2)}=b$

$x=\alpha x^{(1)}+(1-\alpha)x^{(2)},\ \forall\ \alpha\in\mathbb{R}$

$\vphantom{\mbox{Хуй!}}$

$\vphantom{\mbox{Хуй!}}$

Покажем:
\begin{enumerate}
\item Этот вектор-столбец всегда является решением
\item При различных $\alpha$ его значения разные
\end{enumerate}
\begin{enumerate}
\item $Ax=A(\alpha x^{(1)}+(1-\alpha)x^{(2)})=\alpha Ax^{(1)}+(1-\alpha)x^{(2)}=\alpha b+(1-\alpha)b=b$
\item $x'=\alpha'x^{(1)}+(1-\alpha')x^{(2)}$

$x''=\alpha''x^{(1)}+(1-\alpha'')x^{(2)}$

$x''-x'=(\alpha''-\alpha')x^{(1)}+(\alpha'-\alpha'')x^{(2)}=\underbrace{(\alpha'-\alpha'')}_{\scriptsize{\neq0}}\underbrace{(x^{(2)}-x^{(1)})}_{\scriptsize{\neq0}}\neq0$ \then $x''\neq x'$
\end{enumerate}
\end{proof}
\chapter{Системы с квадратной матрицей}
$Ax=b$, \rmatrix{A}{n}{n}. (1)
Пусть $|A|\neq0$. Тогда существует $A^{-1}$.
\begin{enumerate}
\item Если $x$ --- решение (1), то $A^{-1}Ax=A^{-1}b$ \then $x=A^{-1}b$
\item Рассмотрим $x=A^{-1}b$: $A(A^{-1}b)\equiv b$
\end{enumerate}
\begin{theor}
Если $|A|\neq0$, то система (1) совместна и имеет ровно 1 решение: $x=A^{-1}b$.
\end{theor}
\begin{proof}
От противного. (Очевидно, на лекциях его не было)
\end{proof}
\begin{theor}
Если $|A|\neq0$, то единственное решение (1) может быть найдено по следующим формулам (\mybold{формулам Крамера}): $x=(x_1,\ldots,x_n)^T$, где $x_i=\cfrac{|A_i|}{|A|},\ A_i$ получается из $A$ заменой $i$--го столбца на столбец ответов.
\end{theor}
\begin{proof}
$x_i=\{A^{-1}b\}_i=\sum\limits_{j=1}^n\{A^{-1}\}_{ij}b_j=\cfrac1{|A|}\sum\limits_{j=1}^n\{\hat{A}\}_{ij}b_j=\cfrac1{|A|}\sum\limits_{j=1}^nA_{ji}b_j=\cfrac{|A_i|}{|A|}$
\end{proof}
$|A|=0\ \lor$ \rmatrixof{A}{m}{n}, $m\neq n$
\begin{opred}
$Ax=0$ - \mybold{однородная} СЛАУ.
\end{opred}
Однородные СЛАУ всегда совместны.
\begin{opred}
$x=(0,\ldots,0)^T$ --- \mybold{тривиальное} решение однородной СЛАУ; $x=(x_1,\ldots,x_n)^T,\ \exists\ x_i\neq0$ --- \mybold{нетривиальное} решение однородной СЛАУ.
\end{opred}
\begin{theor}
Однородная СЛАУ с квадратной матрицей имеет нетривиальное решение тогда и только тогда, когда $|A|=0$
\end{theor}
\begin{proof}
$|A|=0\ \Leftrightarrow$ столбцы $A$ л/з $\Leftrightarrow$ $\exists\ c_1,\ldots,c_n\in\mathbb{R},\ \exists\ c_j\neq0:\ c_1a_1+\cdots+c_na_n=0$,

$x=(c_1,\ldots,c_n)^T$ --- решение системы.
\end{proof}
\begin{remark}
$Ax=0,\ |A|=0$ \then существует бесконечное количество решений.
\end{remark}
\chapter{-Системы общего вида-}
\section{-Системы с верхней трапецевидной матрицей-}
\section{-Системы с верхней ступенчатой матрицей-}
\section{-Случай общей матрицы-}
\section{-Критерий совместности и определённости СЛАУ-}
\chapter{-Геометрические свойства решений систем-}
\section{-Однородные системы-}
\section{-Случай неоднородной системы-}

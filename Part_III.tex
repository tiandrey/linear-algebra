\chapter{Постановка задачи}
\begin{opred}
\mybold{Системой линейных алгебраических уравнений(СЛАУ) с $n$ неизвестными $x_1,\ldots,x_n$} будем называть систему:

$\begin{cases}
a_{11}x_1+\cdots+a_{1n}x_n=b_1 \\
\cdots \\
a_{m1}x_1+\cdots+a_{mn}x_n=b_m
\end{cases}\ (1)$
\end{opred}
\begin{opred}
\mybold{Решение (1)} - это упорядоченная совокупность чисел $(c_1,\ldots,c_n)$, которые после подстановки $x_1=c_1;\ldots;x_n=c_n$ обращают каждое уравнение в тождество.
\end{opred}
\begin{opred}
Говорят, что система (1) \mybold{совместна}, если у неё есть хотя бы одно решение, и \mybold{несовместна} --- в противном случае.
\end{opred}
\begin{opred}
Говорят, что система (1) \mybold{определена}, если у неё ровно одно решение, и \mybold{неопределена}, если у неё несколько решений.
\end{opred}
\begin{opred}
\mybold{Исследовать и решить систему (1)} значит
\begin{enumerate}
\item Выяснить, совместна ли она
\item В случае совместности описать множество всех решений
\end{enumerate}
\end{opred}
$\begin{pmatrix}
a_{11} & \cdots & a_{1n} \\
\vdots & \ddots & \vdots \\
a_{m1} & \cdots & a_{mn}
\end{pmatrix}=A$ --- матрица коэффициентов

$\begin{pmatrix}
x_1    \\
\vdots \\
x_n
\end{pmatrix}=x$ --- столбец неизвестных

$\begin{pmatrix}
b_1    \\
\vdots \\
b_m
\end{pmatrix}=b$ --- столбец ответов

$Ax=b$ --- матричная форма записи

$a_1x_1+\cdots+a_mx_m=b$ --- векторная форма записи
\begin{theor}(О структуре множества решений СЛАУ)

СЛАУ либо не имеет решений, либо имеет единственное решение, либо имеет бесконечное количество решений.
\end{theor}
\begin{proof}
Пусть СЛАУ имеет более 1 решения. Значит, существуют как минимум 2 различных столбца $x^{(1)},\ x^{(2)}$, что $Ax^{(1)}=b,\ Ax^{(2)}=b$

$x=\alpha x^{(1)}+(1-\alpha)x^{(2)},\ \forall\ \alpha\in\mathbb{R}$

$\vphantom{\mbox{Хуй!}}$

$\vphantom{\mbox{Хуй!}}$

Покажем:
\begin{enumerate}
\item Этот вектор-столбец всегда является решением
\item При различных $\alpha$ его значения разные
\end{enumerate}
\begin{enumerate}
\item $Ax=A(\alpha x^{(1)}+(1-\alpha)x^{(2)})=\alpha Ax^{(1)}+(1-\alpha)x^{(2)}=\alpha b+(1-\alpha)b=b$
\item $x'=\alpha'x^{(1)}+(1-\alpha')x^{(2)}$

$x''=\alpha''x^{(1)}+(1-\alpha'')x^{(2)}$

$x''-x'=(\alpha''-\alpha')x^{(1)}+(\alpha'-\alpha'')x^{(2)}=\underbrace{(\alpha'-\alpha'')}_{\scriptsize{\neq0}}\underbrace{(x^{(2)}-x^{(1)})}_{\scriptsize{\neq0}}\neq0$ \then $x''\neq x'$
\end{enumerate}
\end{proof}
\chapter{Системы с квадратной матрицей}
$Ax=b$, \rmatrix{A}{n}{n}. (1)
Пусть $|A|\neq0$. Тогда существует $A^{-1}$.
\begin{enumerate}
\item Если $x$ --- решение (1), то $A^{-1}Ax=A^{-1}b$ \then $x=A^{-1}b$
\item Рассмотрим $x=A^{-1}b$: $A(A^{-1}b)\equiv b$
\end{enumerate}
\begin{theor}
Если $|A|\neq0$, то система (1) совместна и имеет ровно 1 решение: $x=A^{-1}b$.
\end{theor}
\begin{proof}
От противного. (Очевидно, на лекциях его не было)
\end{proof}
\begin{theor}
Если $|A|\neq0$, то единственное решение (1) может быть найдено по следующим формулам (\mybold{формулам Крамера}): $x=(x_1,\ldots,x_n)^T$, где $x_i=\cfrac{|A_i|}{|A|},\ A_i$ получается из $A$ заменой $i$--го столбца на столбец ответов.
\end{theor}
\begin{proof}
$x_i=\{A^{-1}b\}_i=\sum\limits_{j=1}^n\{A^{-1}\}_{ij}b_j=\cfrac{1}{|A|}\sum\limits_{j=1}^n\{\hat{A}\}_{ij}b_j=\cfrac{1}{|A|}\sum\limits_{j=1}^nA_{ji}b_j=\cfrac{|A_i|}{|A|}$
\end{proof}
$|A|=0\ \lor$ \rmatrix{A}{m}{n}, $m\neq n$
\begin{opred}
$Ax=0$ - \mybold{однородная} СЛАУ.
\end{opred}
Однородные СЛАУ всегда совместны.
\begin{opred}
$x=(0,\ldots,0)^T$ --- \mybold{тривиальное} решение однородной СЛАУ; $x=(x_1,\ldots,x_n)^T,\ \exists\ x_i\neq0$ --- \mybold{нетривиальное} решение однородной СЛАУ.
\end{opred}
\begin{theor}
Однородная СЛАУ с квадратной матрицей имеет нетривиальное решение тогда и только тогда, когда $|A|=0$
\end{theor}
\begin{proof}
$|A|=0\ \Leftrightarrow$ столбцы $A$ л/з $\Leftrightarrow$ $\exists\ c_1,\ldots,c_n\in\mathbb{R},\ \exists\ c_j\neq0:\ c_1a_1+\cdots+c_na_n=0$,

$x=(c_1,\ldots,c_n)^T$ --- решение системы.
\end{proof}
\begin{remark}
$Ax=\T,\ |A|=0$ \then существует бесконечное количество решений.
\end{remark}
\chapter{Системы общего вида}
$Ax=b$, \rmatrix{A}{m}{n}
\section{Системы с верхней трапецевидной матрицей}
$\begin{matrix}
a_{11}x_1+ & a_{12}x_2+& \cdots+ & a_{1r}x_r+ & \cdots+ & a_{1n}x_n = & b_1,   & a_{11}\neq0 \\
           & a_{22}x_2+& \cdots+ & a_{2r}x_r+ & \cdots+ & a_{2n}x_n = & b_2,   & a_{22}\neq0 \\
           &           & \ddots  &            &         &             & \vdots &             \\
           &           &         & a_{rr}x_r+ & \cdots+ & a_{rn}x_n = & b_r,   & a_{rr}\neq0
\end{matrix}$
\begin{itemize}
\item[а)] Если среди $b_{r+1},\ldots,b_m$ есть ненулевые, то эта СЛАУ несовместна.
\item[б)] Пусть $b_{r+1}=\ldots=b_m=\T$, тогда эта СЛАУ равносильна СЛАУ из первых $r$ уравнений; более того, она имеет решение.
\end{itemize}
Если $r=n$, то

$a_{rr}x_r=b_r \Rightarrow x_r=\cfrac{b_r}{a_{rr}}$

$a_{r-1,r-1}x_{r-1}+a_{r-1,r}x_r=b_{r-1}$

$x_{r-1}=\cfrac{1}{a_{r-1,r-1}}\left(b_{r-1}-\cfrac{a_{r-1,r}}{a_{rr}}b_r\right)$

и т. д.
Таким образом, построено решение.
Если $r<n$, то перенесём во всех уравнениях слагаемые с $x_{r+1},\ldots,x_n$ в правую часть. Тогда для $\forall$ наперёд заданных значений $x_{r+1},\ldots,x_n$ остальные неизвестные определяются однозначно:

$\begin{matrix}
x_1=f_1\left(x_{r+1},\ldots,x_n\right)\\
\vdots \\
x_r=f_r\left(x_{r+1},\ldots,x_n\right)
\end{matrix}$
\begin{opred}$x_{r+1},\ldots,x_n$ --- \mybold{свободные неизвестные}\end{opred}
\begin{opred}$x_{1},\ldots,x_r$ --- \mybold{главные неизвестные}\end{opred}
\section{Системы с верхней ступенчатой матрицей}
$x_{k_1}$ - номер первого ненулевого элемента в первой строчке;

...

$x_{k_r}$ - номер первого ненулевого элемента в r-той строчке;

далее действуем аналогично прошлому пункту, только роль $x_1,\ldots,x_r$ играют $x_{k_1},\ldots,x_{k_r}$.
\section{Случай общей матрицы}
$Ax=b\epofs$(ВСтупФ)$\tilde{A}x=\tilde{b}$

$Ax=b$ и $\tilde{A}x=\tilde{b}$ --- эквивалентны, если они совместны. Множество их решений совпадает.

ЭП строк приводит к эквивалентной системе уравнений.

Метод Гаусса исследования и решения СЛАУ --- очевидно.
\section{Критерий совместности и определённости СЛАУ}
$(A|b)\epofs\left(\tilde{A}|\tilde{b}\right)\ (*)$
\begin{enumerate}
\item ЭП строк расширенной матрицы приводят к эквивалентной системе.

Совместна $Ax=b\ \Longleftrightarrow$ совместна (*).
\item ЭП строк не меняют ранги основной и расширенной матриц.

Таким образом, совместность системы может быть выражена следующим образом.

Совместность (*) означает, что $rg$ основной матрицы равен количеству ненулевых строк $\tilde{A}$ и равен $rg$ расширенной матрицы $\rightarrow$ нет ненулевых элементов среди	 $b_{r+1},\ldots,b_n$.
\end{enumerate}
\begin{theor}Кронекера-Капелли (критерий совместности)

$Ax=b$ совместна $\Longleftrightarrow\ rg\ (A|b)=rg\ A$
\end{theor}
\begin{proof}
См. выше
\end{proof}
\begin{theor}Критерий определённости

Совместная система $Ax=b$ имеет единственное решение $\Longleftrightarrow\ rg\ A=n$ - количество неизвестных
\end{theor}
\begin{proof}
См. выше
\end{proof}
Следствие: система $Ax=b$ имеет единственное решение $\Longleftrightarrow\ rg\ (A|b)=rg\ A=n$

Рассмотрим $Ax=\T$, \rmatrix{A}{m}{n}
\begin{itemize}
\item[а)] Всегда совместна
\item[б)] Определена $\Longleftrightarrow\ rg\ A=n$
\end{itemize}
Информации о $n$ и $m$ недостаточно для того, чтобы сделать вывод о количестве решений системы.
\chapter{Геометрические свойства решений систем}
\section{Однородные системы}
$Ax=\T$, \rmatrix{A}{m}{n}
\begin{theor}Множество решений однородной системы $Ax=\T$ образует линейное пространство арифметических векторов из $\R^n$: $\N=\left\{x\in\R^n|Ax=\T\right\}$
\end{theor}
\begin{proof}
$\N$ - вещественное линейной пространство. Сложение, умножение на число не выводят из $\N$.
Аксиомы:\begin{enumerate}
\item *
\item *
\item $\T\in\N$, так как $Ax=\T$ всегда имеет тривиальное решение.
\item если $x\in\N$ $Ax=\T$ \then $A(-x)=\T$ \then $A(-x)\in\N$
\item *
\item 1*x=x
\item *
\item *
\end{enumerate}
* - аксиома не проверяется, так как множество всех векторов из $\R^n$ этим аксиомам удовлетворяет.

$x^{(1)},x^{(2)}\in\N\Longrightarrow\ Ax^{(1)}=\T,\ Ax^{(2)}=\T\Longrightarrow\ A(x^{(1)}+x^{(2)})=Ax^{(1)}+Ax^{(2)}=\T+\T=\T\Longrightarrow (x^{(1)}+x^{(2)})\in\N$

$A(\alpha x)=\alpha(Ax)=\alpha\T=\T\Longrightarrow (\alpha x)\in\N\ \forall\ \alpha \in \R$
\end{proof}
\begin{remark}
Другими словами, в теореме показано, что множество $\N$ является линейным подпространством в $\R^n$
\end{remark}
Пусть $Ax=\T$ имеет не только тривиальное решение. ($rg\ A\equiv r<n$)
\begin{opred}
Упорядлченная совокупность $e_1,\ldots,e_k\in\R^n$ называется \mybold{фундаментальной системой решений (ФСР)}, если:
\begin{enumerate}
\item $\forall\ j\in[1,k]\ Ae_j=\T$
\item $e_1,\ldots,e_k$ линейно независимы
\item $\forall$ решение $x\colon Ax=\T$ линейно выражается через $e_1,\ldots,e_k\colon x=\alpha_1e_1+\ldots+\alpha_ke_k$
\end{enumerate}
\end{opred}
\begin{theor}
Если в $Ax=\T\ r\equiv rg\ A<n$, то для этой системы $\exists$ ФСР, причём она состоит из $(n-r)$ векторов.
\end{theor}
\begin{proof}
Не ограничивая общности будем считать, что $x_1,\ldots,x_r$ -- главные неизвестные, а $x_{r+1},\ldots,x_n$ -- свободные неизвестные. ($n-r>0$)

$\begin{BMAT}(b,20pt,15pt){c|cccc|c}{c|c|c|c|c}
x_1\ldots x_r       & x_{r+1} & x_{r+2} & \ldots & x_n &        \\
c_{11}\ldots c_{1r} & 1       & 0       & \ldots & 0   & e_1    \\
c_{21}\ldots c_{2r} & 0       & 1       & \ldots & 0   & e_2    \\
\ldots              &         &         & \ldots & 0   & \ldots \\
c_{k1}\ldots c_{kr} & 0       & 0       & \ldots & 1   & e_k    \\
\end{BMAT}$

$k\equiv n-r$

$\alpha_1\ldots\alpha_r,\alpha_{r+1}\ldots\alpha_n$
\begin{enumerate}
\item $e_1,\ldots,e_k$ линейно независимы, так как матрица в постоенной таблице имеет ранг, равный $k$.
\item рассмотрим $\forall$ решение $Ax=\T$

По последним неизвестным $x=(\alpha_1,\ldots,\alpha_r,\alpha_{r+1},\ldots,\alpha_n) x_{r+1},\ldots,x_n$ выполнено "равенство".

$\alpha_{r+1}e_1+\ldots+\alpha_ne_k=(\alpha_1,\ldots,\alpha_n)\equiv x$

Покажем, что это равенство верно и по неизвестным $x_1,\ldots,x_r$

$y=x-(\alpha_{r+1}e_1+\ldots+\alpha_ne_k)=(\ldots,\underbrace{0,\ldots,0}_{n-r})$

Множество решений - линейное пространство \then $y$ -- решение $Ax=0$, в котором все свободные неизвестные равны нулю. Так как главные элементы определяются по свободным однозначно и $\T$ -- решение, то $y$ -- тривиальное решение, а $x=\alpha_{r+1}e_1+\ldots+\alpha_ne_k$
\end{enumerate}
\end{proof}
\begin{remark}
Конец предыдущего доказательства я не вкурил, поэтому оно может быть неправильным. Уточню на консультации.
\end{remark}
\section*{Геометрический смысл понятия ФСР}
... становится ясен, если рассмотреть однородное решение $Ax=\T,\ n=3$.

$rg\ A=0,1,2,3$
\begin{enumerate}
\item $rg\ A=3$ \then есть только тривиальное решение
\item $rg\ A=2$ \then $\exists$ ФСР из $n-r=1$ вектора $e_1\colon e_1\neq\T,\ e_1=(\alpha_1,\alpha_2,\alpha_3)$; общее решение имеет вид $x=\alpha_1e_1,\ \forall \ \alpha_1\in\R$ --- это уравнение прямой, проходящей через начало координат.
\item $rg\ A=1$ \then $\exists$ ФСР из $n-r=2$ векторов $e_1,e_2$ -- линейно независимы, т.е. компланарны; общее решение $x=\alpha_1e_1+\alpha_2e_2$, это уравнение плоскости, проходящей через начало координат.
\item $rg\ A=0$ \then $\T x=\T\ \Longleftrightarrow$ множество решений -- $\R^3$, т. е. все пространство.
\end{enumerate}
\section{Неоднородные системы}
$Ax=b,\ A \in \R^{m\times n},\ b\neq\T$, система совместна.
Множество решений $\M=\left\{x\in\R^m|Ax=b\right\}$ не образует линейное пространство:

$x^{(1)}, x^{(2)}\colon Ax^{(1)}=b,\ Ax^{(2)}=b\Longrightarrow A(x^{(1)}+x^{(2)})=Ax^{(1)}+Ax^{(2)}=b+b=2b\neq b$
\begin{theor}
$\M=\N+\xs$ --- множество решений неоднородной СЛАУ; $\N=\left\{x\in\R^n|Ax=\T\right\}$; $A\xs=b$
\end{theor}
\begin{proof}
$\xs+\N\subset\M\colon\forall\ x\in\ \xs+\N\colon\xs+\dot{x},\ A\dot{x}=\T,\  Ax=A(\xs+\dot{x})=A\xs+A\dot{x}=b+\T=b\ \Longrightarrow\ x\in\N$

$M\subset\xs+\N\colon\forall\ x\in\M\colon Ax=b\colon x=\xs+(x-\xs),\ A(x-\xs)=Ax-A\xs=b-b=\T\Longrightarrow x-\xs\in\N$
\end{proof}
ФСР для неоднородной системы $Ax=b,\ b\neq\T$ вводить не имеет смысла.

Однако для описания множества $\left\{x|Ax=b\right\}$ можно использовать ФСР однородной системы $\left\{x|Ax=\T\right\}$

$\M=\N+\xs\ \Longrightarrow x=\xs+\alpha_1e_1+\ldots+\alpha_ke_k;\ x$ -- частное решение $Ax=b;\ \alpha_1e_1+\ldots+\alpha_ke_k$ -- линейная комбинация ФСР $Ax=\T$.

\documentclass{report}
\usepackage[utf8]{inputenc}
\usepackage[russian]{babel}
\usepackage{amsmath}
\usepackage{amsthm}
\usepackage{amsfonts}
\usepackage{amssymb}
\usepackage{dsfont}
\usepackage{euscript}
\newtheorem*{lemma}{Лемма}
\newtheorem*{theor}{Теорема}
\newtheorem*{opred}{Определение}
\newcommand{\A}{\mathcal{A}}
\renewcommand{\a}{\alpha}
\newcommand{\B}{\EuScript{B}}
\newcommand{\D}{\EuScript{D}}
\newcommand{\E}{\EuScript{E}}
\newcommand{\J}{\mathcal{J}}
\newcommand{\K}{\EuScript{K}}
\newcommand{\o}{\sigma}
\renewcommand{\P}{\mathcal{P}}
\newcommand{\R}{\mathbb{R}}
\newcommand{\w}{\omega}
\newcommand{\W}{\Omega}
\newcommand{\1}{\mathds{1}}
\textwidth=190mm % 210
\oddsidemargin=-15.4mm
\textheight=267mm % 297
\topmargin=-15.4mm
\headheight=10mm
\hoffset=0mm
\voffset=-20mm
\parindent=0in
\begin{document}
\chapter*{Случайные величины}
\section*{Определение}
Рассмотрим вероятностное пространство $(\W, \A, \P)$.

$\K$ - некоторый класс подмножеств $\W$.
\begin{opred}
Сигма-алгебра $\o(\K)$ называется порождающим классов $\K$, если
\begin{enumerate}
\item $\K\subseteq\o(\K)$
\item среди всех таких сигма-алгебр эта алгебра минимальная.
\end{enumerate}
\end{opred}
\begin{opred}
Борелевской сигма-алгеброй $(\B)$ подмножества вещественной прямой называется минимальная сигма-алгебра, содержащая все интервалы $\J=\{(a,b):a\in\R, b>a\}$

$\B=\o(\J)$
\end{opred}
\begin{opred}
Отображение $\xi:\W\mapsto\R$ называется случайной величиной, если $\forall B\in\B$ $\xi^{-1}(B)=\{\w:\xi(\w)\in B\}\in\A$
\end{opred}
\section*{Примеры случайных величин}
\subsection*{Индикатор}
$A\in\A$

$\xi(\w)=\left\{\begin{matrix}
1, \w\in A\\
0, \w\notin A
\end{matrix}\right.=\1_A(\w)$
\subsection*{Дискретная случайная величина}
$x_1,x_2,\ldots,x_n\colon x_i\in\R$

$A_1,A_2,\ldots,A_n\colon A_i\in\A; A_i\cap A_j=\emptyset$

$\xi(\w)=\sum\limits_{i=1}^nx_i\1_{A_i}(\w)$

Если слагаемых конечное количество, то это простая случайная величина.
\section*{Пример использования}
\paragraph{Задача} $\phantom{1}$

Привести пример функции $\xi$ такой, что $\xi(\w)$ - не случайная величина, а $\xi^2(\w)$ - случайная величина.
\paragraph{Решение} $\phantom{1}$

$\A\neq2^\W, A\notin\A$

$\xi(w)=\1_A(\w)$ - не случайная величина:

$B \ni 1, -1\notin B$

$\xi^{-1}(B)=A\notin\A$

А $\xi^2(\w)$ - случайная величина:

$\xi^2(\w)=1\  \forall\w\in\W$

$B\ni1\  \Rightarrow\ {\xi^2}^{-1}(B)=\W\in\A$

$B\not\ni1\  \Rightarrow\ {\xi^2}^{-1}(B)=\emptyset\in\A$
\section*{Утверждение}
Пусть $\E$ - некоторая система подмножеств $\R:\E:\o(\E)=\B$

Тогда для того, чтобы $\xi(\w)$ была случайной величиной, необходимо и достаточно, чтобы

$\forall E\in\E\ \xi^{-1}(E)=\{\w:\xi(\w)\in E\}\in\A$
\begin{proof}$\phantom{1}$

$\Rightarrow$: очевидно

$\Leftarrow$:

$\D=\{D:D\in\B,\ \xi^{-1}(D)\in\A\}$

$\xi^{-1}(\underset{\a}{\cup}$
\end{proof}
\end{document}
